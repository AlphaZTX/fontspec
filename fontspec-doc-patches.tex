
%%%%%%%%%%%%%%%%%%%%%%%%%%%%%%%%%%%%%%%%%%%%%%%%%%%%%%%%%%%%%%%%%%%%%%%%%%%%

\part{The `improvement' of \LaTeXe\ and other packages}
\label{sec:patching}

This part of the package code contains patches to various
\LaTeX\ components and third-party packages to improve the default
behaviour.

\section{Verbatim}
\label{sec:verb}

Many verbatim mechanisms assume the existence of a `visible space' character that exists in the \textsc{ascii} space slot of the typewriter font. This character is known in Unicode as \unichar{2423}{box open}, which looks like this: `\verb*| |'.

When a Unicode typewriter font is used, \LaTeX\ no longer prints visible spaces for the |verbatim*| environment and |\verb*| command. This problem is fixed by using the correct Unicode glyph, and the following packages are patched to do the same:
\pkg{listings}, \pkg{fancyvrb}, \pkg{moreverb}, and \pkg{verbatim}.

In the case that the typewriter font does not contain `\verb*| |', the Latin Modern Mono font is used as a fallback.

\section{Discretionary hyphenation: \cmd\-}
\label{sec:hyphen}

\LaTeX\ defines the macro \cmd\-\ to insert discretionary hyphenation points.
However, it is hard-coded in \LaTeX\ to use the hyphen |-| character. Since \pkg{fontspec}
makes it easy to change the hyphenation character on a per font basis, it would
be nice if \cmd\-\ adjusted automatically --- and now it does.

\section{Commands for old-style and lining numbers}

\DescribeMacro{\oldstylenums}
\DescribeMacro{\liningnums}
\LaTeX's definition of \cs{oldstylenums} relies on strange font encodings.
We provide a \pkg{fontspec}-compatible alternative and while we're at it
also throw in the reverse option as well. Use \cs{oldstylenums}\marg{text}
to explicitly use old-style (or lowercase) numbers in \meta{text}, and
the reverse for \cs{liningnums}\marg{text}.

