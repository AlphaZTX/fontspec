
% This package should work with any encoding name.
% Simply define an expansion for \UTFencname before loading this file,
% otherwise the encoding name will be 'U'.
\providecommand{\UTFencname}{U}

% Use \DeclareUTFcharacter to assign a cs-name to
% access a Unicode code-point...
\newcommand{\DeclareUTFcharacter}[3][\UTFencname]{%
  \let\add@flag\@ne % ==> add support in this encoding
  \check@hexcom@digits #2@@@@@!@{#1}{#2}{#3}%
}

% ... or use \UndeclareUTFcharacter to cancel a declaration
% when the appropriate code-point is not supported in the
% desired text-font.
\newcommand{\UndeclareUTFcharacter}[3][\UTFencname]{%
  \let\add@flag\z@ % ==> remove support in this encoding
  \check@hexcom@digits #2@@@@@!@{#1}{#2}{#3}%
}

\def\check@hexcom@digits#1#2@!@#3#4#5{%
 \ifx x#1\relax
  \check@hexcom@digits@#2@!@{#3}{#4}{#5}%
 \else
  \UTFacc@warning@{code #4 for #3-\string#5 fails to start with 'x'}%
 \fi
}

% Use \DeclareUTFcomposite to assign a cs-name to access
% accents or composite characters via Unicode code-points,
% or the Unicode "Composing Character" mechanism ...
\newcommand{\DeclareUTFcomposite}[4][\UTFencname]{{%
  \let\add@flag\@ne % ==> add support in this encoding
  \check@hex@digits #2@@@@@!@{#1}{#2}{#3}{#4}%
}}
\newcommand{\DeclareUTFmulticomposite}[4][\UTFencname]{{%
  \let\add@flag\@ne % ==> add support in this encoding
  \check@hex@digits #2@@@@@!@{#1}{#2}{#3}{#4}%
}}

% ... or use \UndeclareUTFcomposite to cancel a declaration
% when the appropriate code-point is not supported in the
% desired text-font.
\newcommand{\UndeclareUTFcomposite}[4][\UTFencname]{{%
  \let\add@flag\z@ % ==> remove support in this encoding
  \check@hex@digits #2@@@@@!@{#1}{#2}{#3}{#4}%
}}


\def\check@hex@digits#1#2@!@#3#4#5#6{%
 \ifx x#1\relax
   \check@hex@digits@#2@!@{#3}{#4}{#5}{#6}%
 \else
  \UTFacc@warning@{code #4 for #3-\string#5#6 fails to start with 'x'}%
 \fi
}

% indirect conditionals, to avoid unbalance when reloaded
\def\UTF@ignore#1{\csname iffalse\endcsname}
\def\UTF@doit#1{\csname iftrue\endcsname}


%% these next macros need to have " with correct \catcode
%%
{\catcode`\"=12
%
\gdef\check@hexcom@digits@#1#2#3#4#5@!@#6#7#8{%
 \ifx @#4\relax
  \UTFacc@warning@{insufficient hex digits #7 for #6-\string#8}%
 \else
  \ifcat \active\noexpand#8%
   \ifx\add@flag\@ne %
    \expandafter\def\csname\UTFencname\string#8\endcsname{\char"#1#2#3#4\relax}%
    \ifx\unDeFiNed@#8%
     \ifx\cf@encoding\UTFencname
      \DeclareTextCommand{#8}{OT1}{\undefined}%
     \else
      \DeclareTextCommand{#8}{\cf@encoding}{\undefined}%
     \fi
    \else {% macro #8 exists already ...
      \let\protect\noexpand
      \edef\UTF@testi{#8}\def\UTF@testii{#8}%
      \ifx\UTF@testi\UTF@testii\aftergroup\UTF@ignore
      \else\aftergroup\UTF@doit\fi
     }%
     \iffalse
      % ... but when it isn't robust, make it so
      \expandafter\let\csname?-\string#8\endcsname#8\relax
      \edef\next@UTF@{{\cf@encoding}%
        {\expandafter\noexpand\csname?-\string#8\endcsname}}%
      \expandafter\DeclareTextCommand\expandafter
         {\expandafter#8\expandafter}\next@UTF@
     \fi
    \fi %
   \else % \add@flag \z@
    \expandafter\global\expandafter
      \let\csname\UTFencname\string#8\endcsname\relax
   \fi % end of \add@flag switch
  \else % not active catcode --- shouldn't happen
  % \typeout{*** did you really mean #8 ? ***}%
   \ifx\add@flag\@ne %
    \edef\tmp@name{\expandafter\string\csname\UTFencname\endcsname
      \expandafter\string\csname#8\endcsname}%
    \expandafter\def\csname\tmp@name\endcsname{\char"#1#2#3#4\relax}%
    \ifx\cf@encoding\UTFencname
     \expandafter\DeclareTextCommand\expandafter
       {\csname#8\endcsname}{OT1}{\undefined}%
    \else
     \expandafter\DeclareTextCommand\expandafter
       {\csname#8\endcsname}{\cf@encoding}{\undefined}%
    \fi
   \else % \add@flag \z@
    \expandafter\global\expandafter\let\csname#8\endcsname\relax
   \fi % end of \add@flag switch
  \fi % end of \ifcat
 \fi}
\gdef\check@hex@digits@#1#2#3#4#5@!@#6#7#8#9{%
 \ifx @#4\relax
  \UTFacc@warning@{insufficient hex digits #7 for #6-\string#8#9}%
 \else
  \def\UTFchar{\char"#1#2#3#4\relax}%
  \expandafter\expandafter\expandafter\declare@utf@composite
  \expandafter\expandafter\expandafter
   {\expandafter\csname#6\endcsname}{\UTFchar}{#8}{#9}\relax
 \fi}
%\gdef\add@UTF@accent#1#2#3{#2\char"#1\relax}
\gdef\add@UTF@accent#1#2#3{\ifx\relax#2\relax\char"#3\else
 \ifx\ #2\relax\char"#3\else
 \expandafter\ifx\UTF@space#2\relax\char"#3\else
 \ifx~#2\char"#3\else#2\char"#1\fi\fi\fi\fi\relax}
\gdef\add@UTF@accents#1#2#3{#2\char"#1\char"#3\relax}
\gdef\add@set@accentCOMP#1#2#3{\add@accent{"#1}{#2}}
\gdef\add@set@accentMOD#1#2#3{\add@accent{"#3}{#2}}
\gdef\declare@hex@command#1#2{\gdef#2{#1}}%
%
}%  end of \catcode`\"=12


{\catcode`\ =10\relax%
\gdef\UTF@@space{ }}
\edef\UTF@space{\UTF@@space}


\def\declare@utf@composite#1#2#3#4{%
 \expandafter\ifcat\expandafter A\string#4\relax
  {\ifx\add@flag\@ne %
   \expandafter\xdef\csname\string#1\string#3-#4\endcsname{#2}%
  \else
   \expandafter\global\expandafter
    \let\csname\string#1\string#3-#4\endcsname\relax
  \fi}%
 \else
  {\ifx\add@flag\@ne %
   \expandafter\xdef\csname\string#1\string#3-\string#4\endcsname{#2}%
  \else
   \expandafter\global\expandafter
    \let\csname\string#1\string#3-\string#4\endcsname\relax
  \fi}%
 \fi
}

% new command:  {\DeclareEncodedCompositeCharacter}[4]{%
  %  #1 = encoding
  %  #2 = accent-macro in TeX
  %  #3 = position of combining glyph in Unicode
  %  #4 = bare accent position, in Unicode
  %  ##1 = slot for the accented letter
\newcommand{\DeclareEncodedCompositeCharacter}[4]{%
  \expandafter\def\expandafter\next@ii\expandafter{%
   \expandafter\expandafter\expandafter\@text@composite\expandafter
    \csname #1\string#2\endcsname####1\@empty
     \@text@composite{\add@encoded@accent{#3}{####1}{#4}}}%
  \expandafter\def\expandafter\next@i\expandafter{\expandafter\expandafter
   \expandafter\def\expandafter\csname #1\string#2\endcsname####1}%
  \expandafter\next@i\expandafter{\next@ii}%
}

\newcommand{\DeclareEncodedCompositeAccents}[4]{%
  \expandafter\def\expandafter\next@ii\expandafter{%
   \expandafter\expandafter\expandafter\@text@composite\expandafter
    \csname #1\string#2\endcsname####1\@empty
     \@text@composite{\add@encoded@accent{#4}{####1}{#3}}}%
  \expandafter\def\expandafter\next@i\expandafter{\expandafter\expandafter
   \expandafter\def\expandafter\csname #1\string#2\endcsname####1}%
  \expandafter\next@i\expandafter{\next@ii}%
}

\let\add@encoded@accent\add@UTF@accent
\let\add@encoded@accents\add@UTF@accents

% bring \textsuperscript and \textsubscript into the fold of macros 
% dependent on encoding
\let\realLaTeXsuperscript\textsuperscript
\let\realLaTeXsubscript\textsubscript
\DeclareTextAccent{\textsuperscript}{OT1}{999}
\expandafter\expandafter\expandafter\let\expandafter
 \csname?\string\textsuperscript\endcsname\realLaTeXsuperscript
\DeclareTextAccent{\textsubscript}{OT1}{999}
\expandafter\expandafter\expandafter\let\expandafter
 \csname?\string\textsubscript\endcsname\realLaTeXsubscript
\let\super\textsuperscript

\endinput
