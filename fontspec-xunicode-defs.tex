% This package should work with any encoding name.
% Simply define an expansion for \UTFencname before loading this file,
% otherwise the encoding name will be 'U'.
\providecommand{\UTFencname}{U}
%
%
%
%
% Use \DeclareUTFcharacter to assign a cs-name to
% access a Unicode code-point...
%
\newcommand{\DeclareUTFcharacter}[3][\UTFencname]{%
  \let\add@flag\@ne % ==> add support in this encoding
  \check@hexcom@digits #2@@@@@!@{#1}{#2}{#3}%
}
%
% ... or use \UndeclareUTFcharacter to cancel a declaration
% when the appropriate code-point is not supported in the
% desired text-font.
%
\newcommand{\UndeclareUTFcharacter}[3][\UTFencname]{%
  \let\add@flag\z@ % ==> remove support in this encoding
  \check@hexcom@digits #2@@@@@!@{#1}{#2}{#3}%
}
%
%
%
\def\check@hexcom@digits#1#2@!@#3#4#5{%
 \ifx x#1\relax
  \check@hexcom@digits@#2@!@{#3}{#4}{#5}%
 \else
  \UTFacc@warning@{code #4 for #3-\string#5 fails to start with 'x'}%
 \fi
}
%
% Use \DeclareUTFcomposite to assign a cs-name to access
% accents or composite characters via Unicode code-points,
% or the Unicode "Composing Character" mechanism ...
%
\newcommand{\DeclareUTFcomposite}[4][\UTFencname]{{%
  \let\add@flag\@ne % ==> add support in this encoding
  \check@hex@digits #2@@@@@!@{#1}{#2}{#3}{#4}%
}}
\newcommand{\DeclareUTFmulticomposite}[4][\UTFencname]{{%
  \let\add@flag\@ne % ==> add support in this encoding
  \check@hex@digits #2@@@@@!@{#1}{#2}{#3}{#4}%
}}
%
% ... or use \UndeclareUTFcomposite to cancel a declaration
% when the appropriate code-point is not supported in the
% desired text-font.
%
\newcommand{\UndeclareUTFcomposite}[4][\UTFencname]{{%
  \let\add@flag\z@ % ==> remove support in this encoding
  \check@hex@digits #2@@@@@!@{#1}{#2}{#3}{#4}%
}}
%
% Allow already defined math-commands to be used also in text.
% Use of this is controlled by the 'mathastext' package option.
\newcommand\DeclareMathAsUTFtext[3]{%
 \expandafter\let\csname keepmathUTF#1\endcsname#2\relax
 \DeclareUTFcharacter[\UTFencname]{#3}{#2}%
 \expandafter\let\csname keeptextUTF#1\expandafter\endcsname
  \csname\UTFencname\string#2\endcsname\relax
 \DeclareTextCommand{#2}{\UTFencname}{%
  \ifmmode % the saved math version
   \expandafter\mathrm\csname keepmathUTF#1\endcsname
  \else   % the text version
   {\csname keeptextUTF#1\endcsname}%
 \fi}%
}
% useage: \DeclareMathAsUTFtext{aleph}{\aleph}{x2135}
%
%
%
%
\def\check@hex@digits#1#2@!@#3#4#5#6{%
 \ifx x#1\relax
   \check@hex@digits@#2@!@{#3}{#4}{#5}{#6}%
 \else
  \UTFacc@warning@{code #4 for #3-\string#5#6 fails to start with 'x'}%
 \fi
}
%
% indirect conditionals, to avoid unbalance when reloaded
\def\UTF@ignore#1{\csname iffalse\endcsname}
\def\UTF@doit#1{\csname iftrue\endcsname}
%
%%
%% these next macros need to have " with correct \catcode
%%
{\catcode`\"=12
%
\gdef\check@hexcom@digits@#1#2#3#4#5@!@#6#7#8{%
 \ifx @#4\relax
  \UTFacc@warning@{insufficient hex digits #7 for #6-\string#8}%
 \else
  \ifcat \active\noexpand#8%
   \ifx\add@flag\@ne %
    \expandafter\def\csname\UTFencname\string#8\endcsname{\char"#1#2#3#4\relax}%
    \ifx\unDeFiNed@#8%
     \ifx\cf@encoding\UTFencname
      \DeclareTextCommand{#8}{OT1}{\undefined}%
     \else
      \DeclareTextCommand{#8}{\cf@encoding}{\undefined}%
     \fi
    \else {% macro #8 exists already ...
      \let\protect\noexpand
      \edef\UTF@testi{#8}\def\UTF@testii{#8}%
      \ifx\UTF@testi\UTF@testii\aftergroup\UTF@ignore
      \else\aftergroup\UTF@doit\fi
     }%
     \iffalse
      % ... but when it isn't robust, make it so
      \expandafter\let\csname?-\string#8\endcsname#8\relax
      \edef\next@UTF@{{\cf@encoding}%
        {\expandafter\noexpand\csname?-\string#8\endcsname}}%
      \expandafter\DeclareTextCommand\expandafter
         {\expandafter#8\expandafter}\next@UTF@
     \fi
    \fi %
   \else % \add@flag \z@
    \expandafter\global\expandafter
      \let\csname\UTFencname\string#8\endcsname\relax
   \fi % end of \add@flag switch
  \else % not active catcode --- shouldn't happen
  % \typeout{*** did you really mean #8 ? ***}%
   \ifx\add@flag\@ne %
    \edef\tmp@name{\expandafter\string\csname\UTFencname\endcsname
      \expandafter\string\csname#8\endcsname}%
    \expandafter\def\csname\tmp@name\endcsname{\char"#1#2#3#4\relax}%
    \ifx\cf@encoding\UTFencname
     \expandafter\DeclareTextCommand\expandafter
       {\csname#8\endcsname}{OT1}{\undefined}%
    \else
     \expandafter\DeclareTextCommand\expandafter
       {\csname#8\endcsname}{\cf@encoding}{\undefined}%
    \fi
   \else % \add@flag \z@
    \expandafter\global\expandafter\let\csname#8\endcsname\relax
   \fi % end of \add@flag switch
  \fi % end of \ifcat
 \fi}
\gdef\check@hex@digits@#1#2#3#4#5@!@#6#7#8#9{%
 \ifx @#4\relax
  \UTFacc@warning@{insufficient hex digits #7 for #6-\string#8#9}%
 \else
  \def\UTFchar{\char"#1#2#3#4\relax}%
  \expandafter\expandafter\expandafter\declare@utf@composite
  \expandafter\expandafter\expandafter
   {\expandafter\csname#6\endcsname}{\UTFchar}{#8}{#9}\relax
 \fi}
%\gdef\add@UTF@accent#1#2#3{#2\char"#1\relax}
\gdef\add@UTF@accent#1#2#3{\ifx\relax#2\relax\char"#3\else
 \ifx\ #2\relax\char"#3\else
 \expandafter\ifx\UTF@space#2\relax\char"#3\else
 \ifx~#2\char"#3\else#2\char"#1\fi\fi\fi\fi\relax}
\gdef\add@UTF@accents#1#2#3{#2\char"#1\char"#3\relax}
\gdef\add@set@accentCOMP#1#2#3{\add@accent{"#1}{#2}}
\gdef\add@set@accentMOD#1#2#3{\add@accent{"#3}{#2}}
\gdef\declare@hex@command#1#2{\gdef#2{#1}}%
%
}%  end of \catcode`\"=12
%
{\catcode`\ =10\relax%
\gdef\UTF@@space{ }}%
\edef\UTF@space{\UTF@@space}
%
\def\declare@utf@composite#1#2#3#4{%
 \expandafter\ifcat\expandafter A\string#4\relax
  {\ifx\add@flag\@ne %
   \expandafter\xdef\csname\string#1\string#3-#4\endcsname{#2}%
  \else
   \expandafter\global\expandafter
    \let\csname\string#1\string#3-#4\endcsname\relax
  \fi}%
 \else
  {\ifx\add@flag\@ne %
   \expandafter\xdef\csname\string#1\string#3-\string#4\endcsname{#2}%
  \else
   \expandafter\global\expandafter
    \let\csname\string#1\string#3-\string#4\endcsname\relax
  \fi}%
 \fi
}
%
% new command:  {\DeclareEncodedCompositeCharacter}[4]{%
  %  #1 = encoding
  %  #2 = accent-macro in TeX
  %  #3 = position of combining glyph in Unicode
  %  #4 = bare accent position, in Unicode
  %  ##1 = slot for the accented letter
%\newcommand{\DeclareEncodedCompositeCharacter}[4]{%
%  \expandafter\def\csname #1\string#2\endcsname##1{%
%    \expandafter\@text@composite \csname #1\string#2\endcsname##1\@empty
%    \@text@composite{\add@encoded@accent{#3}{##1}{#4}}}%
%}
\newcommand{\DeclareEncodedCompositeCharacter}[4]{%
  \expandafter\def\expandafter\next@ii\expandafter{%
   \expandafter\expandafter\expandafter\@text@composite\expandafter
    \csname #1\string#2\endcsname####1\@empty
     \@text@composite{\add@encoded@accent{#3}{####1}{#4}}}%
  \expandafter\def\expandafter\next@i\expandafter{\expandafter\expandafter
   \expandafter\def\expandafter\csname #1\string#2\endcsname####1}%
  \expandafter\next@i\expandafter{\next@ii}%
}
%\newcommand{\DeclareEncodedCompositeAccents}[4]{%
%  \expandafter\def\csname #1\string#2\endcsname##1{%
%    \expandafter\@text@composite \csname #1\string#2\endcsname##1\@empty
%    \@text@composite{\add@encoded@accents{#4}{##1}{#3}}}%
%}
\newcommand{\DeclareEncodedCompositeAccents}[4]{%
  \expandafter\def\expandafter\next@ii\expandafter{%
   \expandafter\expandafter\expandafter\@text@composite\expandafter
    \csname #1\string#2\endcsname####1\@empty
     \@text@composite{\add@encoded@accent{#4}{####1}{#3}}}%
  \expandafter\def\expandafter\next@i\expandafter{\expandafter\expandafter
   \expandafter\def\expandafter\csname #1\string#2\endcsname####1}%
  \expandafter\next@i\expandafter{\next@ii}%
}

\let\add@encoded@accent\add@UTF@accent
\let\add@encoded@accents\add@UTF@accents
%\let\add@encoded@accent\add@set@accentCOMP
%\let\add@encoded@accent\add@set@accentMOD
%
% bring \textsuperscript and \textsubscript into the fold of macros 
% dependent on encoding
%
\let\realLaTeXsuperscript\textsuperscript
\let\realLaTeXsubscript\textsubscript
\DeclareTextAccent{\textsuperscript}{OT1}{999}
\expandafter\expandafter\expandafter\let\expandafter
 \csname?\string\textsuperscript\endcsname\realLaTeXsuperscript
\DeclareTextAccent{\textsubscript}{OT1}{999}
\expandafter\expandafter\expandafter\let\expandafter
 \csname?\string\textsubscript\endcsname\realLaTeXsubscript
\let\super\textsuperscript
%

%: separator

% need to patch the accents for use with TIPA's T3 encoded letters
% and shorthand for double-accents

\input t3enc.def

\let\realLaTeXacute\'    \def\tipaacuteaccent{\TIPAaccent{\realLaTeXacute}}
\let\realLaTeXgrave\`    \def\tipagraveaccent{\TIPAaccent{\realLaTeXgrave}}
\let\realLaTeXcircum\^   \def\tipacircumaccent{\TIPAaccent{\realLaTeXcircum}}
\let\realLaTeXumlaut\"   \def\tipaumlautaccent{\TIPAaccent{\realLaTeXumlaut}}
\let\realLaTeXmacron\=   \def\tipamacronaccent{\TIPAaccent{\realLaTeXmacron}}
\let\realLaTeXtilde\~    \def\tipatildeaccent{\TIPAaccent{\realLaTeXtilde}}
\let\realLaTeXdot\.      \def\tipadotaccent{\TIPAaccent{\realLaTeXdot}}
\let\realLaTeXring\r     \def\tiparingaccent{\TIPAaccent{\realLaTeXring}}
\let\realLaTeXbreve\u    \def\tipabreveaccent{\TIPAaccent{\realLaTeXbreve}}
\let\realLaTeXcaron\v    \def\tipacaronaccent{\TIPAaccent{\realLaTeXcaron}}
\let\realLaTeXogonek\k   \def\tipaogonekaccent{\TIPAaccent{\realLaTeXogonek}}
\let\realLaTeXhungar\H   \def\tipahungaraccent{\TIPAaccent{\realLaTeXhungar}}
\let\realLaTeXcedilla\c  \def\tipacedillaaccent{\TIPAaccent{\realLaTeXcedilla}}
\let\realLaTeXmisc\m     \def\tipamiscaccent{\TIPAaccent{\realLaTeXmisc}}
\let\realLaTeXtie\t      \def\tipatieaccent{\TIPAaccent{\realLaTeXtie}}
\let\realLaTeXvert\|     \def\tipavertaccent{\TIPAaccent{\TIPAvertmacro}}%

\let\realLaTeXmathcolon\:      \def\tipacolonmacro{%
  \ifmmode\expandafter\realLaTeXmathcolon\else\expandafter\TIPAcolonmacro\fi}
\let\realLaTeXmathsemicolon\:  \def\tipasemicolonmacro{%
  \ifmmode\expandafter\realLaTeXmathsemicolon\else\expandafter\TIPAsemicolonmacro\fi}
\let\realLaTeXmathstar\*       \def\tipastarmacro{%
  \ifmmode\expandafter\realLaTeXmathstar\else\expandafter\TIPAstarmacro\fi}
\let\realLaTeXmathexclam\!     \def\tipaexclammacro{%
  \ifmmode\expandafter\realLaTeXmathexclam\else\expandafter\TIPAexclammacro\fi}
%\let\realLaTeXmathvert\|        \def\tipavertmacro{%
%  \ifmmode\expandafter\realLaTeXmathvert\else\expandafter\TIPAvertmacro\fi}


\DeclareTextAccent{\TIPAcolonmacro}{OT1}{999}
\DeclareTextAccent{\TIPAsemicolonmacro}{OT1}{999}
\DeclareTextAccent{\TIPAstarmacro}{OT1}{999}
\DeclareTextAccent{\TIPAexclammacro}{OT1}{999}
\DeclareTextAccent{\TIPAvertmacro}{OT1}{999}

\def\setupTIPAaccents{%
  \let\'\tipaacuteaccent
  \let\^\tipacircumaccent
  \let\`\tipagraveaccent
  \let\~\tipatildeaccent
  \let\=\tipamacronaccent
  \let\"\tipaumlautaccent
  \let\.\tipadotaccent
  \let\r\tiparingaccent
  \let\k\tipaogonekaccent
  \let\c\tipacedillaaccent
  \let\u\tipabreveaccent
  \let\v\tipawedgeaccent
  \let\H\tipahungaraccent
  \let\t\tipatieaccent
  \let\m\tipamiscaccent
  \let\|\tipavertaccent
%
  \let\s\textsyllabic
  \let\:\tipacolonmacro
  \let\;\tipasemicolonmacro
  \let\*\tipastarmacro
  \let\!\tipaexclammacro
}


\def\TIPAaccent#1#2{\bgroup
  \def\donextaccent{\egroup#1#2}%
%  \def\doexpandaccent{\egroup\expandafter#1#2}%
  \def\doexpandaccentchar{\egroup\expandafter#1#2}%
  \def\doexpandaccentgroup{\egroup\expandafter#1\expandafter{#2}}%
  \def\tmpa{#2}\expandafter\def\expandafter\tmpb\expandafter{#2}%
  \ifx\tmpa\tmpb\else
    % have to do more here for nested accents !!! 
    \expandafter\def\expandafter\tmp\expandafter{\string#2}%
    \expandafter\testforslash\tmp\$!\$%
  \fi \donextaccent }
\def\testforslash#1#2\$!\${\def\tmp{#1}%
  \ifx\tmp\TIPAbareslash
   \ifx\relax#2\relax\else
    % for nested accents or macros
    \let\donextaccent\doexpandaccentgroup
   \fi
  \else
    % for active characters
   \let\donextaccent\doexpandaccentchar
  \fi}
\def\catchbareslash#1#2\$!\${\def\TIPAbareslash{#1}}
\edef\next{\string\ }
\expandafter\catchbareslash\next\$!\$


% This assumes that the shorthands for double-accents
% are used only within  \tipatext{...} portions:

\def\tipasubacuteaccent{\TIPAaccent{\textsubacute}}
\def\tipadotacuteaccent{\TIPAaccent{\textdotacute}}
\def\tipaacutemacronaccent{\TIPAaccent{\textacutemacron}}
\def\tipasubgraveaccent{\TIPAaccent{\textsubgrave}}
\def\tipasubcircumaccent{\TIPAaccent{\textsubcircum}}
\def\tipasubtildeaccent{\TIPAaccent{\textsubtilde}}
\def\tipasubdotaccent{\TIPAaccent{\textsubdot}}
\def\tipacircumdotaccent{\TIPAaccent{\textcircumdot}}
\def\tipatildedotaccent{\TIPAaccent{\texttildedot}}
\def\tipagravedotaccent{\TIPAaccent{\textgravedot}}
\def\tipagravemacronaccent{\TIPAaccent{\textgravemacron}}
\def\tipagravecircumaccent{\TIPAaccent{\textgravecircum}}
\def\tipasubbaraccent{\TIPAaccent{\textsubbar}}
\def\tipasubringaccent{\TIPAaccent{\textsubring}}
\def\tipasubwedgeaccent{\TIPAaccent{\textsubwedge}}
\def\tipasubumlautaccent{\TIPAaccent{\textsubumlaut}}
\def\tipadoublegraveaccent{\TIPAaccent{\textdoublegrave}}
\def\tiparingmacronaccent{\TIPAaccent{\textringmacron}}
\def\tipabrevemacronaccent{\TIPAaccent{\textbrevemacron}}
\def\tipaacutewedgeaccent{\TIPAaccent{\textacutewedge}}

\def\tipasubbridgeaccent{\TIPAaccent{\textsubbridge}}
\def\tipainvsubbridgeaccent{\TIPAaccent{\textinvsubbridge}}
\def\tipasublhalfringaccent{\TIPAaccent{\textsublhalfring}}
\def\tipasubrhalfringaccent{\TIPAaccent{\textsubrhalfring}}
\def\tiparoundcapaccent{\TIPAaccent{\textroundcap}}
\def\tipasubplusaccent{\TIPAaccent{\textsubplus}}
\def\tiparaisingaccent{\TIPAaccent{\textraising}}
\def\tipaloweringaccent{\TIPAaccent{\textlowering}}
\def\tipaadvancingaccent{\TIPAaccent{\textadvancing}}
\def\tiparetractingaccent{\TIPAaccent{\textretracting}}
\def\tipaovercrossaccent{\TIPAaccent{\textovercross}}
\def\tipasubwaccent{\TIPAaccent{\textsubw}}
\def\tipaundertieaccent{\TIPAaccent{\textundertie}}
\def\tipasubarchaccent{\TIPAaccent{\textsubarch}}
\def\tipaseagullaccent{\TIPAaccent{\textseagull}}

\newcount\tipaTiiicode
\renewcommand{\tipaloweraccent}[2][]{%
 \def\next{}%
 \tipaTiiicode=#2\relax
 \ifcase\tipaTiiicode\relax
  \notipaloweraccent{#2}%      % 0 L    0 U = grave
 \or\notipaloweraccent{#2}%    % 1 U = acute
 \or\let\next\textsubcircum % 2 L    2 U = circum
 \or\let\next\textsubtilde  % 3 L    3 U = tilde
 \or\let\next\textsubumlaut % 4 L    4 U = umlaut
 \or\notipaloweraccent{#2}%    % 5 U = hungar-umlaut
 \or\let\next\textsubring   % 6 L    6 U = ring
 \or\let\next\textsubwedge  % 7 L    7 U = caron
 \or\let\next\tipasubarch   % 8 L    8 U = breve
 \or\let\next\textsubbar    % 9 L    9 U = macron
 \or\let\next\textsubdot    % 10 L  10 U = dot
 \or\let\next\c             % 11 L  cedilla
 \or\let\next\textpolhook   % 12 L  ogonek
 \or\let\next\textdoublegrave % 13 U = textdoublegrave
 \or\let\next\textsubgrave  % 14 L
 \or\let\next\textsubacute  % 15 L
 \or\notipaloweraccent{#2}%    % 16 U = textroundcap
 \or\let\next\textsubarch   % 16 L
 \or\let\next\textsubbridge % 17 L
 \or\let\next\textinvsubbridge % 18 L
 \or\let\next\textsubsquare % 19 L
 \or\let\next\textsubrhalfring % 20 L
 \or\let\next\textsublhalfring % 21 L
 \or\let\next\textsubw      % 22 L
% \or\let\next\textoverw     % 22 U = textoverw
 \or\let\next\textseagull   % 23 L
 \or\notipaloweraccent{#2}%    % 24 U = textovercross
 \or\notipaloweraccent{#2}%    % 25
 \or\notipaloweraccent{#2}%    % 26
 \or\let\next\textsubplus   % 27 L
 \or\let\next\textraising   % 28 L
 \or\let\next\textlowering  % 29 L
 \or\let\next\textadvancing % 30 L
 \or\let\next\textretracting% 31 L
 \or\notipaloweraccent{#2}%    % 32
 \or\notipaloweraccent{#2}%    % 33
 \or\let\next\textsyllabic  % 34 L
 \or\expandafter % 35
 \or\expandafter % 36
 \or\expandafter % 37
 \or\let\next\textsuperimposetilde % 38 lap
 \or\expandafter % 39
 \or\expandafter % 4
 \or\expandafter % 5
 \or\let\next\textundertieaccent % 60 L
 \or\let\next\texttoptiebar    % 62 U
 \or\let\next\textmidacute     % 152 U
 \or\let\next\textgravemid     % 153 U
 \or\let\next\textgravecircum  % 154 U
 \or\let\next\textcircumacute  % 155 U
 \or\let\next\textvbaraccent   % 156 U
 \or\let\next\textdoublevbaraccent % 157 U
 \or\let\next\textgravedot     % 158 U
 \or\let\next\textdotacute     % 159 U
% 128 + 24 = 152
 \else
  \expandafter\def\expandafter\next\expandafter{%
   \expandafter\tipaxloweraccent\expandafter{\the\tipaTiiicode}}%
 \fi
 \next }
\renewcommand{\tipaupperaccent}[2][]{%
 \def\next{}%
 \tipaTiiicode=#2\relax
 \ifcase\tipaTiiicode\relax 
  \let\next\`%    0  grave accent
 \or\let\next\'%  1  acute accent
 \or\let\next\^%  2  circum accent
 \or\let\next\~%  3  tilde accent
 \or\let\next\"%  4  umlaut accent
 \or\let\next\H%  5  hungar-umlaut accent
 \or\let\next\r%  6  ring accent
 \or\let\next\v%  7  caron accent
 \or\let\next\u%  8  breve accent
 \or\let\next\=%  9  macron accent
 \or\let\next\.%  10 dot-above accent
 \or\notipaupperaccent{#2}%  11
 \or\notipaupperaccent{#2}%  12
 \or\notipaupperaccent{#2}%  13
 \or\notipaupperaccent{#2}%  14
 \or\notipaupperaccent{#2}%  15
 \or\let\next\textroundcap % 16
 \or\notipaupperaccent{#2}%  17
 \or\notipaupperaccent{#2}%  18
 \or\notipaupperaccent{#2}%  19
 \or\notipaupperaccent{#2}%  20
 \or\notipaupperaccent{#2}%  21
 \or\let\next\textoverw %    22
 \or\notipaupperaccent{#2}%  23
 \or\let\next\textovercross% 24 
 \else
  \expandafter\def\expandafter\next\expandafter{%
   \expandafter\tipaxupperaccent\expandafter{\the\tipaTiiicode}}%
 \fi
 \next }

\def\notipaloweraccent#1{\typeout{*** no lowerable accent in position #1 ***}}
\def\notipaupperaccent#1{\typeout{*** no raiseable accent in position #1 ***}}

\let\realtipaupperaccent\tipaUpperaccent
\def\supsdimi{.2ex}
\def\supsdimii{.8ex}
\providecommand{\sups}[2]{{%\tracingall
 \textipa{\realtipaupperaccent[\supsdimi]%
  {\lower\supsdimii\hbox{\super{#2}}}{#1}}}}

% Constructed multiple accents
\def\tipaDOTacuteaccent#1{{\.{\ }\kern-.25em#1\kern-.2em\'{\ }}}
\def\tipagraveDOTaccent#1{{\`{\ }\kern-.25em#1\kern-.2em\.{\ }}}
\AtBeginDocument{%
 \let\textdotacute\tipaDOTacuteaccent
 \let\textgravedot\tipagraveDOTaccent
}


%: separator


% must not declare this, as the encoding is picked up by \textsc 
\newcommand{\faketextsc}[1]{\scalebox{.7}[.7]{#1}}
\def\TIPAtextscq{{\faketextsc{Q}}}
%\providecommand{\textscf}{\faketextsc{F}}% UxA730
\providecommand{\textscq}{\faketextsc{Q}}%
\providecommand{\textscdelta}{\faketextsc{\textDelta}}%

% Here's a way to implement TIPA's T3 encoding for uppercase letters and digits
% It's not the best possible way to implement this; but it is as good as can be
% done in macros, without introducing token-by-token parsing.

\def\setTIPAcatcodes{%
  \catcode `A = \active
  \catcode `B = \active
  \catcode `C = \active
  \catcode `D = \active
  \catcode `E = \active
  \catcode `F = \active
  \catcode `G = \active
  \catcode `g = \active
  \catcode `H = \active
  \catcode `I = \active
  \catcode `J = \active
  \catcode `K = \active
  \catcode `L = \active
  \catcode `M = \active
  \catcode `N = \active
  \catcode `O = \active
  \catcode `P = \active
  \catcode `Q = \active
  \catcode `R = \active
  \catcode `S = \active
  \catcode `T = \active
  \catcode `U = \active
  \catcode `V = \active
  \catcode `W = \active
  \catcode `X = \active
  \catcode `Y = \active
  \catcode `Z = \active
  \catcode `0 = \active
  \catcode `1 = \active
  \catcode `2 = \active
  \catcode `3 = \active
  \catcode `4 = \active
  \catcode `5 = \active
  \catcode `6 = \active
  \catcode `7 = \active
  \catcode `8 = \active
  \catcode `9 = \active
  \catcode `\@ = \active
  \catcode `\; = \active
  \catcode `\: = \active
  \catcode `\" = \active
  \catcode `| = \active
}
% these catcodes need to be unset, when following \*
\def\unsetTIPAcatcodes{%
  \catcode `A = 11
  \catcode `B = 11
  \catcode `C = 11
  \catcode `D = 11
  \catcode `E = 11
  \catcode `F = 11
  \catcode `G = 11
  \catcode `g = 11
  \catcode `H = 11
  \catcode `I = 11
  \catcode `J = 11
  \catcode `K = 11
  \catcode `L = 11
  \catcode `M = 11
  \catcode `N = 11
  \catcode `O = 11
  \catcode `P = 11
  \catcode `Q = 11
  \catcode `R = 11
  \catcode `S = 11
  \catcode `T = 11
  \catcode `U = 11
  \catcode `V = 11
  \catcode `W = 11
  \catcode `X = 11
  \catcode `Y = 11
  \catcode `Z = 11
  \catcode `0 = 12
  \catcode `1 = 12
  \catcode `2 = 12
  \catcode `3 = 12
  \catcode `4 = 12
  \catcode `5 = 12
  \catcode `6 = 12
  \catcode `7 = 12
  \catcode `8 = 12
  \catcode `9 = 12
  \catcode `\@ = 12
  \catcode `\; = 12
  \catcode `\: = 12
  \catcode `\" = 12
  \catcode `| = 12
}

{\global\let\setuptipaaccents\setupTIPAaccents
 \def\next{\textscriptg}%
 \def\nextii{\catcode `g = 11 }%
 \catcode `g = \active 
 \catcode `\" = \active 
 \expandafter\def\expandafter\next\expandafter{%
  \expandafter\def\expandafter g\expandafter{\next}}
 \setTIPAcatcodes
 \nextii
 \let\zdef\gdef
 \expandafter\zdef\expandafter\activatetipa\expandafter{%
  \expandafter\setuptipaaccents\next 
 \def A{\textscripta}%
% \def B{\textbeta}%  name taken for the greek letter
 \def B{\ss}%
 \def C{\textctc}%
 \def D{\dh}%
 \def E{\textepsilon}%
 \def F{\textphi}%
% \def G{\textgamma}%
 \def G{\textbabygamma}%
 \def H{\texthth}%
 \def I{\textsci}%
 \def J{\textctj}%
 \def K{\textinvscr}%
 \def L{\textturny}%
 \def M{\textltailm}%
 \def N{\ng}%
 \def O{\textopeno}%
 \def P{\textglotstop}%
 \def Q{\textrevglotstop}%
 \def R{\textfishhookr}%
 \def S{\textesh}%
 \def T{\texttheta}%
% \def U{\textupsilon}%
 \def U{\textscupsilon}%
 \def V{\textscriptv}%
 \def W{\textturnm}%
 \def X{\textchi}%
 \def Y{\textscy}%
% \def Z{\textyogh}%
 \def Z{\textezh}%
 \def 0{\textbaru}%
 \def 1{\textbari}%
 \def 2{\textturnv}%
 \def 3{\textrevepsilon}%
 \def 4{\textturnh}%
 \def 5{\textturna}%
 \def 6{\textturnscripta}%
 \def 7{\textramshorns}%
 \def 8{\textbaro}%
 \def 9{\textreve}%
 \def @{\textschwa}%
 \def :{\textlengthmark}%
 \def ;{\texthalflength}%
 \def |{\textpipe}%
 \let "\tipaprimstress
 }
}%  end of  \setTIPAcatcodes
%\show\activatetipa

% Primary and Secondary stress shortcuts
\def\tipaprimstress{\futurelet\next\tipaprimstressi}
\def\tipaprimstressi{%
 \ifx\next\tipaprimstress
  \def\next##1{\textsecstress}%
 \else
  \def\next{\textprimstress}%
 \fi \next
}


% \textipa really needs to parse all tokens,
% otherwise many macros cannot be used effectively
\def\tipanoendline{\endlinechar=-1}
\def\tipalettercatcode{11}
\def\tipacatchg#1{{%
 \def\tmpa{#1}\def\tmpb{g}%
 \ifx\tmpa\tmpb \def\next{\textscriptg}%
 \else
  \def\next{\implementTIPAtext{%
   \catcode`g \tipalettercatcode\relax
    \tipanoendline\relax
    \scantokens{#1}%
  }}%
 \fi \next
}}

% To set a special font; e.g. Doulos SIL
% use standard techniques to define the font-switch
% then redefine  \useTIPAfont  as follows:
%   \def\useTIPAfont{\doulos}
\def\useTIPAfont{} 

\DeclareRobustCommand{\implementTIPAtext}{%
 \bgroup
  \let\rtone\TIPArtonebar
  \let\stone\TIPAstonebar
  \let\tone\TIPAtonebar
%  \setTIPAcatcodes
  \activatetipa
  \useTIPAfont
  \implementTIPAtextx
}
\def\implementTIPAtextx#1{#1\egroup}
\def\implementTIPAtextxx#1{\endlinechar=-1 \scantokens{#1}\egroup}
\AtBeginDocument{%
 \let\textipa\tipacatchonechar %\tipacatchg %\implementTIPAtext
 \let\rtone\TIPArtonebar
 \let\stone\TIPAstonebar
 \let\tone\TIPAtonebar
 }
\expandafter\ifx\csname scantokens\endcsname\relax
\else
 \AtBeginDocument{\@ifpackageloaded{linguex}{%
  \let\implementTIPAtextx\implementTIPAtextxx
  }{}}%
\fi

% this is used with \t and \sliding
\def\checkfortipa#1#2#3{%
 \ifx\textipa#3\def\next##1{\tipacatchonechar{#1{##1}}}%
 \else\def\next{#2#3}\fi \next }


\newtoks\tipasavetokens
\newtoks\tipachecktokens
\newif\iftipaonetoken
\def\tipalasttoken{!@! do nothing with this !@!}
\def\tipacatchonechar#1{\begingroup
 \def\textipa##1{##1}% prevent recursion
 %\let\implementTIPAtextx\implementTIPAtextxx % forces \scantokens
 \tipaonetokentrue
 \let\s\textsyllabic
 \tipasavetokens={#1}\tipatestforonechar#1\tipalasttoken}
\def\tipatestforonechar{\futurelet\next\tipatesttoken}
\def\tipatesttoken{%
 \def\next@##1{\tipaaddtoken{##1}\tipatestforonechar}%
 \ifx\next\tipalasttoken
  \def\next@##1{\tipaprocessthetokens}%
 \else\ifx\bgroup\next \tipaonetokenfalse
  \def\next@##1{\tipaparsegrouping{##1}\tipatestforonechar}%
 \else\expandafter\ifx\space\next
  \expandafter\def\expandafter\next@\space{\tipaaddtoken{ }\tipatestforonechar}%
 \else\ifx\*\next \def\next@{\gettipastar}%
 \else\ifx\;\next \def\next@{\gettipasemicolon}%
 \else\ifx\:\next \def\next@{\gettipacolon}%
 \else\ifx\!\next \def\next@{\gettipaexclam}%
 \else\ifx\|\next \def\next@{\gettipavert}%
 \else\ifx\tone\next \def\next@##1##2{\tipaaddtoken{##1{##2}}\tipatestforonechar}%
 \else\ifx\rtone\next \def\next@##1##2{\tipaaddtoken{##1{##2}}\tipatestforonechar}%
 \else\ifx\stone\next \def\next@##1##2{\tipaaddtoken{##1{##2}}\tipatestforonechar}%
 \else\expandafter\ifcat\noexpand\next\active\tipaonetokenfalse
 \else\ifcat\next A\relax % letter
  \def\next@{\gettipaletter}%
 \else\ifcat\next 0\relax % other character
  \def\next@{\gettipaother}%
 \else \tipaonetokenfalse
 \fi\fi\fi\fi\fi\fi\fi\fi\fi\fi\fi\fi\fi\fi
 \next@
}
\def\tipaaddtoken#1{\expandafter\tipachecktokens\expandafter{\the\tipachecktokens#1}}
\def\tipaprocessthetokens{%
% \iftipaonetoken
%  \expandafter\def\expandafter\next\expandafter{\expandafter
%   \implementTIPAtext\expandafter{\the\tipachecktokens}\endgroup}%
% \else
%  \expandafter\def\expandafter\next\expandafter{\expandafter
%   \tipacatchg\expandafter{\the\tipachecktokens}\endgroup}%
% \fi  \next
 \expandafter\implementTIPAtext\expandafter{\the\tipachecktokens}\endgroup
}
\def\tipaparsegrouping#1{\begingroup
 \tipachecktokens={}%
 \let\tipaprocessthetokens\relax
 \tipatestforonechar#1\tipalasttoken
 \expandafter\def\expandafter\next\expandafter{\expandafter\endgroup
  \expandafter\tipaaddtoken\expandafter{\expandafter{\the\tipachecktokens}}}%
% \show\next
 \next
}
\def\gettipaletter#1{%
 \ifx A#1\tipaaddtoken{\textscripta}%
 \else\ifx B#1\tipaaddtoken{\ss}%
 \else\ifx C#1\tipaaddtoken{\textctc}%
 \else\ifx D#1\tipaaddtoken{\dh}%
 \else\ifx E#1\tipaaddtoken{\textepsilon}%
 \else\ifx F#1\tipaaddtoken{\textphi}%
 \else\ifx G#1\tipaaddtoken{\textbabygamma}%
 \else\ifx g#1\tipaaddtoken{\textscriptg}%
 \else\ifx H#1\tipaaddtoken{\texthth}%
 \else\ifx I#1\tipaaddtoken{\textsci}%
 \else\ifx J#1\tipaaddtoken{\textctj}%
 \else\ifx K#1\tipaaddtoken{\textinvscr}%
 \else\ifx L#1\tipaaddtoken{\textturny}%
 \else\ifx M#1\tipaaddtoken{\textltailm}%
 \else\ifx N#1\tipaaddtoken{\ng}%
 \else\ifx O#1\tipaaddtoken{\textopeno}%
 \else\ifx P#1\tipaaddtoken{\textglotstop}%
 \else\ifx Q#1\tipaaddtoken{\textrevglotstop}%
 \else\ifx R#1\tipaaddtoken{\textfishhookr}%
 \else\ifx S#1\tipaaddtoken{\textesh}%
 \else\ifx T#1\tipaaddtoken{\texttheta}%
 \else\ifx U#1\tipaaddtoken{\textscupsilon}%
 \else\ifx V#1\tipaaddtoken{\textscriptv}%
 \else\ifx W#1\tipaaddtoken{\textturnm}%
 \else\ifx X#1\tipaaddtoken{\textchi}%
 \else\ifx Y#1\tipaaddtoken{\textscy}%
 \else\ifx Z#1\tipaaddtoken{\textyogh}%
 \else\tipaaddtoken{#1}%
 \fi\fi\fi\fi\fi\fi\fi\fi\fi\fi\fi\fi\fi\fi
 \fi\fi\fi\fi\fi\fi\fi\fi\fi\fi\fi\fi\fi
 \tipatestforonechar
}
{\catcode `\" 12 \catcode `\@ 12
\gdef\gettipaother#1{%
 \ifx @#1\tipaaddtoken{\textschwa}%
 \else\ifx 0#1\tipaaddtoken{\textbaru}%
 \else\ifx 1#1\tipaaddtoken{\textbari}%
 \else\ifx 2#1\tipaaddtoken{\textturnv}%
 \else\ifx 3#1\tipaaddtoken{\textrevepsilon}%
 \else\ifx 4#1\tipaaddtoken{\textturnh}%
 \else\ifx 5#1\tipaaddtoken{\textturna}%
 \else\ifx 6#1\tipaaddtoken{\textturnscripta}%
 \else\ifx 7#1\tipaaddtoken{\textramshorns}%
 \else\ifx 8#1\tipaaddtoken{\textbaro}%
 \else\ifx 9#1\tipaaddtoken{\textreve}%
 \else\ifx :#1\tipaaddtoken{\textlengthmark}%
 \else\ifx ;#1\tipaaddtoken{\texthalflength}%
 \else\ifx |#1\tipaaddtoken{\textpipe}%
 \else\ifx "#1\tipaaddtoken{\tipaprimstress}%
 \else\tipaaddtoken{#1}%
 \fi\fi\fi\fi\fi\fi\fi\fi\fi\fi\fi\fi\fi\fi\fi
 \tipatestforonechar
}}
\def\gettipastar\*#1{%
 \ifx f#1\tipaaddtoken{\textObardotlessj}%
 \else\ifx k#1\tipaaddtoken{\textturnk}%
 \else\ifx r#1\tipaaddtoken{\textturnr}%
 \else\ifx t#1\tipaaddtoken{\textturnt}%
 \else\ifx w#1\tipaaddtoken{\textturnw}%
 \else\ifx j#1\tipaaddtoken{\textbardotlessj}%
 \else\ifx n#1\tipaaddtoken{\textltailn}%
 \else\ifx h#1\tipaaddtoken{\texthbar}%
 \else\ifx l#1\tipaaddtoken{\textbeltl}%
 \else\ifx z#1\tipaaddtoken{\textlyoghlig}%
 \else\tipaaddtoken{#1}%
 \fi\fi\fi\fi\fi\fi\fi\fi\fi\fi
 \tipatestforonechar
}
\def\gettipasemicolon\;#1{%
 \ifx A#1\tipaaddtoken{\textsca}%
 \else\ifx B#1\tipaaddtoken{\textscb}%
 \else\ifx E#1\tipaaddtoken{\textsce}%
 \else\ifx G#1\tipaaddtoken{\textscg}%
 \else\ifx H#1\tipaaddtoken{\textsch}%
 \else\ifx J#1\tipaaddtoken{\textscj}%
 \else\ifx L#1\tipaaddtoken{\textscl}%
 \else\ifx N#1\tipaaddtoken{\textscn}%
 \else\ifx Q#1\tipaaddtoken{\TIPAtextscq}%
 \else\ifx R#1\tipaaddtoken{\textscr}%
 \else\ifx U#1\tipaaddtoken{\textscu}%
 \else\tipaaddtoken{#1}%
 \fi\fi\fi\fi\fi\fi\fi\fi\fi\fi\fi
 \tipatestforonechar
}
\def\gettipacolon\:#1{%
 \ifx d#1\tipaaddtoken{\textrtaild}%
 \else\ifx l#1\tipaaddtoken{\textrtaill}%
 \else\ifx n#1\tipaaddtoken{\textrtailn}%
 \else\ifx r#1\tipaaddtoken{\textrtailr}%
 \else\ifx R#1\tipaaddtoken{\textturnrrtail}%
 \else\ifx s#1\tipaaddtoken{\textrtails}%
 \else\ifx t#1\tipaaddtoken{\textrtailt}%
 \else\ifx z#1\tipaaddtoken{\textrtailz}%
 \else\tipaaddtoken{#1}%
 \fi\fi\fi\fi\fi\fi\fi\fi
 \tipatestforonechar
}
\def\gettipaexclam\!#1{%
 \ifx b#1\tipaaddtoken{\texthtb}%
 \else\ifx d#1\tipaaddtoken{\texthtd}%
 \else\ifx g#1\tipaaddtoken{\texthtg}%
 \else\ifx j#1\tipaaddtoken{\texthtbardotlessj}%
 \else\ifx G#1\tipaaddtoken{\texthtscg}%
 \else\ifx o#1\tipaaddtoken{\textbullseye}%
 \else\tipaaddtoken{#1}%
 \fi\fi\fi\fi\fi\fi
 \tipatestforonechar
}
\gdef\gettipavert\|#1{%
 \ifx [#1\tipaaddtoken{\tipasubbridgeaccent}%
 \else\ifx ]#1\tipaaddtoken{\tipainvsubbridgeaccent}%
 \else\ifx (#1\tipaaddtoken{\tipasublhalfringaccent}%
 \else\ifx )#1\tipaaddtoken{\tipasubrhalfringaccent}%
 \else\ifx c#1\tipaaddtoken{\tiparoundcapaccent}%
 \else\ifx +#1\tipaaddtoken{\tipasubplusaccent}%
 \else\ifx '#1\tipaaddtoken{\tiparaisingaccent}%
 \else\ifx `#1\tipaaddtoken{\tipaloweringaccent}%
 \else\ifx <#1\tipaaddtoken{\tipaadvancingaccent}%
 \else\ifx >#1\tipaaddtoken{\tiparetractingaccent}%
 \else\ifx x#1\tipaaddtoken{\tipaovercrossaccent}%
 \else\ifx w#1\tipaaddtoken{\tipasubwaccent}%
 \else\ifx m#1\tipaaddtoken{\tipaseagullaccent}%
 \else\tipaaddtoken{#1}%
 \fi\fi\fi\fi\fi\fi\fi\fi\fi\fi\fi\fi\fi
 \tipatestforonechar
}

% With parsing, \activatetipa doesn't need to do as much.
\def\activatetipa{%
 %
}

\def\ttipatestforonechar#1#2!@!{%
 \ifx\relax#2\relax
  \def\next{\implementTIPAtext#1\endgroup}%
 \else
   \def\next{\tipacatchg{#1#2}\endgroup}%
 \fi \next
}
\def\parseTIPAtext#1{%
 \implementTIPAtext#1%
}

% default for these \TIPA… macros  is to just return #1
% But \* first makes it into a non-active character.
%
\def\DeclareTIPAstarmacro#1#2{%
 \expandafter\def\csname#1\string#2\endcsname##1{%
  \expandafter\@text@composite\csname#1\string#2\endcsname##1\@empty\@text@composite
  {{\endlinechar=-1 \unsetTIPAcatcodes\scantokens{##1}}}}%
}
\def\DeclareTIPAmacro#1#2{%
 \expandafter\def\csname#1\string#2\endcsname##1{%
  \expandafter\@text@composite\csname#1\string#2\endcsname##1\@empty\@text@composite
  {##1}}%
}

\DeclareRobustCommand{\T}[1]{\~{\m{#1}}}
\let\tipasafemode\relax

%: separator

{
 \gdef\real@five{5}%
 \gdef\real@four{4}%
 \gdef\real@three{3}%
 \gdef\real@two{2}%
 \gdef\real@one{1}%
 \catcode`5\active
 \catcode`4\active
 \catcode`3\active
 \catcode`2\active
 \catcode`1\active
 \gdef\TIPAresetdigits{%
  \edef1{\real@one}%
  \edef2{\real@two}%
  \edef3{\real@three}%
  \edef4{\real@four}%
  \edef5{\real@five}%
 }
}
\def\TIPAtonebar#1{\TIPAtonebari#1!!@!!}
\def\TIPAtonebari#1#2!!@!!{\TIPAtonebarx{#1}%
 \def\nextchar{#2}\ifx\nextchar\@empty
 \else
  \TIPAtonebar{#2}%
 \fi
}
\def\TIPAtonebarx#1{{\TIPAresetdigits
 \edef\TIPAtonedata{#1}\expandafter\tonebar
 \expandafter{\TIPAtonedata}}}

\def\TIPAstonebar#1{\TIPAstonebari#1!!@!!}
\def\TIPAstonebari#1#2!!@!!{\TIPAtonebarx{#1}%
 \def\nextchar{#2}\def\firstchar{#2}\ifx\nextchar\@empty
 \else
  \ifx\nextchar\firstchar
  \else
  \TIPAstonebar{#2}%
 \fi\fi
}

\def\TIPArtonebar#1{\TIPArtonebari#1!!@!!}
\def\TIPArtonebari#1#2!!@!!{\TIPArtonebarx{#1}%
 \def\nextchar{#2}\def\firstchar{#2}\ifx\nextchar\@empty
 \else
  \ifx\nextchar\firstchar
  \else
  \TIPArtonebar{#2}%
 \fi\fi
}
\def\TIPArtonebarx#1{{\TIPAresetdigits
 \edef\TIPAtonedata{#1}\expandafter\rtonebar
 \expandafter{\TIPAtonedata}}}

% faking right tone bars from left ones
\RequirePackage{graphicx}
\def\TIPAfakertonebar#1{{\TIPAreversedata#1!!@!!%
 \reflectbox{\expandafter\TIPAfakertonebari\expandafter{\TIPAtonedata}}}}
\def\TIPAfakertonebari#1{\TIPAfakertonebarii#1!!@!!}
\def\TIPAfakertonebarii#1#2!!@!!{\TIPAtonebarx{#1}%
 \def\nextchar{#2}\ifx\nextchar\@empty
 \else
  \TIPAfakertonebari{#2}%
 \fi
}
\def\TIPAreversedata#1#2!!@!!{%
 \expandafter\def\expandafter\TIPAtonedata\expandafter{\expandafter#1\TIPAtonedata}%
 \ifx\relax#2\relax\let\next\relax
 \else
  \def\next{\TIPAreversedata#2!!@!!}%
 \fi \next }
\def\TIPAtonedata{}
\def\UseFakeRightTones{\RequirePackage{graphicx}%
 \let\TIPArtonebar\TIPAfakertonebar}





\def\ReloadXunicode#1{\def\UTFencname{#1}%
  \makeatletter
   
%% FORK OF XUNICODE.STY BY WILL ROBERTSON
%
%% ORIGINAL HEADER TEXT:

% Copyright 2004-2011 Ross Moore <ross.moore@mq.edu.au>
% Released under the LPPL licence, version 1.3c or later.

\ProvidesFile{fontspec-xunicode.sty}[2011/09/09 v0.1 fork of xunicode]

\ProvidesExplFile{fontspec-xunicode-defs.tex}{2015/10/01}{v0.1}{fontspec-xunicode definitions}

% Use \DeclareUTFcharacter to assign a cs-name to
% access a Unicode code-point...
\newcommand{\DeclareUTFcharacter}[3][\UTFencname]
  {
    \DeclareUTFcharacter:nnn {#1} {#2} {#3}
  }

% Use \DeclareUTFcomposite to assign a cs-name to access
% accents or composite characters via Unicode code-points,
% or the Unicode "Composing Character" mechanism ...
\newcommand{\DeclareUTFcomposite}[4][\UTFencname]
  {
    \cs_set:cpn
      { \cs_to_str:N #1 \string #3 - \string #4 }
      { \char #2 \relax }
  }
  
\newcommand{\DeclareEncodedCompositeCharacter}[4]
  %  #1 = encoding
  %  #2 = accent-macro in TeX
  %  #3 = position of combining glyph in Unicode
  %  #4 = bare accent position, in Unicode
  %  ##1 = slot for the accented letter
  {
    \expandafter\def\expandafter\next@ii\expandafter{
     \expandafter\expandafter\expandafter\@text@composite\expandafter
      \csname #1\string#2\endcsname####1\@empty
       \@text@composite{\add@UTF@accent{#3}{####1}{#4}}}
    \expandafter\def\expandafter\next@i\expandafter{\expandafter\expandafter
     \expandafter\def\expandafter\csname #1\string#2\endcsname####1}
    \expandafter\next@i\expandafter{\next@ii}
    % i tried simplifying the above (as below) and it didn't work :(
  }

\newcommand{\DeclareEncodedCompositeAccents}[4]
  {
    \cs_set_nopar:cpx { #1 \string #2 } ##1
      {
        \exp_not:n { \@text@composite }
        \exp_not:c { #1 \string #2 } ##1
        \exp_not:n { \@empty \@text@composite }
          { \add@UTF@accents {#4} {##1} {#3} } 
      }
  }

\cs_set:Npn \DeclareUTFcharacter:nnn #1#2#3
  {
     \ifcat \active\noexpand#3

       \cs_set:cpn { \UTFencname \string#3 } { \char#2\relax }
       
       \cs_if_free:NTF #3
        {
          \cs_if_eq:NNTF \cf@encoding \UTFencname
           { \DeclareTextCommand{#3}{OT1}{\undefined} }
           { \DeclareTextCommand{#3}{\cf@encoding}{\undefined} }
        }
        {
          \group_begin: % macro #3 exists already ...
            \let\protect\noexpand
            \edef\UTF@testi{#3}\def\UTF@testii{#3}%
            \cs_if_eq:NNTF \UTF@testi \UTF@testii
              { \group_insert_after:N \use_none:n }
              { \group_insert_after:N \use:n }
          \group_end:
          {
           % ... but when it isn't robust, make it so
           \cs_set_eq:cN {?-\string#3} #3
           \use:x
             {
               \exp_not:n
                 { \DeclareTextCommand {#3} }
               {\cf@encoding}
               {\expandafter\noexpand\csname?-\string#3\endcsname}
             }
          }
        }
      
     \else % not active catcode --- shouldn't happen
       % WSPR: actually happens once!!

       \cs_set:cpn
         {
           \expandafter\string\csname\UTFencname\endcsname
           \expandafter\string\csname#3\endcsname
         }
         {\char#2\relax}

       \cs_if_eq:NNTF \cf@encoding \UTFencname
         {
           \exp_args:Nc \DeclareTextCommand {#3}{OT1}{\undefined}
         }
         {
           \exp_args:Nc \DeclareTextCommand {#3}{\cf@encoding}{\undefined}
         }

     \fi % end of \ifcat     
  }

\cs_set_protected:Npn \add@UTF@accent  #1#2#3 { #2 \char "#1 \relax }
\cs_set_protected:Npn \add@UTF@accents #1#2#3 { #2 \char "#1 \char "#3 \relax}

\endinput

%%%%%%%%%%%%%%%%%%%%%%%%%%%%%%%%%%%%%%%%%%%%%%%%%%%%%%%%%%%%%%%%%%%%%%%%%%%

\expandafter\expandafter\expandafter\let\expandafter
 \csname\UTFencname\string\textsuperscript\endcsname\realLaTeXsuperscript
\expandafter\expandafter\expandafter\let\expandafter
 \csname\UTFencname\string\textsubscript\endcsname\realLaTeXsubscript

\@ifundefined{textoverh}{\DeclareTextAccent{\textoverh}{OT1}{999}}{}
\DeclareEncodedCompositeCharacter{\UTFencname}{\textoverh}{02B0}{02B0} % ??? Combining h above
\@ifundefined{textoverheng}{\DeclareTextAccent{\textoverheng}{OT1}{999}}{}
\DeclareEncodedCompositeCharacter{\UTFencname}{\textoverheng}{02B1}{02B1} % ??? Combining heng above
\@ifundefined{textoverj}{\DeclareTextAccent{\textoverj}{OT1}{999}}{}
\DeclareEncodedCompositeCharacter{\UTFencname}{\textoverj}{02B2}{02B2} % ??? Combining j above
\@ifundefined{textoverr}{\DeclareTextAccent{\textoverr}{OT1}{999}}{}
\DeclareEncodedCompositeCharacter{\UTFencname}{\textoverh}{02B3}{02B3} % ??? Combining r above
\@ifundefined{textoverw}{\DeclareTextAccent{\textoverw}{OT1}{999}}{}
\DeclareEncodedCompositeCharacter{\UTFencname}{\textoverw}{1DD3}{02B7} % ??? Combining w above

\@ifundefined{textovery}{\DeclareTextAccent{\textovery}{OT1}{999}}{}
\DeclareEncodedCompositeCharacter{\UTFencname}{\textovery}{02B8}{02B8} % ??? Combining h above
\DeclareUTFcharacter[\UTFencname]{x1DFE}{\textspleftarrow}
\renewcommand{\textspleftarrow}{}

\@ifundefined{textsbleftarrow}{\DeclareTextAccent{\textsbleftarrow}{OT1}{999}}{}
\DeclareEncodedCompositeCharacter{\UTFencname}{\textsbleftarrow}{02FF}{02CB}  % Combining left arrow below

\DeclareUTFcharacter[\UTFencname]{x02F5}{\textgravedbl} % middle
\DeclareUTFcharacter[\UTFencname]{x02F7}{\texttildebelow} % middle
\DeclareUTFcharacter[\UTFencname]{x02F9}{\textopencorner} % 
\DeclareUTFcharacter[\UTFencname]{x02FA}{\textcorner} % 
\providecommand{\textrectangle}{\textopencorner\textcorner}\renewcommand{\textrectangle}{}

\DeclareUTFcharacter[\UTFencname]{x034D}{\textsubdoublearrow} % ??
\DeclareUTFcharacter[\UTFencname]{x0350}{\textrptr} % ??
\DeclareUTFcharacter[\UTFencname]{x0362}{\textsubrightarrow} % ??


\DeclareEncodedCompositeCharacter{\UTFencname}{\`}{0300}{02CB}  % Combining grave accent
\@ifundefined{capitalgrave}{\DeclareTextAccent{\capitalgrave}{OT1}{999}}{}
\DeclareEncodedCompositeCharacter{\UTFencname}{\capitalgrave}{0300}{02CB}  % textcomp grave accent
\DeclareEncodedCompositeCharacter{\UTFencname}{\'}{0301}{02CA}  % Combining acute accent
\@ifundefined{capitalacute}{\DeclareTextAccent{\capitalacute}{OT1}{999}}{}
\DeclareEncodedCompositeCharacter{\UTFencname}{\capitalacute}{0301}{02CA}  % textcomp acute accent
\DeclareEncodedCompositeCharacter{\UTFencname}{\^}{0302}{02C6}  % Combining circumflex accent
\@ifundefined{capitalcircumflex}{\DeclareTextAccent{\capitalcircumflex}{OT1}{999}}{}
\DeclareEncodedCompositeCharacter{\UTFencname}{\capitalcircumflex}{0302}{02C6}  % textcomp circumflex accent
\DeclareEncodedCompositeCharacter{\UTFencname}{\~}{0303}{02DC}  % Combining tilde
\@ifundefined{capitaltilde}{\DeclareTextAccent{\capitaltilde}{OT1}{999}}{}
\DeclareEncodedCompositeCharacter{\UTFencname}{\capitaltilde}{0303}{02DC}  % textcomp tilde
\DeclareEncodedCompositeCharacter{\UTFencname}{\=}{0304}{02C9}  % Combining macron
\@ifundefined{capitalmacron}{\DeclareTextAccent{\capitalmacron}{OT1}{999}}{}
\DeclareEncodedCompositeCharacter{\UTFencname}{\capitalmacron}{0304}{02C9}  % textcomp macron
\DeclareTextAccent{\textoverline}{OT1}{999}
\DeclareEncodedCompositeCharacter{\UTFencname}{\textoverline}{0305}{203E}  % Combining overline
\DeclareEncodedCompositeCharacter{\UTFencname}{\u}{0306}{02D8}  % Combining breve
\@ifundefined{capitalbreve}{\DeclareTextAccent{\capitalbreve}{OT1}{999}}{}
\DeclareEncodedCompositeCharacter{\UTFencname}{\capitalbreve}{0306}{02D8}  % textcomp breve
\DeclareEncodedCompositeCharacter{\UTFencname}{\.}{0307}{02D9}  % Combining dot above
\@ifundefined{capitaldotaccent}{\DeclareTextAccent{\capitaldotaccent}{OT1}{999}}{}
\DeclareEncodedCompositeCharacter{\UTFencname}{\capitaldotaccent}{0307}{02D9}  % textcomp dot above
\DeclareEncodedCompositeCharacter{\UTFencname}{\"}{0308}{00A8}  % Combining diaeresis
\@ifundefined{capitaldieresis}{\DeclareTextAccent{\capitaldieresis}{OT1}{999}}{}
\DeclareEncodedCompositeCharacter{\UTFencname}{\capitaldieresis}{0308}{00A8}  % textcomp diaeresis
\@ifundefined{m}{\DeclareTextAccent{\m}{OT1}{999}}{} % miscellaneous IPA symbols
\DeclareEncodedCompositeCharacter{\UTFencname}{\m}{0309}{0309}  % (Combining hook above)
\DeclareTextAccent{\texthookabove}{OT1}{999}
\DeclareEncodedCompositeCharacter{\UTFencname}{\texthookabove}{0309}{0309}  % Combining hook above
\DeclareEncodedCompositeCharacter{\UTFencname}{\r}{030A}{02DA}  % Combining ring above
\@ifundefined{capitalring}{\DeclareTextAccent{\capitalring}{OT1}{999}}{}
\DeclareEncodedCompositeCharacter{\UTFencname}{\capitalring}{030A}{02DA}  % textcomp ring above
\DeclareEncodedCompositeCharacter{\UTFencname}{\H}{030B}{02DD}  % Combining double acute
\@ifundefined{capitalhungarumlaut}{\DeclareTextAccent{\capitalhungarumlaut}{OT1}{999}}{}
\DeclareEncodedCompositeCharacter{\UTFencname}{\capitalhungarumlaut}{030B}{02DD}  % textcomp double acute
\DeclareEncodedCompositeCharacter{\UTFencname}{\v}{030C}{02C7}  % Combining caron
\@ifundefined{capitalcaron}{\DeclareTextAccent{\capitalcaron}{OT1}{999}}{}
\DeclareEncodedCompositeCharacter{\UTFencname}{\capitalcaron}{030C}{02C7}  % textcomp caron
\@ifundefined{textvbaraccent}{\DeclareTextAccent{\textvbaraccent}{OT1}{999}}{}
\DeclareEncodedCompositeCharacter{\UTFencname}{\textvbaraccent}{030D}{02C8}  % Combining vertical line above
\@ifundefined{textdoublevbaraccent}{\DeclareTextAccent{\textdoublevbaraccent}{OT1}{999}}{}
\DeclareEncodedCompositeCharacter{\UTFencname}{\textdoublevbaraccent}{030E}{030E}  % Combining double vertical line above
\@ifundefined{U}{\DeclareTextAccent{\U}{OT1}{999}}{}
\DeclareEncodedCompositeCharacter{\UTFencname}{\U}{030E}{}  % Combining double vertical line above
\@ifundefined{textdoublegrave}{\DeclareTextAccent{\textdoublegrave}{OT1}{999}}{}
\DeclareEncodedCompositeCharacter{\UTFencname}{\textdoublegrave}{030F}{02F5}  % Combining double grave accent
\@ifundefined{G}{\DeclareTextAccent{\G}{OT1}{999}}{}
\DeclareEncodedCompositeCharacter{\UTFencname}{\G}{030F}{02F5}  % Combining double grave accent
\@ifundefined{textdotbreve}{\DeclareTextAccent{\textdotbreve}{OT1}{999}}{}
\DeclareEncodedCompositeCharacter{\UTFencname}{\textdotbreve}{0310}{0310}  % Combining candrabindu
\@ifundefined{textroundcap}{\DeclareTextAccent{\textroundcap}{OT1}{999}}{}
\DeclareEncodedCompositeCharacter{\UTFencname}{\textroundcap}{0311}{0311}  % Combining inverted breve
\@ifundefined{newtie}{\DeclareTextAccent{\newtie}{OT1}{999}}{}
\DeclareEncodedCompositeCharacter{\UTFencname}{\newtie}{0311}{0311}  % Combining inverted breve
\@ifundefined{capitalnewtie}{\DeclareTextAccent{\capitalnewtie}{OT1}{999}}{}
\DeclareEncodedCompositeCharacter{\UTFencname}{\capitalnewtie}{0311}{0311}  % Combining inverted breve
\DeclareTextAccent{\textturncommaabove}{OT1}{999}
\DeclareEncodedCompositeCharacter{\UTFencname}{\textturncommaabove}{0312}{02BB}  % Combining turned comma above
\DeclareTextAccent{\textcommaabove}{OT1}{999}
\DeclareEncodedCompositeCharacter{\UTFencname}{\textcommaabove}{0313}{02BC}  % Combining comma above
\DeclareTextAccent{\textrevcommaabove}{OT1}{999}
\DeclareEncodedCompositeCharacter{\UTFencname}{\textrevcommaabove}{0314}{02BD}  % Combining reversed comma above
\DeclareTextAccent{\textcommaabover}{OT1}{999}
\DeclareEncodedCompositeCharacter{\UTFencname}{\textcommaabover}{0315}{02BC}  % Combining comma above right
\@ifundefined{textsubgrave}{\DeclareTextAccent{\textsubgrave}{OT1}{999}}{}
\DeclareEncodedCompositeCharacter{\UTFencname}{\textsubgrave}{0316}{02CE}  % Combining grave accent below
\@ifundefined{textsubacute}{\DeclareTextAccent{\textsubacute}{OT1}{999}}{}
\DeclareEncodedCompositeCharacter{\UTFencname}{\textsubacute}{0317}{02CF}  % Combining acute accent below
\@ifundefined{textadvancing}{\DeclareTextAccent{\textadvancing}{OT1}{999}}{}
\DeclareEncodedCompositeCharacter{\UTFencname}{\textadvancing}{0318}{0318}  % Combining left tack below
\@ifundefined{textretracting}{\DeclareTextAccent{\textretracting}{OT1}{999}}{}
\DeclareEncodedCompositeCharacter{\UTFencname}{\textretracting}{0319}{0319}  % Combining right tack below
\DeclareTextAccent{\textlangleabove}{OT1}{999}
\DeclareEncodedCompositeCharacter{\UTFencname}{\textlangleabove}{031A}{031A}  % Combining left angle above
\DeclareTextAccent{\textrighthorn}{OT1}{999}
\DeclareEncodedCompositeCharacter{\UTFencname}{\textrighthorn}{031B}{031B}  % Combining horn
\@ifundefined{textsublhalfring}{\DeclareTextAccent{\textsublhalfring}{OT1}{999}}{}
\DeclareEncodedCompositeCharacter{\UTFencname}{\textsublhalfring}{031C}{02D3}  % Combining left half ring below
\@ifundefined{textraising}{\DeclareTextAccent{\textraising}{OT1}{999}}{}
\DeclareEncodedCompositeCharacter{\UTFencname}{\textraising}{031D}{02D4}  % Combining up tack below
\@ifundefined{textlowering}{\DeclareTextAccent{\textlowering}{OT1}{999}}{}
\DeclareEncodedCompositeCharacter{\UTFencname}{\textlowering}{031E}{02D5}  % Combining down tack below
\@ifundefined{textsubplus}{\DeclareTextAccent{\textsubplus}{OT1}{999}}{}
\DeclareEncodedCompositeCharacter{\UTFencname}{\textsubplus}{031F}{02D6}  % Combining plus sign below
\@ifundefined{textsubminus}{\DeclareTextAccent{\textsubminus}{OT1}{999}}{}
\DeclareEncodedCompositeCharacter{\UTFencname}{\textsubminus}{0320}{02D7}  % Combining minus sign below
\DeclareTextAccent{\textpalhookbelow}{OT1}{999}
\DeclareEncodedCompositeCharacter{\UTFencname}{\textpalhookbelow}{0321}{0321}  % Combining palatalized hook below
\@ifundefined{M}{\DeclareTextAccent{\M}{OT1}{999}}{} % more Miscellaneous IPA characters
\DeclareEncodedCompositeCharacter{\UTFencname}{\M}{0322}{0322}  % (Combining retroflex hook below)
\DeclareTextAccent{\textrethookbelow}{OT1}{999}
\DeclareEncodedCompositeCharacter{\UTFencname}{\textrethookbelow}{0322}{0322}  % Combining retroflex hook below
\DeclareEncodedCompositeCharacter{\UTFencname}{\d}{0323}{0323}  % Combining dot below
\@ifundefined{textsubdot}{\DeclareTextAccent{\textsubdot}{OT1}{999}}{}
\DeclareEncodedCompositeCharacter{\UTFencname}{\textsubdot}{0323}{0323}  % Combining dot below
\@ifundefined{textsubumlaut}{\DeclareTextAccent{\textsubumlaut}{OT1}{999}}{}
\DeclareEncodedCompositeCharacter{\UTFencname}{\textsubumlaut}{0324}{0324}  % Combining diaeresis below
\@ifundefined{textsubring}{\DeclareTextAccent{\textsubring}{OT1}{999}}{}
\DeclareEncodedCompositeCharacter{\UTFencname}{\textsubring}{0325}{02F3}  % Combining ring below
\DeclareTextAccent{\textcommabelow}{OT1}{999}
\DeclareEncodedCompositeCharacter{\UTFencname}{\textcommabelow}{0326}{0326}  % Combining comma below
\DeclareEncodedCompositeCharacter{\UTFencname}{\c}{0327}{00B8}  % Combining cedilla
\@ifundefined{capitalcedilla}{\DeclareTextAccent{\capitalcedilla}{OT1}{999}}{}
\DeclareEncodedCompositeCharacter{\UTFencname}{\capitalcedilla}{0327}{00B8}  % Combining cedilla
\DeclareEncodedCompositeCharacter{\UTFencname}{\k}{0328}{02DB}  % Combining ogonek
\@ifundefined{capitalogonek}{\DeclareTextAccent{\capitalogonek}{OT1}{999}}{}
\DeclareEncodedCompositeCharacter{\UTFencname}{\capitalogonek}{0328}{02DB}  % Combining ogonek
\@ifundefined{textpolhook}{\DeclareTextAccent{\textpolhook}{OT1}{999}}{}
\DeclareEncodedCompositeCharacter{\UTFencname}{\textpolhook}{0328}{02DB}  % Combining polish hook (ogonek)
\@ifundefined{textsyllabic}{\DeclareTextAccent{\textsyllabic}{OT1}{999}}{}
\DeclareEncodedCompositeCharacter{\UTFencname}{\textsyllabic}{0329}{02CC}  % Combining vertical line below
\@ifundefined{textsubbridge}{\DeclareTextAccent{\textsubbridge}{OT1}{999}}{}
\DeclareEncodedCompositeCharacter{\UTFencname}{\textsubbridge}{032A}{032A}  % Combining bridge below
\let\dental\textsubbridge
\@ifundefined{textsubw}{\DeclareTextAccent{\textsubw}{OT1}{999}}{}
\DeclareEncodedCompositeCharacter{\UTFencname}{\textsubw}{032B}{032B}  % Combining inverted double arch below
\@ifundefined{textsubwedge}{\DeclareTextAccent{\textsubwedge}{OT1}{999}}{}
\DeclareEncodedCompositeCharacter{\UTFencname}{\textsubwedge}{032C}{032C}  % Combining caron below
\@ifundefined{textsubcircum}{\DeclareTextAccent{\textsubcircum}{OT1}{999}}{}
\DeclareEncodedCompositeCharacter{\UTFencname}{\textsubcircum}{032D}{032D}  % Combining circumflex accent below
\@ifundefined{textundertie}{\DeclareTextAccent{\textundertie}{OT1}{999}}{}
\DeclareEncodedCompositeCharacter{\UTFencname}{\textundertie}{032E}{203F}  % Combining breve below
\@ifundefined{textsubarch}{\DeclareTextAccent{\textsubarch}{OT1}{999}}{}
\DeclareEncodedCompositeCharacter{\UTFencname}{\textsubarch}{032F}{032F}  % Combining inverted breve below
\let\underarch\textsubarch
\@ifundefined{textsubtilde}{\DeclareTextAccent{\textsubtilde}{OT1}{999}}{}
\DeclareEncodedCompositeCharacter{\UTFencname}{\textsubtilde}{0330}{02F7}  % Combining tilde below
\@ifundefined{textsubbar}{\DeclareTextAccent{\textsubbar}{OT1}{999}}{}
\DeclareEncodedCompositeCharacter{\UTFencname}{\textsubbar}{0331}{02CD}  % Combining macron below
\DeclareEncodedCompositeCharacter{\UTFencname}{\b}{0332}{005F}  % Combining low line
\@ifundefined{subdoublebar}{\DeclareTextAccent{\subdoublebar}{OT1}{999}}{} 
\DeclareEncodedCompositeCharacter{\UTFencname}{\subdoublebar}{0333}{0333}  % ??? see x0347
\DeclareTextAccent{\textsuperimposetilde}{OT1}{999}
\DeclareEncodedCompositeCharacter{\UTFencname}{\textsuperimposetilde}{0334}{007E}  % Combining tilde overlay,  x02DC ?
\@ifundefined{B}{\DeclareTextAccent{\B}{OT1}{999}}{} % barred variants for TIPA
\DeclareEncodedCompositeCharacter{\UTFencname}{\B}{0335}{02D7}  % (Combining short stroke overlay)
\DeclareTextAccent{\textsstrokethru}{OT1}{999} % stroke thru lowercase letters
\DeclareEncodedCompositeCharacter{\UTFencname}{\textsstrokethru}{0335}{00AF}  % Combining short stroke overlay,  x02D7 ?
\DeclareTextAccent{\textlstrokethru}{OT1}{999} % stroke thru Uppercase letters
\DeclareEncodedCompositeCharacter{\UTFencname}{\textlstrokethru}{0336}{0336}  % Combining long stroke overlay
\DeclareTextAccent{\textsstrikethru}{OT1}{999} % strike out lowercase letters
\DeclareEncodedCompositeCharacter{\UTFencname}{\textsstrikethru}{0337}{0337}  % Combining short solidus overlay
\DeclareTextAccent{\textlstrikethru}{OT1}{999} % strike out Uppercase letters
\DeclareEncodedCompositeCharacter{\UTFencname}{\textlstrikethru}{0338}{0338}  % Combining long solidus overlay
\@ifundefined{textsubrhalfring}{\DeclareTextAccent{\textsubrhalfring}{OT1}{999}}{}
\DeclareEncodedCompositeCharacter{\UTFencname}{\textsubrhalfring}{0339}{02D2}  % Combining right half ring below
\@ifundefined{textinvsubbridge}{\DeclareTextAccent{\textinvsubbridge}{OT1}{999}}{}
\DeclareEncodedCompositeCharacter{\UTFencname}{\textinvsubbridge}{033A}{033A}  % Combining inverted bridge below
\@ifundefined{textsubsquare}{\DeclareTextAccent{\textsubsquare}{OT1}{999}}{}
\DeclareEncodedCompositeCharacter{\UTFencname}{\textsubsquare}{033B}{033B}  % Combining square below
\@ifundefined{textseagull}{\DeclareTextAccent{\textseagull}{OT1}{999}}{}
\DeclareEncodedCompositeCharacter{\UTFencname}{\textseagull}{033C}{033C}  % Combining seagull below
\@ifundefined{textovercross}{\DeclareTextAccent{\textovercross}{OT1}{999}}{}
\DeclareEncodedCompositeCharacter{\UTFencname}{\textovercross}{033D}{033D}  % Combining x above
\@ifundefined{textdoubleoverline}{\DeclareTextAccent{\textdoubleoverline}{OT1}{999}}{}
\DeclareEncodedCompositeCharacter{\UTFencname}{\textdoubleoverline}{033F}{033F}  % Combining double overline
\@ifundefined{overbridge}{\DeclareTextAccent{\overbridge}{OT1}{999}}{}
\DeclareEncodedCompositeCharacter{\UTFencname}{\overbridge}{0346}{0346}  % Combining bridge above
\@ifundefined{subdoublevert}{\DeclareTextAccent{\subdoublevert}{OT1}{999}}{}
\DeclareEncodedCompositeCharacter{\UTFencname}{\subdoublevert}{0348}{0348}  % Combining double vertical line below
\@ifundefined{subcorner}{\DeclareTextAccent{\subcorner}{OT1}{999}}{}
\DeclareEncodedCompositeCharacter{\UTFencname}{\subcorner}{0349}{0349}  % Combining left angle below
\@ifundefined{crtilde}{\DeclareTextAccent{\crtilde}{OT1}{999}}{}
\DeclareEncodedCompositeCharacter{\UTFencname}{\crtilde}{034A}{034A} % Combining not tilde above
\@ifundefined{dottedtilde}{\DeclareTextAccent{\dottedtilde}{OT1}{999}}{}
\DeclareEncodedCompositeCharacter{\UTFencname}{\dottedtilde}{034B}{034B}  % Combining homothetic above
\@ifundefined{doubletilde}{\DeclareTextAccent{\doubletilde}{OT1}{999}}{}
\DeclareEncodedCompositeCharacter{\UTFencname}{\doubletilde}{034C}{034C}  % Combining almost equal to above
\@ifundefined{spreadlips}{\DeclareTextAccent{\spreadlips}{OT1}{999}}{}
\DeclareEncodedCompositeCharacter{\UTFencname}{\spreadlips}{034D}{034D}  % Combining left right arrow below
\@ifundefined{whistle}{\DeclareTextAccent{\whistle}{OT1}{999}}{}
\DeclareEncodedCompositeCharacter{\UTFencname}{\whistle}{034E}{02F0}  % Combining upwards arrow below
\DeclareTextAccent{\textgraphemejoin}{OT1}{999}
\DeclareEncodedCompositeCharacter{\UTFencname}{\textgraphemejoin}{034F}{034F}  % Combining grapheme joiner
\DeclareTextAccent{\textrightarrowhead}{OT1}{999}
\DeclareEncodedCompositeCharacter{\UTFencname}{\textrightarrowhead}{0350}{02C3}  % Combining right arrowhead above
\DeclareTextAccent{\textlefthalfring}{OT1}{999}
\DeclareEncodedCompositeCharacter{\UTFencname}{\textlefthalfring}{0351}{02D3}  % Combining left half ring above
\@ifundefined{sublptr}{\DeclareTextAccent{\sublptr}{OT1}{999}}{}
\DeclareEncodedCompositeCharacter{\UTFencname}{\sublptr}{0354}{02F1}  % Combining left arrowhead below
\@ifundefined{subrptr}{\DeclareTextAccent{\subrptr}{OT1}{999}}{}
\DeclareEncodedCompositeCharacter{\UTFencname}{\subrptr}{0355}{02F2}  % Combining right arrowhead below
\DeclareTextAccent{\textrightuparrowhead}{OT1}{999}
\DeclareEncodedCompositeCharacter{\UTFencname}{\textrightuparrowhead}{0356}{0356}  % Combining right arrowhead and up
\DeclareTextAccent{\textrighthalfring}{OT1}{999}
\DeclareEncodedCompositeCharacter{\UTFencname}{\textrighthalfring}{0357}{02D2}  % Combining right half ring above
\@ifundefined{textdoublebrevebelow}{\DeclareTextAccent{\textdoublebrevebelow}{OT1}{999}}{}
\DeclareEncodedCompositeCharacter{\UTFencname}{\textdoublebrevebelow}{035C}{035C}  % Combining double breve below
\DeclareTextAccent{\textdoublebreve}{OT1}{999}
\DeclareEncodedCompositeCharacter{\UTFencname}{\textdoublebreve}{035D}{035D}  % Combining double breve
\DeclareTextAccent{\textdoublemacron}{OT1}{999}
\DeclareEncodedCompositeCharacter{\UTFencname}{\textdoublemacron}{035E}{035E}  % Combining double macron
\DeclareTextAccent{\textdoublemacronbelow}{OT1}{999}
\DeclareEncodedCompositeCharacter{\UTFencname}{\textdoublemacronbelow}{035F}{035F}  % Combining double macron below
\DeclareTextAccent{\textdoubletilde}{OT1}{999}
\DeclareEncodedCompositeCharacter{\UTFencname}{\textdoubletilde}{0360}{0360}  % Combining double tilde
\@ifundefined{t}{\DeclareTextAccent{\t}{OT1}{999}}{}
\DeclareEncodedCompositeCharacter{\UTFencname}{\t}{0361}{0361}  % Combining double inverted breve
\@ifundefined{capitaltie}{\DeclareTextAccent{\capitaltie}{OT1}{999}}{}
\DeclareEncodedCompositeCharacter{\UTFencname}{\capitaltie}{0361}{0361}  % Combining double inverted breve
\@ifundefined{texttoptiebar}{\DeclareTextAccent{\texttoptiebar}{OT1}{999}}{}
\DeclareEncodedCompositeCharacter{\UTFencname}{\texttoptiebar}{0361}{0361}  % Combining double inverted breve

% LaTeX uses \t as  \t{oo}, so this needs implementing also:
\expandafter\expandafter\expandafter\let
 \expandafter\csname\expandafter \UTFencname\expandafter\string\expandafter\tie
  \expandafter\expandafter\expandafter\endcsname
  \expandafter\csname \UTFencname\string\t\endcsname

% *** need to look for \textipa and expand its contents before the slide ***
\expandafter\edef\csname \UTFencname\string\t\endcsname#1{%
 \noexpand\checkfortipa{\noexpand\t}%
  {\expandafter\noexpand\expandafter\csname \UTFencname\string\tie\endcsname}#1}

% these become just alternative names for the same construction
\expandafter\let\csname\UTFencname\string\capitaltie\expandafter\endcsname\csname \UTFencname\string\t\endcsname
%\expandafter\let\csname \UTFencname\string\texttoptiebar\expandafter\endcsname\csname \UTFencname\string\tie\endcsname

\expandafter\expandafter\expandafter\let\expandafter\csname \expandafter\UTFencname\expandafter\string\expandafter\texttoptiebar\expandafter\expandafter\expandafter\endcsname\expandafter\csname \UTFencname\string\t\endcsname

%\expandafter\show\csname \UTFencname\string\texttoptiebar\endcsname
%
\@ifundefined{sliding}{\DeclareTextAccent{\sliding}{OT1}{999}}{}
\DeclareEncodedCompositeCharacter{\UTFencname}{\sliding}{0362}{0362}  % Combining double rightwards arrow
\expandafter\let\csname \UTFencname\string\slidingtie\expandafter\endcsname\csname\UTFencname\string\sliding\endcsname
% *** need to look for \textipa and expand its contents before the slide ***
\expandafter\def\csname \UTFencname\string\sliding\expandafter\endcsname#1{%
 \checkfortipa{\sliding}{\csname \UTFencname\string\slidingtie\endcsname}#1}


% Combining diacritics Supplement
\DeclareTextAccent{\texthighrise}{OT1}{999}
\DeclareEncodedCompositeCharacter{\UTFencname}{\texthighrise}{1DC4}{1DC4}  % Combining macron acute
\DeclareTextAccent{\textlowrise}{OT1}{999}
\DeclareEncodedCompositeCharacter{\UTFencname}{\textlowrise}{1DC5}{1DC5}   % Combining grave macron
\DeclareTextAccent{\textrisefall}{OT1}{999}
\DeclareEncodedCompositeCharacter{\UTFencname}{\textrisefall}{1DC8}{1DC8}  % Combining grave acute grave
\DeclareTextAccent{\textfallrise}{OT1}{999}
\DeclareEncodedCompositeCharacter{\UTFencname}{\textfallrise}{1DC9}{1DC9}  % Combining acute grave acute
\DeclareEncodedCompositeCharacter{\UTFencname}{\textaolig}{1DD5}{1DD5}  % Combining ao-lig 

\@ifundefined{bibridge}{\DeclareTextAccent{\bibridge}{OT1}{999}}{}
\DeclareEncodedCompositeAccents{\UTFencname}{\bibridge}{032A}{0346}  % IPA bi-bridge
\@ifundefined{textmidacute}{\DeclareTextAccent{\textmidacute}{OT1}{999}}{}
\DeclareEncodedCompositeAccents{\UTFencname}{\textmidacute}{0304}{0301}  % macron-acute ligature
\@ifundefined{textgravemid}{\DeclareTextAccent{\textgravemid}{OT1}{999}}{}
\DeclareEncodedCompositeAccents{\UTFencname}{\textgravemid}{0300}{0304}  % grave-macron ligature
\@ifundefined{textgravecircum}{\DeclareTextAccent{\textgravecircum}{OT1}{999}}{}
\DeclareEncodedCompositeAccents{\UTFencname}{\textgravecircum}{0300}{0302}  % grave-circumflex ligature
\@ifundefined{textcircumacute}{\DeclareTextAccent{\textcircumacute}{OT1}{999}}{}
\DeclareEncodedCompositeAccents{\UTFencname}{\textcircumacute}{0301}{0302}  % circumflex-acute ligature
\@ifundefined{textgravedot}{\DeclareTextAccent{\textgravedot}{OT1}{999}}{}
\DeclareEncodedCompositeAccents{\UTFencname}{\textgravedot}{0300}{0307}  % grave-dot ligature
\@ifundefined{textdotacute}{\DeclareTextAccent{\textdotacute}{OT1}{999}}{}
\DeclareEncodedCompositeAccents{\UTFencname}{\textdotacute}{0307}{0301}  % dot-acute ligature
\@ifundefined{textacutemacron}{\DeclareTextAccent{\textacutemacron}{OT1}{999}}{}
\DeclareEncodedCompositeAccents{\UTFencname}{\textacutemacron}{0301}{0304}  % acute-macron ligature
\@ifundefined{textgravemacron}{\DeclareTextAccent{\textgravemacron}{OT1}{999}}{}
\DeclareEncodedCompositeAccents{\UTFencname}{\textgravemacron}{0300}{0304}  % grave-macron ligature
\@ifundefined{textacutewedge}{\DeclareTextAccent{\textacutewedge}{OT1}{999}}{}
\DeclareEncodedCompositeAccents{\UTFencname}{\textacutewedge}{0301}{030C}  % acute-wedge ligature
\@ifundefined{textcircumdot}{\DeclareTextAccent{\textcircumdot}{OT1}{999}}{}
\DeclareEncodedCompositeAccents{\UTFencname}{\textcircumdot}{0302}{0307}  % circumflex-dot ligature
\@ifundefined{texttildedot}{\DeclareTextAccent{\texttildedot}{OT1}{999}}{}
\DeclareEncodedCompositeAccents{\UTFencname}{\texttildedot}{0303}{0307}  % tilde-dot ligature
\@ifundefined{textringmacron}{\DeclareTextAccent{\textringmacron}{OT1}{999}}{}
\DeclareEncodedCompositeAccents{\UTFencname}{\textringmacron}{030A}{0304}  % ring-macron ligature
\@ifundefined{textbrevemacron}{\DeclareTextAccent{\textbrevemacron}{OT1}{999}}{}
\DeclareEncodedCompositeAccents{\UTFencname}{\textbrevemacron}{0306}{0304}  % breve-macron ligature

\@ifundefined{texthookcircum}{\DeclareTextAccent{\texthookcircum}{OT1}{999}}{}
\DeclareEncodedCompositeAccents{\UTFencname}{\texthookcircum}{0309}{0302}
\@ifundefined{texttildecircum}{\DeclareTextAccent{\texttildecircum}{OT1}{999}}{}
\DeclareEncodedCompositeAccents{\UTFencname}{\texttildecircum}{0303}{0302}
\@ifundefined{textdieresisoverline}{\DeclareTextAccent{\textdieresisoverline}{OT1}{999}}{}
\DeclareEncodedCompositeAccents{\UTFencname}{\textdieresisoverline}{0304}{0308}
\@ifundefined{textdieresisacute}{\DeclareTextAccent{\textdieresisacute}{OT1}{999}}{}
\DeclareEncodedCompositeAccents{\UTFencname}{\textdieresisacute}{0301}{0308}
\@ifundefined{textdieresisgrave}{\DeclareTextAccent{\textdieresisgrave}{OT1}{999}}{}
\DeclareEncodedCompositeAccents{\UTFencname}{\textdieresisgrave}{0300}{0308}
\@ifundefined{textdieresiscaron}{\DeclareTextAccent{\textdieresiscaron}{OT1}{999}}{}
\DeclareEncodedCompositeAccents{\UTFencname}{\textdieresiscaron}{030C}{0308}
\@ifundefined{texttildeoverline}{\DeclareTextAccent{\texttildeoverline}{OT1}{999}}{}
\DeclareEncodedCompositeAccents{\UTFencname}{\texttildeoverline}{0303}{0304}
\@ifundefined{textdotoverline}{\DeclareTextAccent{\textdotoverline}{OT1}{999}}{}
\DeclareEncodedCompositeAccents{\UTFencname}{\textdotoverline}{0304}{0307}
\@ifundefined{textringacute}{\DeclareTextAccent{\textringacute}{OT1}{999}}{}
\DeclareEncodedCompositeAccents{\UTFencname}{\textringacute}{0301}{030A}
\@ifundefined{textcircumdotbelow}{\DeclareTextAccent{\textcircumdotbelow}{OT1}{999}}{}
\DeclareEncodedCompositeAccents{\UTFencname}{\textcircumdotbelow}{0302}{0323}
\@ifundefined{textbreveacute}{\DeclareTextAccent{\textbreveacute}{OT1}{999}}{}
\DeclareEncodedCompositeAccents{\UTFencname}{\textbreveacute}{0301}{0306}
\@ifundefined{textbrevegrave}{\DeclareTextAccent{\textbrevegrave}{OT1}{999}}{}
\DeclareEncodedCompositeAccents{\UTFencname}{\textbrevegrave}{0300}{0306}
\@ifundefined{textbrevehook}{\DeclareTextAccent{\textbrevehook}{OT1}{999}}{}
\DeclareEncodedCompositeAccents{\UTFencname}{\textbrevehook}{0309}{0306}
\@ifundefined{textbrevetilde}{\DeclareTextAccent{\textbrevetilde}{OT1}{999}}{}
\DeclareEncodedCompositeAccents{\UTFencname}{\textbrevetilde}{0303}{0306}
\@ifundefined{textbrevedotbelow}{\DeclareTextAccent{\textbrevedotbelow}{OT1}{999}}{}
\DeclareEncodedCompositeAccents{\UTFencname}{\textbrevedotbelow}{0323}{0306}
\@ifundefined{textacutehorn}{\DeclareTextAccent{\textacutehorn}{OT1}{999}}{}
\DeclareEncodedCompositeAccents{\UTFencname}{\textacutehorn}{0301}{031B}
\@ifundefined{textgravehorn}{\DeclareTextAccent{\textgravehorn}{OT1}{999}}{}
\DeclareEncodedCompositeAccents{\UTFencname}{\textgravehorn}{0300}{031B}
\@ifundefined{texthookhorn}{\DeclareTextAccent{\texthookhorn}{OT1}{999}}{}
\DeclareEncodedCompositeAccents{\UTFencname}{\texthookhorn}{0309}{031B}
\@ifundefined{texttildehorn}{\DeclareTextAccent{\texttildehorn}{OT1}{999}}{}
\DeclareEncodedCompositeAccents{\UTFencname}{\texttildehorn}{0303}{031B}
\@ifundefined{textdotbelowhorn}{\DeclareTextAccent{\textdotbelowhorn}{OT1}{999}}{}
\DeclareEncodedCompositeAccents{\UTFencname}{\textdotbelowhorn}{0323}{031B}
\@ifundefined{textogonekoverline}{\DeclareTextAccent{\textogonekoverline}{OT1}{999}}{}
\DeclareEncodedCompositeAccents{\UTFencname}{\textogonekoverline}{0328}{0304}

\@ifundefined{textmiddledot}{\DeclareTextAccent{\textmiddledot}{OT1}{999}}{}
\DeclareEncodedCompositeCharacter{\UTFencname}{\textmiddledot}{05BC}{05BC}

\DeclareEncodedCompositeCharacter{\UTFencname}{\textupstep}{A71B}{A71B}  % tone uparrow
\DeclareEncodedCompositeCharacter{\UTFencname}{\textdownstep}{A71C}{A71B}  % tone downarrow



% Constructed Characters, having no Unicode point
\def\textaolig{{a\kern-.25em o}}
\expandafter\def\csname\UTFencname\string\textturncelig\endcsname{{\textopeno\kern-.2em e}}
\def\textctdctzlig{{\textctd\kern-.08em\textctz}}
\def\textinvscripta{{\raise1ex\hbox{\scalebox{1}[-1]{\textscripta}}}}
\def\textinvsca{{\raise1ex\hbox{\scalebox{1}[-1]{\textsca}}}}
\def\textscaolig{{\textsca\kern-.2em o}}%\renewcommand{\textscaolig}{}
\def\textctstretchc{{\textpalhookbelow{\textstretchc}}}%\renewcommand{\textctstretchc}{}
\def\textctstretchcvar{{\textpalhookbelow{\textstretchc}}}%\renewcommand{\textctstretchcvar}{}
\def\textfrhookd{{\textpalhookbelow{d}}}\renewcommand{\textfrhookd}{d}
\def\textfrhookdvar{{\textpalhookbelow{d}}}\renewcommand{\textfrhookdvar}{d}
\def\textbktailgamma{{\textrethookbelow{\textgrgamma}}}
\def\textfrtailgamma{{\textpalhookbelow{\textgrgamma}}}
\def\textrtailhth{{\textrethookbelow{\texthth}}}
\def\textraisevibyi{{\raise.45ex\hbox{\textvibyi}}}
\def\textturnsck{{\raise1ex\hbox{\scalebox{-1}{\textsck}}}}
\def\textlfishhookrlig{{l\kern-.28em\textfishhookr}}
\def\textrevscl{{\scalebox{-1}[1]{\faketextsc{L}}}}
\def\texthmlig{{\raise1ex\hbox{\scalebox{-1}{\textturnmrleg}}}}
\def\textfrbarn{{\textpalhookbelow{n}}}\renewcommand{\textfrbarn}{}
\def\textinvomega{{\raise1ex\hbox{\scalebox{1}[-1]{\textomega}}}}
\def\textfrhookt{{\textpalhookbelow{t}}}\renewcommand{\textfrhookt}{}
\def\textctturnt{{\textpalhookbelow{\textturnt}}}\renewcommand{\textctturnt}{}
\def\textcttctclig{{\textctt\kern-.08em\textctc}}
\def\textturnscu{{\raise1ex\hbox{\scalebox{-1}{\faketextsc{U}}}}}
\def\textturntwo{{\raise1ex\hbox{\scalebox{-1}{2}}}}
\def\textturnthree{{\raise1ex\hbox{\scalebox{-1}{3}}}}
\def\textglotstopvariii{\textglotstop}%\renewcommand{\textglotstopvariii}{}
\def\textctinvglotstop{{\textpalhookbelow{\textinvglotstop}}}%\renewcommand{\textctinvglotstop}{}
\def\textturnglotstop{{\raise1ex\hbox{\scalebox{-1}{\textglotstop}}}}
\def\textdoublebarslash{{\textlstrikethru{=}}}
\def\textrevpolhook#1{{\let\tipaencoding\relax\tipaLoweraccent{\scalebox{-1}[1]{\textpolhook{}}}{#1}}}
\def\textoverw#1{{\let\tipaencoding\relax\tipaUpperaccent{\kern-.25em\scalebox{-1}[1]{\textsubw{}}}{#1}}}
\def\textretractingvar{}
\def\texthooktop{{\raise1ex\hbox{\kern1em\scalebox{-1}{\textpalhookbelow{\ }}}}}
\def\textpalhook{{\textpalhookbelow{\ }\kern1em}}% U+0321
\def\textpalhooklong{{\textpalhookbelow{\ }\kern1em}}% U+0321
\def\textpalhookvar{{\textpalhookbelow{\ }\kern1em}}% U+0321
\def\textrthook{{\textrethookbelow{\ }\kern1em}}% U+0322
\def\textrthooklong{{\textrethookbelow{\ }\kern1em}}% U+0322
\def\partvoiceless{}
\def\inipartvoiceless{}
\def\finpartvoiceless{}
\def\partvoice{}
\def\inipartvoice{}
\def\finpartvoice{}

% not many fonts support the correct characters: UxA71B, UxA71C
\def\textfakeupstep{{\raise.4ex\hbox{\scalebox{.7}{\textupfullarrow}}}}
\def\textfakedownstep{{\raise.45ex\hbox{\scalebox{.7}{\textdownfullarrow}}}}

%: separator

% setup the correspondence between Unicode characters and macro names

\DeclareUTFcharacter[\UTFencname]{x0022}{\textquotedbl}
\DeclareUTFcharacter[\UTFencname]{x0023}{\texthash}
\DeclareUTFcharacter[\UTFencname]{x0024}{\textdollar}
\DeclareUTFcharacter[\UTFencname]{x0025}{\textpercent}
\DeclareUTFcharacter[\UTFencname]{x0026}{\textampersand}
\DeclareUTFcharacter[\UTFencname]{x0027}{\textquotesingle}
\DeclareUTFcharacter[\UTFencname]{x002A}{\textasteriskcentered}
\DeclareUTFcharacter[\UTFencname]{x003C}{\textless}
\DeclareUTFcharacter[\UTFencname]{x003D}{\textequals}
\DeclareUTFcharacter[\UTFencname]{x003E}{\textgreater}
\DeclareUTFcharacter[\UTFencname]{x005C}{\textbackslash}
\DeclareUTFcharacter[\UTFencname]{x005E}{\textasciicircum}% see also x02C6
\DeclareUTFcharacter[\UTFencname]{x005F}{\textunderscore}
\DeclareUTFcharacter[\UTFencname]{x0060}{\textasciigrave}% see also x02CB
\DeclareUTFcharacter[\UTFencname]{x0067}{\textg}
\DeclareUTFcharacter[\UTFencname]{x007B}{\textbraceleft}
\DeclareUTFcharacter[\UTFencname]{x007C}{\textbar}
\DeclareUTFcharacter[\UTFencname]{x007D}{\textbraceright}
\DeclareUTFcharacter[\UTFencname]{x007E}{\textasciitilde}% see also x02DC
\DeclareUTFcharacter[\UTFencname]{x00A0}{\nobreakspace}
\DeclareUTFcharacter[\UTFencname]{x00A1}{\textexclamdown}
\DeclareUTFcharacter[\UTFencname]{x00A2}{\textcent}
\DeclareUTFcharacter[\UTFencname]{x00A3}{\textsterling}
\DeclareUTFcharacter[\UTFencname]{x00A4}{\textcurrency}
\DeclareUTFcharacter[\UTFencname]{x00A5}{\textyen}
\DeclareUTFcharacter[\UTFencname]{x00A6}{\textbrokenbar}
\DeclareUTFcharacter[\UTFencname]{x00A7}{\textsection}
\DeclareUTFcharacter[\UTFencname]{x00A8}{\textasciidieresis}
\DeclareUTFcharacter[\UTFencname]{x00A9}{\textcopyright}
\DeclareUTFcharacter[\UTFencname]{x00A9}{\copyright}
\DeclareUTFcharacter[\UTFencname]{x00AA}{\textordfeminine}
\DeclareUTFcomposite[\UTFencname]{x00AA}{\textsuperscript}{a}
\DeclareUTFcharacter[\UTFencname]{x00AB}{\guillemotleft}
\DeclareUTFcharacter[\UTFencname]{x00AC}{\textlogicalnot}
\DeclareUTFcharacter[\UTFencname]{x00AE}{\textregistered}
\DeclareUTFcharacter[\UTFencname]{x00AF}{\textasciimacron}% see also x02C9
\DeclareUTFcharacter[\UTFencname]{x00B0}{\textdegree}
\DeclareUTFcharacter[\UTFencname]{x00B1}{\textpm}
\DeclareUTFcharacter[\UTFencname]{x00B2}{\texttwosuperior}
\DeclareUTFcomposite[\UTFencname]{x00B2}{\textsuperscript}{2}
\DeclareUTFcharacter[\UTFencname]{x00B3}{\textthreesuperior}
\DeclareUTFcomposite[\UTFencname]{x00B3}{\textsuperscript}{3}
\DeclareUTFcharacter[\UTFencname]{x00B4}{\textasciiacute}% see also x02CA
\DeclareUTFcharacter[\UTFencname]{x00B5}{\textmu}
\DeclareUTFcharacter[\UTFencname]{x00B6}{\textparagraph}
\DeclareUTFcharacter[\UTFencname]{x00B6}{\textpilcrow}
\DeclareUTFcharacter[\UTFencname]{x00B7}{\textcentereddot}
\DeclareUTFcharacter[\UTFencname]{x00B7}{\textperiodcentered}
\DeclareUTFcharacter[\UTFencname]{x00B8}{\textasciicedilla}% see also x02DB
\DeclareUTFcharacter[\UTFencname]{x00B9}{\textonesuperior}
\DeclareUTFcomposite[\UTFencname]{x00B9}{\textsuperscript}{1}
\DeclareUTFcharacter[\UTFencname]{x00BA}{\textordmasculine}
\DeclareUTFcomposite[\UTFencname]{x00BA}{\textsuperscript}{o}
\DeclareUTFcharacter[\UTFencname]{x00BB}{\guillemotright}
\DeclareUTFcharacter[\UTFencname]{x00BC}{\textonequarter}
\DeclareUTFcharacter[\UTFencname]{x00BD}{\textonehalf}
\DeclareUTFcharacter[\UTFencname]{x00BE}{\textthreequarters}
\DeclareUTFcharacter[\UTFencname]{x00BF}{\textquestiondown}
\DeclareUTFcomposite[\UTFencname]{x00C0}{\`}{A}
\DeclareUTFcomposite[\UTFencname]{x00C0}{\capitalgrave}{A}
\DeclareUTFcomposite[\UTFencname]{x00C1}{\'}{A}
\DeclareUTFcomposite[\UTFencname]{x00C1}{\capitalacute}{A}
\DeclareUTFcomposite[\UTFencname]{x00C2}{\^}{A}
\DeclareUTFcomposite[\UTFencname]{x00C2}{\capitalcircumflex}{A}
\DeclareUTFcomposite[\UTFencname]{x00C3}{\~}{A}
\DeclareUTFcomposite[\UTFencname]{x00C3}{\capitaltilde}{A}
\DeclareUTFcomposite[\UTFencname]{x00C4}{\"}{A}
\DeclareUTFcomposite[\UTFencname]{x00C4}{\capitaldieresis}{A}
\DeclareUTFcomposite[\UTFencname]{x00C5}{\r}{A}
\DeclareUTFcomposite[\UTFencname]{x00C5}{\capitalring}{A}
\DeclareUTFcharacter[\UTFencname]{x00C5}{\AA}
\DeclareUTFcharacter[\UTFencname]{x00C6}{\AE}
\DeclareUTFcomposite[\UTFencname]{x00C7}{\c}{C}
\DeclareUTFcomposite[\UTFencname]{x00C7}{\capitalcedilla}{C}
\DeclareUTFcomposite[\UTFencname]{x00C8}{\`}{E}
\DeclareUTFcomposite[\UTFencname]{x00C8}{\capitalgrave}{E}
\DeclareUTFcomposite[\UTFencname]{x00C9}{\'}{E}
\DeclareUTFcomposite[\UTFencname]{x00C9}{\capitalacute}{E}
\DeclareUTFcomposite[\UTFencname]{x00CA}{\^}{E}
\DeclareUTFcomposite[\UTFencname]{x00CA}{\capitalcircumflex}{E}
\DeclareUTFcomposite[\UTFencname]{x00CB}{\"}{E}
\DeclareUTFcomposite[\UTFencname]{x00CB}{\capitaldieresis}{E}
\DeclareUTFcomposite[\UTFencname]{x00CC}{\`}{I}
\DeclareUTFcomposite[\UTFencname]{x00CC}{\capitalgrave}{I}
\DeclareUTFcomposite[\UTFencname]{x00CD}{\'}{I}
\DeclareUTFcomposite[\UTFencname]{x00CD}{\capitalacute}{I}
\DeclareUTFcomposite[\UTFencname]{x00CE}{\^}{I}
\DeclareUTFcomposite[\UTFencname]{x00CE}{\capitalcircumflex}{I}
\DeclareUTFcomposite[\UTFencname]{x00CF}{\"}{I}
\DeclareUTFcomposite[\UTFencname]{x00CF}{\capitaldieresis}{I}
\DeclareUTFcomposite[\UTFencname]{x00D0}{\M}{D}
\DeclareUTFcharacter[\UTFencname]{x00D0}{\DH}
\DeclareUTFcomposite[\UTFencname]{x00D1}{\~}{N}
\DeclareUTFcomposite[\UTFencname]{x00D1}{\capitaltilde}{N}
\DeclareUTFcomposite[\UTFencname]{x00D2}{\`}{O}
\DeclareUTFcomposite[\UTFencname]{x00D2}{\capitalgrave}{O}
\DeclareUTFcomposite[\UTFencname]{x00D3}{\'}{O}
\DeclareUTFcomposite[\UTFencname]{x00D3}{\capitalacute}{O}
\DeclareUTFcomposite[\UTFencname]{x00D4}{\^}{O}
\DeclareUTFcomposite[\UTFencname]{x00D4}{\capitalcircumflex}{O}
\DeclareUTFcomposite[\UTFencname]{x00D5}{\~}{O}
\DeclareUTFcomposite[\UTFencname]{x00D5}{\capitaltilde}{O}
\DeclareUTFcomposite[\UTFencname]{x00D6}{\"}{O}
\DeclareUTFcomposite[\UTFencname]{x00D6}{\capitaldieresis}{O}
\DeclareUTFcharacter[\UTFencname]{x00D7}{\texttimes}
\DeclareUTFcharacter[\UTFencname]{x00D8}{\O}
\DeclareUTFcomposite[\UTFencname]{x00D9}{\`}{U}
\DeclareUTFcomposite[\UTFencname]{x00D9}{\capitalgrave}{U}
\DeclareUTFcomposite[\UTFencname]{x00DA}{\'}{U}
\DeclareUTFcomposite[\UTFencname]{x00DA}{\capitalacute}{U}
\DeclareUTFcomposite[\UTFencname]{x00DB}{\^}{U}
\DeclareUTFcomposite[\UTFencname]{x00DB}{\capitalcircumflex}{U}
\DeclareUTFcomposite[\UTFencname]{x00DC}{\"}{U}
\DeclareUTFcomposite[\UTFencname]{x00DC}{\capitaldieresis}{U}
\DeclareUTFcomposite[\UTFencname]{x00DD}{\'}{Y}
\DeclareUTFcomposite[\UTFencname]{x00DD}{\capitalacute}{Y}
\DeclareUTFcharacter[\UTFencname]{x00DE}{\TH}
\DeclareUTFcharacter[\UTFencname]{x00DE}{\Thorn}
\DeclareUTFcharacter[\UTFencname]{x00DF}{\ss}    % TIPA-B
\DeclareUTFcomposite[\UTFencname]{x00E0}{\`}{a}
\DeclareUTFcomposite[\UTFencname]{x00E1}{\'}{a}
\DeclareUTFcomposite[\UTFencname]{x00E2}{\^}{a}
\DeclareUTFcomposite[\UTFencname]{x00E3}{\~}{a}
\DeclareUTFcomposite[\UTFencname]{x00E4}{\"}{a}
\DeclareUTFcomposite[\UTFencname]{x00E5}{\r}{a}
\DeclareUTFcharacter[\UTFencname]{x00E5}{\aa}
\DeclareUTFcharacter[\UTFencname]{x00E6}{\ae}
\DeclareUTFcomposite[\UTFencname]{x00E7}{\c}{c}
\DeclareUTFcomposite[\UTFencname]{x00E8}{\`}{e}
\DeclareUTFcomposite[\UTFencname]{x00E9}{\'}{e}
\DeclareUTFcomposite[\UTFencname]{x00EA}{\^}{e}
\DeclareUTFcomposite[\UTFencname]{x00EB}{\"}{e}
\DeclareUTFcomposite[\UTFencname]{x00EC}{\`}{i}
\DeclareUTFcomposite[\UTFencname]{x00EC}{\`}{\i}
\DeclareUTFcomposite[\UTFencname]{x00ED}{\'}{i}
\DeclareUTFcomposite[\UTFencname]{x00ED}{\'}{\i}
\DeclareUTFcomposite[\UTFencname]{x00EE}{\^}{i}
\DeclareUTFcomposite[\UTFencname]{x00EE}{\^}{\i}
\DeclareUTFcomposite[\UTFencname]{x00EF}{\"}{i}
\DeclareUTFcomposite[\UTFencname]{x00EF}{\"}{\i}
\DeclareUTFcharacter[\UTFencname]{x00F0}{\dh}   % TIPA-D
\DeclareUTFcomposite[\UTFencname]{x00F1}{\~}{n}
\DeclareUTFcomposite[\UTFencname]{x00F2}{\`}{o}
\DeclareUTFcomposite[\UTFencname]{x00F3}{\'}{o}
\DeclareUTFcomposite[\UTFencname]{x00F4}{\^}{o}
\DeclareUTFcomposite[\UTFencname]{x00F5}{\~}{o}
\DeclareUTFcomposite[\UTFencname]{x00F6}{\"}{o}
\DeclareUTFcharacter[\UTFencname]{x00F7}{\textdiv}
\DeclareUTFcharacter[\UTFencname]{x00F8}{\o}
\DeclareUTFcomposite[\UTFencname]{x00F9}{\`}{u}
\DeclareUTFcomposite[\UTFencname]{x00FA}{\'}{u}
\DeclareUTFcomposite[\UTFencname]{x00FB}{\^}{u}
\DeclareUTFcomposite[\UTFencname]{x00FC}{\"}{u}
\DeclareUTFcomposite[\UTFencname]{x00FD}{\'}{y}
\DeclareUTFcharacter[\UTFencname]{x00FE}{\th}
\DeclareUTFcharacter[\UTFencname]{x00FE}{\textthorn}
\DeclareUTFcharacter[\UTFencname]{x00FE}{\textthornvari} % ?? IPA
\DeclareUTFcharacter[\UTFencname]{x00FE}{\textthornvarii} % ?? IPA
\DeclareUTFcharacter[\UTFencname]{x00FE}{\textthornvariii} % ?? IPA
\DeclareUTFcharacter[\UTFencname]{x00FE}{\textthornvariv} % ?? IPA
\DeclareUTFcomposite[\UTFencname]{x00FF}{\"}{y}
\DeclareUTFcomposite[\UTFencname]{x0100}{\=}{A}
\DeclareUTFcomposite[\UTFencname]{x0100}{\capitalmacron}{A}
\DeclareUTFcomposite[\UTFencname]{x0101}{\=}{a}
\DeclareUTFcomposite[\UTFencname]{x0102}{\u}{A}
\DeclareUTFcomposite[\UTFencname]{x0102}{\capitalbreve}{A}
\DeclareUTFcomposite[\UTFencname]{x0103}{\u}{a}
\DeclareUTFcomposite[\UTFencname]{x0104}{\k}{A}
\DeclareUTFcomposite[\UTFencname]{x0104}{\capitalogonek}{A}
\DeclareUTFcharacter[\UTFencname]{x0104}{\Aogonek} %  regi-cp1250
\DeclareUTFcomposite[\UTFencname]{x0105}{\k}{a}
\DeclareUTFcomposite[\UTFencname]{x0105}{\capitalogonek}{a}
\DeclareUTFcharacter[\UTFencname]{x0105}{\aogonek} %  regi-cp1250
\DeclareUTFcomposite[\UTFencname]{x0106}{\'}{C}
\DeclareUTFcomposite[\UTFencname]{x0106}{\capitalacute}{C}
\DeclareUTFcomposite[\UTFencname]{x0107}{\'}{c}
\DeclareUTFcomposite[\UTFencname]{x0108}{\^}{C}
\DeclareUTFcomposite[\UTFencname]{x0108}{\capitalcircumflex}{C}
\DeclareUTFcomposite[\UTFencname]{x0109}{\^}{c}
\DeclareUTFcomposite[\UTFencname]{x010A}{\.}{C}
\DeclareUTFcomposite[\UTFencname]{x010A}{\capitaldotaccent}{C}
\DeclareUTFcomposite[\UTFencname]{x010B}{\.}{c}
\DeclareUTFcomposite[\UTFencname]{x010C}{\v}{C}
\DeclareUTFcomposite[\UTFencname]{x010C}{\capitalcaron}{C}
\DeclareUTFcomposite[\UTFencname]{x010D}{\v}{c}
\DeclareUTFcomposite[\UTFencname]{x010E}{\v}{D}
\DeclareUTFcomposite[\UTFencname]{x010E}{\capitalcaron}{D}
\DeclareUTFcomposite[\UTFencname]{x010F}{\v}{d}
\DeclareUTFcomposite[\UTFencname]{x0110}{\B}{D}
\DeclareUTFcharacter[\UTFencname]{x0110}{\DJ}
\DeclareUTFcomposite[\UTFencname]{x0111}{\B}{d}
\DeclareUTFcharacter[\UTFencname]{x0111}{\dj}
\DeclareUTFcharacter[\UTFencname]{x0111}{\textcrd}
\DeclareUTFcomposite[\UTFencname]{x0112}{\=}{E}
\DeclareUTFcomposite[\UTFencname]{x0112}{\capitalmacron}{E}
\DeclareUTFcomposite[\UTFencname]{x0113}{\=}{e}
\DeclareUTFcomposite[\UTFencname]{x0114}{\u}{E}
\DeclareUTFcomposite[\UTFencname]{x0114}{\capitalbreve}{E}
\DeclareUTFcomposite[\UTFencname]{x0115}{\u}{e}
\DeclareUTFcomposite[\UTFencname]{x0116}{\.}{E}
\DeclareUTFcomposite[\UTFencname]{x0116}{\capitaldotaccent}{E}
\DeclareUTFcomposite[\UTFencname]{x0117}{\.}{e}
\DeclareUTFcomposite[\UTFencname]{x0118}{\k}{E}
\DeclareUTFcomposite[\UTFencname]{x0118}{\capitalogonek}{E}
\DeclareUTFcomposite[\UTFencname]{x0119}{\k}{e}
\DeclareUTFcomposite[\UTFencname]{x0119}{\capitalogonek}{e}
\DeclareUTFcomposite[\UTFencname]{x011A}{\v}{E}
\DeclareUTFcomposite[\UTFencname]{x011A}{\capitalcaron}{E}
\DeclareUTFcomposite[\UTFencname]{x011B}{\v}{e}
\DeclareUTFcomposite[\UTFencname]{x011C}{\^}{G}
\DeclareUTFcomposite[\UTFencname]{x011C}{\capitalcircumflex}{G}
\DeclareUTFcomposite[\UTFencname]{x011D}{\^}{g}
\DeclareUTFcomposite[\UTFencname]{x011E}{\u}{G}
\DeclareUTFcomposite[\UTFencname]{x011E}{\capitalbreve}{G}
\DeclareUTFcomposite[\UTFencname]{x011F}{\u}{g}
\DeclareUTFcomposite[\UTFencname]{x0120}{\.}{G}
\DeclareUTFcomposite[\UTFencname]{x0120}{\capitaldotaccent}{G}
\DeclareUTFcomposite[\UTFencname]{x0121}{\.}{g}
\DeclareUTFcomposite[\UTFencname]{x0122}{\c}{G}
\DeclareUTFcomposite[\UTFencname]{x0122}{\capitalcedilla}{G}
\DeclareUTFcomposite[\UTFencname]{x0123}{\c}{g}
\DeclareUTFcomposite[\UTFencname]{x0124}{\^}{H}
\DeclareUTFcomposite[\UTFencname]{x0124}{\capitalcircumflex}{H}
\DeclareUTFcomposite[\UTFencname]{x0125}{\^}{h}
\DeclareUTFcomposite[\UTFencname]{x0126}{\B}{H}
\DeclareUTFcharacter[\UTFencname]{x0126}{\textHbar}
\DeclareUTFcomposite[\UTFencname]{x0127}{\B}{h}
\DeclareUTFcharacter[\UTFencname]{x0127}{\textcrh}
\DeclareUTFcharacter[\UTFencname]{x0127}{\texthbar}
\DeclareUTFcomposite[\UTFencname]{x0128}{\~}{I}
\DeclareUTFcomposite[\UTFencname]{x0128}{\capitaltilde}{I}
\DeclareUTFcomposite[\UTFencname]{x0129}{\~}{i}
\DeclareUTFcomposite[\UTFencname]{x0129}{\~}{\i}
\DeclareUTFcomposite[\UTFencname]{x012A}{\=}{I}
\DeclareUTFcomposite[\UTFencname]{x012A}{\capitalmacron}{I}
\DeclareUTFcomposite[\UTFencname]{x012B}{\=}{i}
\DeclareUTFcomposite[\UTFencname]{x012B}{\=}{\i}
\DeclareUTFcomposite[\UTFencname]{x012C}{\u}{I}
\DeclareUTFcomposite[\UTFencname]{x012C}{\capitalbreve}{I}
\DeclareUTFcomposite[\UTFencname]{x012D}{\u}{i}
\DeclareUTFcomposite[\UTFencname]{x012D}{\u}{\i}
\DeclareUTFcomposite[\UTFencname]{x012E}{\k}{I}
\DeclareUTFcomposite[\UTFencname]{x012E}{\capitalogonek}{I}
\DeclareUTFcomposite[\UTFencname]{x012F}{\k}{i}
\DeclareUTFcomposite[\UTFencname]{x012F}{\capitalogonek}{i}
\DeclareUTFcomposite[\UTFencname]{x0130}{\.}{I}
\DeclareUTFcomposite[\UTFencname]{x0130}{\capitaldotaccent}{I}
\DeclareUTFcharacter[\UTFencname]{x0131}{\i}
\DeclareUTFcharacter[\UTFencname]{x0132}{\IJ}
\DeclareUTFcharacter[\UTFencname]{x0133}{\ij}
\DeclareUTFcomposite[\UTFencname]{x0134}{\^}{J}
\DeclareUTFcomposite[\UTFencname]{x0134}{\capitalcircumflex}{J}
\DeclareUTFcomposite[\UTFencname]{x0135}{\^}{j}
\DeclareUTFcomposite[\UTFencname]{x0135}{\^}{\j}
\DeclareUTFcomposite[\UTFencname]{x0136}{\c}{K}
\DeclareUTFcomposite[\UTFencname]{x0136}{\capitalcedilla}{K}
\DeclareUTFcomposite[\UTFencname]{x0137}{\c}{k}
\DeclareUTFcharacter[\UTFencname]{x0138}{\textkra}
\DeclareUTFcomposite[\UTFencname]{x0139}{\'}{L}
\DeclareUTFcomposite[\UTFencname]{x0139}{\capitalacute}{L}
\DeclareUTFcomposite[\UTFencname]{x013A}{\'}{l}
\DeclareUTFcomposite[\UTFencname]{x013B}{\c}{L}
\DeclareUTFcomposite[\UTFencname]{x013B}{\capitalcedilla}{L}
\DeclareUTFcomposite[\UTFencname]{x013C}{\c}{l}
\DeclareUTFcomposite[\UTFencname]{x013D}{\v}{L}
\DeclareUTFcomposite[\UTFencname]{x013D}{\capitalcaron}{L}
\DeclareUTFcharacter[\UTFencname]{x013D}{\Lcaron} %  regi-cp1250
\DeclareUTFcomposite[\UTFencname]{x013E}{\v}{l}
\DeclareUTFcharacter[\UTFencname]{x013E}{\lcaron} %  regi-cp1250
\DeclareUTFcomposite[\UTFencname]{x013F}{\textmiddledot}{L} % with middle dot
\DeclareUTFcomposite[\UTFencname]{x0140}{\textmiddledot}{l} % with middle dot
\DeclareUTFcomposite[\UTFencname]{x0141}{\B}{L}
\DeclareUTFcharacter[\UTFencname]{x0141}{\L}
\DeclareUTFcharacter[\UTFencname]{x0141}{\Lstroke} %  regi-cp1250
\DeclareUTFcomposite[\UTFencname]{x0142}{\B}{l}
\DeclareUTFcharacter[\UTFencname]{x0142}{\l}
\DeclareUTFcharacter[\UTFencname]{x0142}{\lstroke} %  regi-cp1250
\DeclareUTFcharacter[\UTFencname]{x0142}{\textbarl}
\DeclareUTFcomposite[\UTFencname]{x0143}{\'}{N}
\DeclareUTFcomposite[\UTFencname]{x0143}{\capitalacute}{N}
\DeclareUTFcomposite[\UTFencname]{x0144}{\'}{n}
\DeclareUTFcomposite[\UTFencname]{x0145}{\c}{N}
\DeclareUTFcomposite[\UTFencname]{x0145}{\capitalcedilla}{N}
\DeclareUTFcomposite[\UTFencname]{x0146}{\c}{n}
\DeclareUTFcomposite[\UTFencname]{x0147}{\v}{N}
\DeclareUTFcomposite[\UTFencname]{x0147}{\capitalcaron}{N}
\DeclareUTFcomposite[\UTFencname]{x0148}{\v}{n}
\DeclareUTFcomposite[\UTFencname]{x0149}{\textcommaabove}{n}
\DeclareUTFcharacter[\UTFencname]{x0149}{'n}
\DeclareUTFcomposite[\UTFencname]{x014A}{\m}{N}
\DeclareUTFcharacter[\UTFencname]{x014A}{\NG}
\DeclareUTFcomposite[\UTFencname]{x014B}{\m}{n}
\DeclareUTFcharacter[\UTFencname]{x014B}{\ng}   %  TIPA-N
\DeclareUTFcomposite[\UTFencname]{x014C}{\=}{O}
\DeclareUTFcomposite[\UTFencname]{x014C}{\capitalmacron}{O}
\DeclareUTFcomposite[\UTFencname]{x014D}{\=}{o}
\DeclareUTFcomposite[\UTFencname]{x014E}{\u}{O}
\DeclareUTFcomposite[\UTFencname]{x014E}{\capitalbreve}{O}
\DeclareUTFcomposite[\UTFencname]{x014F}{\u}{o}
\DeclareUTFcomposite[\UTFencname]{x0150}{\H}{O}
\DeclareUTFcomposite[\UTFencname]{x0150}{\capitalhungarumlaut}{O}
\DeclareUTFcomposite[\UTFencname]{x0151}{\H}{o}
\DeclareUTFcharacter[\UTFencname]{x0152}{\OE}
\DeclareUTFcharacter[\UTFencname]{x0153}{\oe}
\DeclareUTFcomposite[\UTFencname]{x0154}{\'}{R}
\DeclareUTFcomposite[\UTFencname]{x0154}{\capitalacute}{R}
\DeclareUTFcharacter[\UTFencname]{x0154}{\Racute} %  regi-cp1250
\DeclareUTFcomposite[\UTFencname]{x0155}{\'}{r}
\DeclareUTFcharacter[\UTFencname]{x0155}{\racute} %  regi-cp1250
\DeclareUTFcomposite[\UTFencname]{x0156}{\c}{R}
\DeclareUTFcomposite[\UTFencname]{x0156}{\capitalcedilla}{R}
\DeclareUTFcomposite[\UTFencname]{x0157}{\c}{r}
\DeclareUTFcomposite[\UTFencname]{x0158}{\v}{R}
\DeclareUTFcomposite[\UTFencname]{x0158}{\capitalcaron}{R}
\DeclareUTFcomposite[\UTFencname]{x0159}{\v}{r}
\DeclareUTFcomposite[\UTFencname]{x015A}{\'}{S}
\DeclareUTFcomposite[\UTFencname]{x015A}{\capitalacute}{S}
\DeclareUTFcharacter[\UTFencname]{x015A}{\Sacute} %  regi-cp1250
\DeclareUTFcomposite[\UTFencname]{x015B}{\'}{s}
\DeclareUTFcharacter[\UTFencname]{x015B}{\sacute} %  regi-cp1250
\DeclareUTFcomposite[\UTFencname]{x015C}{\^}{S}
\DeclareUTFcomposite[\UTFencname]{x015C}{\capitalcircumflex}{S}
\DeclareUTFcomposite[\UTFencname]{x015D}{\^}{s}
\DeclareUTFcomposite[\UTFencname]{x015E}{\c}{S}
\DeclareUTFcomposite[\UTFencname]{x015E}{\capitalcedilla}{S}
\DeclareUTFcharacter[\UTFencname]{x015E}{\Scedilla} %  regi-cp1250
\DeclareUTFcomposite[\UTFencname]{x015F}{\c}{s}
\DeclareUTFcharacter[\UTFencname]{x015F}{\scedilla} %  regi-cp1250
\DeclareUTFcomposite[\UTFencname]{x0160}{\v}{S}
\DeclareUTFcomposite[\UTFencname]{x0160}{\capitalcaron}{S}
\DeclareUTFcharacter[\UTFencname]{x0160}{\Scaron} %  regi-cp1250 
\DeclareUTFcomposite[\UTFencname]{x0161}{\v}{s}
\DeclareUTFcharacter[\UTFencname]{x0161}{\scaron} %  regi-cp1250 
\DeclareUTFcomposite[\UTFencname]{x0162}{\c}{T}
\DeclareUTFcomposite[\UTFencname]{x0162}{\capitalcedilla}{T}
\DeclareUTFcomposite[\UTFencname]{x0163}{\c}{t}
\DeclareUTFcomposite[\UTFencname]{x0164}{\v}{T}
\DeclareUTFcomposite[\UTFencname]{x0164}{\capitalcaron}{T}
\DeclareUTFcharacter[\UTFencname]{x0164}{\Tcaron} %  regi-cp1250
\DeclareUTFcomposite[\UTFencname]{x0165}{\v}{t}
\DeclareUTFcharacter[\UTFencname]{x0165}{\tcaron} %  regi-cp1250
\DeclareUTFcomposite[\UTFencname]{x0166}{\B}{T}
\DeclareUTFcharacter[\UTFencname]{x0166}{\textTstroke}
\DeclareUTFcharacter[\UTFencname]{x0166}{\textTbar}
\DeclareUTFcomposite[\UTFencname]{x0167}{\B}{t}
\DeclareUTFcharacter[\UTFencname]{x0167}{\texttstroke}
\DeclareUTFcharacter[\UTFencname]{x0167}{\texttbar}
\DeclareUTFcomposite[\UTFencname]{x0168}{\~}{U}
\DeclareUTFcomposite[\UTFencname]{x0168}{\capitaltilde}{U}
\DeclareUTFcomposite[\UTFencname]{x0169}{\~}{u}
\DeclareUTFcomposite[\UTFencname]{x016A}{\=}{U}
\DeclareUTFcomposite[\UTFencname]{x016A}{\capitalmacron}{U}
\DeclareUTFcomposite[\UTFencname]{x016B}{\=}{u}
\DeclareUTFcomposite[\UTFencname]{x016C}{\u}{U}
\DeclareUTFcomposite[\UTFencname]{x016C}{\capitalbreve}{U}
\DeclareUTFcomposite[\UTFencname]{x016D}{\u}{u}
\DeclareUTFcomposite[\UTFencname]{x016E}{\r}{U}
\DeclareUTFcomposite[\UTFencname]{x016E}{\capitalring}{U}
\DeclareUTFcomposite[\UTFencname]{x016F}{\r}{u}
\DeclareUTFcomposite[\UTFencname]{x0170}{\H}{U}
\DeclareUTFcomposite[\UTFencname]{x0170}{\capitalhungarumlaut}{U}
\DeclareUTFcomposite[\UTFencname]{x0171}{\H}{u}
\DeclareUTFcomposite[\UTFencname]{x0172}{\k}{U}
\DeclareUTFcomposite[\UTFencname]{x0172}{\capitalogonek}{U}
\DeclareUTFcomposite[\UTFencname]{x0173}{\k}{u}
\DeclareUTFcomposite[\UTFencname]{x0173}{\capitalogonek}{u}
\DeclareUTFcomposite[\UTFencname]{x0174}{\^}{W}
\DeclareUTFcomposite[\UTFencname]{x0174}{\capitalcircumflex}{W}
\DeclareUTFcomposite[\UTFencname]{x0175}{\^}{w}
\DeclareUTFcomposite[\UTFencname]{x0176}{\^}{Y}
\DeclareUTFcomposite[\UTFencname]{x0176}{\capitalcircumflex}{Y}
\DeclareUTFcomposite[\UTFencname]{x0177}{\^}{y}
\DeclareUTFcomposite[\UTFencname]{x0178}{\"}{Y}
\DeclareUTFcomposite[\UTFencname]{x0178}{\capitaldieresis}{Y}
\DeclareUTFcomposite[\UTFencname]{x0179}{\'}{Z}
\DeclareUTFcomposite[\UTFencname]{x0179}{\capitalacute}{Z}
\DeclareUTFcomposite[\UTFencname]{x017A}{\'}{z}
\DeclareUTFcomposite[\UTFencname]{x017B}{\.}{Z}
\DeclareUTFcomposite[\UTFencname]{x017B}{\capitaldotaccent}{Z}
\DeclareUTFcharacter[\UTFencname]{x017B}{\Zdotaccent} %  regi-cp1250
\DeclareUTFcomposite[\UTFencname]{x017C}{\.}{z}
\DeclareUTFcharacter[\UTFencname]{x017C}{\zdotaccent} %  regi-cp1250
\DeclareUTFcomposite[\UTFencname]{x017D}{\v}{Z}
\DeclareUTFcomposite[\UTFencname]{x017D}{\capitalcaron}{Z}
\DeclareUTFcharacter[\UTFencname]{x017D}{\Zcaron} %  regi-cp1250
\DeclareUTFcomposite[\UTFencname]{x017E}{\v}{z}
\DeclareUTFcharacter[\UTFencname]{x017E}{\zcaron} %  regi-cp1250

\DeclareUTFcomposite[\UTFencname]{x0180}{\B}{b}
\DeclareUTFcharacter[\UTFencname]{x0180}{\textcrb}
\DeclareUTFcomposite[\UTFencname]{x0181}{\m}{B}
\DeclareUTFcharacter[\UTFencname]{x0181}{\textBhook}
\DeclareUTFcomposite[\UTFencname]{x0182}{\textoverline}{B} % with topbar
\DeclareUTFcomposite[\UTFencname]{x0183}{\textoverline}{b} % with topbar
\DeclareUTFcomposite[\UTFencname]{x0186}{\m}{O}
\DeclareUTFcharacter[\UTFencname]{x0186}{\textOopen}
\DeclareUTFcomposite[\UTFencname]{x0187}{\m}{C}
\DeclareUTFcharacter[\UTFencname]{x0187}{\textChook}
\DeclareUTFcomposite[\UTFencname]{x0188}{\m}{c}
\DeclareUTFcharacter[\UTFencname]{x0188}{\texthtc}
\DeclareUTFcharacter[\UTFencname]{x0188}{\textchook}
\DeclareUTFcomposite[\UTFencname]{x0189}{\M}{D}
\DeclareUTFcharacter[\UTFencname]{x0189}{\textDafrican}
\DeclareUTFcomposite[\UTFencname]{x018A}{\m}{D}
\DeclareUTFcharacter[\UTFencname]{x018A}{\textDhook}
\DeclareUTFcomposite[\UTFencname]{x018B}{\textoverline}{D} % with topbar
\DeclareUTFcomposite[\UTFencname]{x018C}{\textoverline}{d} % with topbar
\DeclareUTFcharacter[\UTFencname]{x018E}{\textEreversed} % reversed E
\DeclareUTFcomposite[\UTFencname]{x018E}{\M}{E}
\DeclareUTFcomposite[\UTFencname]{x0190}{\m}{E} % Open E
\DeclareUTFcharacter[\UTFencname]{x0190}{\textEopen} % Open E
\DeclareUTFcomposite[\UTFencname]{x0191}{\m}{F}
\DeclareUTFcharacter[\UTFencname]{x0191}{\textFhook} % with hook
\DeclareUTFcomposite[\UTFencname]{x0192}{\m}{f}
\DeclareUTFcharacter[\UTFencname]{x0192}{\textflorin} % with hook
\DeclareUTFcomposite[\UTFencname]{x0193}{\texthookabove}{G} % with hook above
\DeclareUTFcomposite[\UTFencname]{x0194}{\m}{G} % capital Gamma
\DeclareUTFcharacter[\UTFencname]{x0194}{\textgamma}  %  TIPA-G
\DeclareUTFcharacter[\UTFencname]{x0194}{\textGammaafrican}
\DeclareUTFcharacter[\UTFencname]{x0195}{\hv}
\DeclareUTFcharacter[\UTFencname]{x0195}{\texthvlig}
\DeclareUTFcomposite[\UTFencname]{x0196}{\m}{I}
\DeclareUTFcharacter[\UTFencname]{x0196}{\textIotaafrican} %  ?? 
\DeclareUTFcomposite[\UTFencname]{x0197}{\B}{I}
\DeclareUTFcomposite[\UTFencname]{x0198}{\m}{K}
\DeclareUTFcharacter[\UTFencname]{x0198}{\textKhook}
\DeclareUTFcomposite[\UTFencname]{x0199}{\m}{k}
\DeclareUTFcharacter[\UTFencname]{x0199}{\texthtk}
\DeclareUTFcharacter[\UTFencname]{x0199}{\textkhook}
\DeclareUTFcomposite[\UTFencname]{x019A}{\B}{l}
\DeclareUTFcharacter[\UTFencname]{x019A}{\textbarl}
\DeclareUTFcharacter[\UTFencname]{x019B}{\textcrlambda}
\DeclareUTFcomposite[\UTFencname]{x019D}{\m}{J}
\DeclareUTFcharacter[\UTFencname]{x019D}{\textNhookleft}
\DeclareUTFcharacter[\UTFencname]{x019E}{\textnrleg} % N with long right leg
\DeclareUTFcharacter[\UTFencname]{x01A0}{\Ohorn} % O with horn
\DeclareUTFcomposite[\UTFencname]{x01A0}{\textrighthorn}{O} % O with horn
\DeclareUTFcharacter[\UTFencname]{x01A1}{\ohorn} % o with horn
\DeclareUTFcomposite[\UTFencname]{x01A1}{\textrighthorn}{o} % O with horn
\DeclareUTFcomposite[\UTFencname]{x01A4}{\m}{P}
\DeclareUTFcharacter[\UTFencname]{x01A4}{\textPhook}
\DeclareUTFcomposite[\UTFencname]{x01A5}{\m}{p}
\DeclareUTFcharacter[\UTFencname]{x01A5}{\texthtp}
\DeclareUTFcharacter[\UTFencname]{x01A5}{\textphook}
\DeclareUTFcharacter[\UTFencname]{x01A9}{\ESH}
\DeclareUTFcharacter[\UTFencname]{x01A9}{\textEsh}
\DeclareUTFcharacter[\UTFencname]{x01AA}{\textlooptoprevesh} % reversed ESH loop
\DeclareUTFcharacter[\UTFencname]{x01AA}{\textlhtlongi} % Left-hooktop Long I
\DeclareUTFcomposite[\UTFencname]{x01AB}{\textpalhookbelow}{t} % t with palatal hook
\DeclareUTFcharacter[\UTFencname]{x01AB}{\textlhookt}
\DeclareUTFcomposite[\UTFencname]{x01AC}{\m}{T}
\DeclareUTFcharacter[\UTFencname]{x01AC}{\textThook}
\DeclareUTFcomposite[\UTFencname]{x01AD}{\m}{t}
\DeclareUTFcharacter[\UTFencname]{x01AD}{\texthtt}
\DeclareUTFcharacter[\UTFencname]{x01AD}{\textthook}
\DeclareUTFcomposite[\UTFencname]{x01AE}{\M}{T}
\DeclareUTFcharacter[\UTFencname]{x01AE}{\textTretroflexhook}
\DeclareUTFcharacter[\UTFencname]{x01AF}{\Uhorn} % U with horn
\DeclareUTFcomposite[\UTFencname]{x01AF}{\textrighthorn}{U} % U with horn
\DeclareUTFcharacter[\UTFencname]{x01B0}{\uhorn} % u with horn
\DeclareUTFcomposite[\UTFencname]{x01B0}{\textrighthorn}{u} % u with horn
\DeclareUTFcharacter[\UTFencname]{x01B1}{\textupsilon}
\DeclareUTFcomposite[\UTFencname]{x01B1}{\m}{U}
\DeclareUTFcomposite[\UTFencname]{x01B2}{\m}{V}
\DeclareUTFcharacter[\UTFencname]{x01B2}{\textVhook}
\DeclareUTFcomposite[\UTFencname]{x01B3}{\m}{Y}
\DeclareUTFcharacter[\UTFencname]{x01B3}{\textYhook}
\DeclareUTFcomposite[\UTFencname]{x01B4}{\m}{y}
\DeclareUTFcharacter[\UTFencname]{x01B4}{\textyhook}
\DeclareUTFcomposite[\UTFencname]{x01B5}{\B}{Z}
\DeclareUTFcharacter[\UTFencname]{x01B5}{\Zbar}
\DeclareUTFcomposite[\UTFencname]{x01B6}{\B}{z}
\DeclareUTFcomposite[\UTFencname]{x01B7}{\m}{Z}
\DeclareUTFcharacter[\UTFencname]{x01B7}{\EZH}
\DeclareUTFcharacter[\UTFencname]{x01B7}{\textEzh}
\DeclareUTFcharacter[\UTFencname]{x01B9}{\textrevyogh}
\DeclareUTFcharacter[\UTFencname]{x01BA}{\textbenttailyogh} % ezh with tail
\DeclareUTFcomposite[\UTFencname]{x01BB}{\B}{2}
\DeclareUTFcharacter[\UTFencname]{x01BB}{\textcrtwo}
\DeclareUTFcharacter[\UTFencname]{x01BE}{\textcrinvglotstop} % inverted glottal with stroke
\DeclareUTFcharacter[\UTFencname]{x01BF}{\wynn}
\DeclareUTFcharacter[\UTFencname]{x01C0}{\textpipe} % dental click  %  TIPA-|
\DeclareUTFcharacter[\UTFencname]{x01C0}{\textpipevar} % dental click
\DeclareUTFcharacter[\UTFencname]{x01C0}{\textvertline}
\DeclareUTFcharacter[\UTFencname]{x01C1}{\textdoublepipe}
\DeclareUTFcharacter[\UTFencname]{x01C1}{\textdoublepipevar}
\DeclareUTFcharacter[\UTFencname]{x01C2}{\textdoublebarpipe}
\DeclareUTFcharacter[\UTFencname]{x01C2}{\textdoublebarpipevar}
\DeclareUTFcharacter[\UTFencname]{x01C3}{\textrclick} % ??
\DeclareUTFcomposite[\UTFencname]{x01C4}{\v}{\DZ}
\DeclareUTFcomposite[\UTFencname]{x01C5}{\v}{\Dz}
\DeclareUTFcomposite[\UTFencname]{x01C6}{\v}{\dz}
\DeclareUTFcharacter[\UTFencname]{x01C7}{\LJ}
\DeclareUTFcharacter[\UTFencname]{x01C8}{\Lj}
\DeclareUTFcharacter[\UTFencname]{x01C9}{\lj}
\DeclareUTFcharacter[\UTFencname]{x01CA}{\NJ}
\DeclareUTFcharacter[\UTFencname]{x01CB}{\Nj}
\DeclareUTFcharacter[\UTFencname]{x01CC}{\nj}
\DeclareUTFcomposite[\UTFencname]{x01CD}{\v}{A}
\DeclareUTFcomposite[\UTFencname]{x01CD}{\capitalcaron}{A}
\DeclareUTFcomposite[\UTFencname]{x01CE}{\v}{a}
\DeclareUTFcomposite[\UTFencname]{x01CF}{\v}{I}
\DeclareUTFcomposite[\UTFencname]{x01CF}{\capitalcaron}{I}
\DeclareUTFcomposite[\UTFencname]{x01D0}{\v}{i}
\DeclareUTFcomposite[\UTFencname]{x01D0}{\v}{\i}
\DeclareUTFcomposite[\UTFencname]{x01D1}{\v}{O}
\DeclareUTFcomposite[\UTFencname]{x01D1}{\capitalcaron}{O}
\DeclareUTFcomposite[\UTFencname]{x01D2}{\v}{o}
\DeclareUTFcomposite[\UTFencname]{x01D3}{\v}{U}
\DeclareUTFcomposite[\UTFencname]{x01D3}{\capitalcaron}{U}
\DeclareUTFcomposite[\UTFencname]{x01D4}{\v}{u}
\DeclareUTFcomposite[\UTFencname]{x01D5}{\textdieresisoverline}{U}
\DeclareUTFcomposite[\UTFencname]{x01D5}{\=}{\"U}
\DeclareUTFcomposite[\UTFencname]{x01D6}{\textdieresisoverline}{u}
\DeclareUTFcomposite[\UTFencname]{x01D6}{\=}{\"u}
\DeclareUTFcomposite[\UTFencname]{x01D7}{\textdieresisacute}{U}
\DeclareUTFcomposite[\UTFencname]{x01D7}{\'}{\"U}
\DeclareUTFcomposite[\UTFencname]{x01D8}{\textdieresisacute}{u}
\DeclareUTFcomposite[\UTFencname]{x01D8}{\'}{\"u}
\DeclareUTFcomposite[\UTFencname]{x01D9}{\textdieresiscaron}{U}
\DeclareUTFcomposite[\UTFencname]{x01D9}{\v}{\"U}
\DeclareUTFcomposite[\UTFencname]{x01DA}{\textdieresiscaron}{u}
\DeclareUTFcomposite[\UTFencname]{x01DA}{\v}{\"u}
\DeclareUTFcomposite[\UTFencname]{x01DB}{\textdieresisgrave}{U}
\DeclareUTFcomposite[\UTFencname]{x01DB}{\`}{\"U}
\DeclareUTFcomposite[\UTFencname]{x01DC}{\textdieresisgrave}{u}
\DeclareUTFcomposite[\UTFencname]{x01DC}{\`}{\"u}
\DeclareUTFcomposite[\UTFencname]{x01DD}{\M}{e}% upside-down e, same as x0259
\DeclareUTFcharacter[\UTFencname]{x01DD}{\inve}
\DeclareUTFcharacter[\UTFencname]{x01DD}{\texteturned}
\DeclareUTFcomposite[\UTFencname]{x01DE}{\textdieresisoverline}{A}
\DeclareUTFcomposite[\UTFencname]{x01DE}{\=}{\"A}
\DeclareUTFcomposite[\UTFencname]{x01DF}{\textdieresisoverline}{a}
\DeclareUTFcomposite[\UTFencname]{x01DF}{\=}{\"a}
\DeclareUTFcomposite[\UTFencname]{x01E0}{\textdotoverline}{A}
\DeclareUTFcomposite[\UTFencname]{x01E0}{\"}{\.A}
\DeclareUTFcomposite[\UTFencname]{x01E1}{\textdotoverline}{a}
\DeclareUTFcomposite[\UTFencname]{x01E1}{\"}{\.a}
\DeclareUTFcomposite[\UTFencname]{x01E2}{\=}{\AE}
\DeclareUTFcomposite[\UTFencname]{x01E3}{\=}{\ae}
\DeclareUTFcomposite[\UTFencname]{x01E4}{\B}{\G}
\DeclareUTFcomposite[\UTFencname]{x01E5}{\B}{\g}
\DeclareUTFcharacter[\UTFencname]{x01E5}{\textcrg}
\DeclareUTFcomposite[\UTFencname]{x01E6}{\v}{G}
\DeclareUTFcomposite[\UTFencname]{x01E6}{\capitalcaron}{G}
\DeclareUTFcomposite[\UTFencname]{x01E7}{\v}{g}
\DeclareUTFcomposite[\UTFencname]{x01E8}{\v}{K}
\DeclareUTFcomposite[\UTFencname]{x01E8}{\capitalcaron}{K}
\DeclareUTFcomposite[\UTFencname]{x01E9}{\v}{k}
\DeclareUTFcomposite[\UTFencname]{x01EA}{\k}{O}
\DeclareUTFcomposite[\UTFencname]{x01EA}{\capitalogonek}{O}
\DeclareUTFcomposite[\UTFencname]{x01EB}{\k}{o}
\DeclareUTFcomposite[\UTFencname]{x01EB}{\capitalogonek}{o}
\DeclareUTFcomposite[\UTFencname]{x01EC}{\textogonekoverline}{O}
\DeclareUTFcomposite[\UTFencname]{x01EC}{\=}{\k O}
\DeclareUTFcomposite[\UTFencname]{x01ED}{\textogonekoverline}{o}
\DeclareUTFcomposite[\UTFencname]{x01ED}{\=}{\k o}
\DeclareUTFcomposite[\UTFencname]{x01EE}{\v}{\EZH}
\DeclareUTFcharacter[\UTFencname]{x01EE}{\textEzh}
\DeclareUTFcomposite[\UTFencname]{x01EF}{\v}{\ezh}
\DeclareUTFcharacter[\UTFencname]{x01EF}{\textezh}
\DeclareUTFcomposite[\UTFencname]{x01F0}{\v}{j}
\DeclareUTFcomposite[\UTFencname]{x01F0}{\v}{\j}
\DeclareUTFcharacter[\UTFencname]{x01F1}{\DZ}
\DeclareUTFcharacter[\UTFencname]{x01F2}{\Dz}
\DeclareUTFcharacter[\UTFencname]{x01F3}{\dz}

\DeclareUTFcomposite[\UTFencname]{x01F4}{\'}{G}
\DeclareUTFcomposite[\UTFencname]{x01F4}{\capitalacute}{G}
\DeclareUTFcomposite[\UTFencname]{x01F5}{\'}{g}
\DeclareUTFcharacter[\UTFencname]{x01F6}{\HV}
\DeclareUTFcharacter[\UTFencname]{x01F7}{\WYNN}
\DeclareUTFcharacter[\UTFencname]{x01F7}{\textwynn}
\DeclareUTFcomposite[\UTFencname]{x01F8}{\`}{N}
\DeclareUTFcomposite[\UTFencname]{x01F8}{\capitalgrave}{N}
\DeclareUTFcomposite[\UTFencname]{x01F9}{\`}{n}
\DeclareUTFcomposite[\UTFencname]{x01FA}{\'}{\r A}
\DeclareUTFcomposite[\UTFencname]{x01FB}{\'}{\r a}
\DeclareUTFcomposite[\UTFencname]{x01FC}{\'}{\AE}
\DeclareUTFcomposite[\UTFencname]{x01FD}{\'}{\ae}
\DeclareUTFcomposite[\UTFencname]{x01FE}{\'}{\O}
\DeclareUTFcomposite[\UTFencname]{x01FF}{\'}{\o}

\DeclareUTFcomposite[\UTFencname]{x0200}{\G}{A}
\DeclareUTFcomposite[\UTFencname]{x0201}{\G}{a}
\DeclareUTFcomposite[\UTFencname]{x0202}{\textroundcap}{A}
\DeclareUTFcomposite[\UTFencname]{x0203}{\textroundcap}{a}
\DeclareUTFcomposite[\UTFencname]{x0204}{\G}{E}
\DeclareUTFcomposite[\UTFencname]{x0205}{\G}{e}
\DeclareUTFcomposite[\UTFencname]{x0206}{\textroundcap}{E}
\DeclareUTFcomposite[\UTFencname]{x0207}{\textroundcap}{e}
\DeclareUTFcomposite[\UTFencname]{x0208}{\G}{I}
\DeclareUTFcomposite[\UTFencname]{x0209}{\G}{i}
\DeclareUTFcomposite[\UTFencname]{x0209}{\G}{\i}
\DeclareUTFcomposite[\UTFencname]{x020A}{\textroundcap}{I}
\DeclareUTFcomposite[\UTFencname]{x020B}{\textroundcap}{i}
\DeclareUTFcomposite[\UTFencname]{x020B}{\textroundcap}{\i}
\DeclareUTFcomposite[\UTFencname]{x020C}{\G}{O}
\DeclareUTFcomposite[\UTFencname]{x020D}{\G}{o}
\DeclareUTFcomposite[\UTFencname]{x020E}{\textroundcap}{O}
\DeclareUTFcomposite[\UTFencname]{x020F}{\textroundcap}{o}
\DeclareUTFcomposite[\UTFencname]{x0210}{\G}{R}
\DeclareUTFcomposite[\UTFencname]{x0211}{\G}{r}
\DeclareUTFcomposite[\UTFencname]{x0212}{\textroundcap}{R}
\DeclareUTFcomposite[\UTFencname]{x0213}{\textroundcap}{r}
\DeclareUTFcomposite[\UTFencname]{x0214}{\G}{U}
\DeclareUTFcomposite[\UTFencname]{x0215}{\G}{u}
\DeclareUTFcomposite[\UTFencname]{x0216}{\textroundcap}{U}
\DeclareUTFcomposite[\UTFencname]{x0217}{\textroundcap}{u}
\DeclareUTFcomposite[\UTFencname]{x0218}{\textcommabelow}{S}
\DeclareUTFcomposite[\UTFencname]{x0219}{\textcommabelow}{s}
\DeclareUTFcomposite[\UTFencname]{x021A}{\textcommabelow}{T}
\DeclareUTFcomposite[\UTFencname]{x021B}{\textcommabelow}{t}
\DeclareUTFcharacter[\UTFencname]{x021C}{\YOGH}
\DeclareUTFcharacter[\UTFencname]{x021D}{\yogh}
\DeclareUTFcomposite[\UTFencname]{x021E}{\v}{H}
\DeclareUTFcomposite[\UTFencname]{x021E}{\capitalcaron}{H}
\DeclareUTFcomposite[\UTFencname]{x021F}{\v}{h}
\DeclareUTFcomposite[\UTFencname]{x0220}{\M}{N}
\DeclareUTFcharacter[\UTFencname]{x0221}{\textctd}
\DeclareUTFcomposite[\UTFencname]{x0224}{\textcommabelow}{Z} % with comma
\DeclareUTFcomposite[\UTFencname]{x0225}{\textcommabelow}{z} % with comma
\DeclareUTFcharacter[\UTFencname]{x0225}{\textcommatailz}
\DeclareUTFcomposite[\UTFencname]{x0226}{\.}{A}
\DeclareUTFcomposite[\UTFencname]{x0226}{\capitaldotaccent}{A}
\DeclareUTFcomposite[\UTFencname]{x0227}{\.}{a}
\DeclareUTFcomposite[\UTFencname]{x0228}{\c}{E}
\DeclareUTFcomposite[\UTFencname]{x0228}{\capitalcedilla}{E}
\DeclareUTFcomposite[\UTFencname]{x0229}{\c}{e}
\DeclareUTFcomposite[\UTFencname]{x022A}{\=}{\"O}
\DeclareUTFcomposite[\UTFencname]{x022A}{\textdieresisoverline}{O}
\DeclareUTFcomposite[\UTFencname]{x022B}{\=}{\"o}
\DeclareUTFcomposite[\UTFencname]{x022B}{\textdieresisoverline}{o}
\DeclareUTFcomposite[\UTFencname]{x022C}{\=}{\~O}
\DeclareUTFcomposite[\UTFencname]{x022C}{\texttildeoverline}{O}
\DeclareUTFcomposite[\UTFencname]{x022D}{\=}{\~o}
\DeclareUTFcomposite[\UTFencname]{x022D}{\texttildeoverline}{o}
\DeclareUTFcomposite[\UTFencname]{x022E}{\.}{O}
\DeclareUTFcomposite[\UTFencname]{x022E}{\capitaldotaccent}{O}
\DeclareUTFcomposite[\UTFencname]{x022F}{\.}{o}
\DeclareUTFcomposite[\UTFencname]{x0230}{\=}{\.O}
\DeclareUTFcomposite[\UTFencname]{x0230}{\textdotoverline}{O}
\DeclareUTFcomposite[\UTFencname]{x0231}{\=}{\.o}
\DeclareUTFcomposite[\UTFencname]{x0231}{\textdotoverline}{o}
\DeclareUTFcomposite[\UTFencname]{x0232}{\=}{Y}
\DeclareUTFcomposite[\UTFencname]{x0232}{\capitalmacron}{Y}
\DeclareUTFcomposite[\UTFencname]{x0233}{\=}{y}
\DeclareUTFcharacter[\UTFencname]{x0234}{\textctl} % why not ?
\DeclareUTFcharacter[\UTFencname]{x0235}{\textctn}
\DeclareUTFcharacter[\UTFencname]{x0236}{\textctt}
\DeclareUTFcharacter[\UTFencname]{x0237}{\textdotlessj}
\DeclareUTFcharacter[\UTFencname]{x0238}{\textdblig} %
\DeclareUTFcharacter[\UTFencname]{x0239}{\textqplig} %
\DeclareUTFcharacter[\UTFencname]{x023A}{\textstrokea} %
\DeclareUTFcharacter[\UTFencname]{x023B}{\textstrokecapitalc} %
\DeclareUTFcharacter[\UTFencname]{x023C}{\textstrokec} %
\DeclareUTFcharacter[\UTFencname]{x023D}{\textbarcapitall} %
\DeclareUTFcharacter[\UTFencname]{x023E}{\textstrokecapitalt} %
\DeclareUTFcharacter[\UTFencname]{x023F}{\textrts} %
\DeclareUTFcharacter[\UTFencname]{x0240}{\textrtz} %
\DeclareUTFcharacter[\UTFencname]{x0241}{\textglotstopvari} %
\DeclareUTFcharacter[\UTFencname]{x0242}{\textglotstopvarii} %
\DeclareUTFcharacter[\UTFencname]{x0243}{\textbarcapitalb} %
\DeclareUTFcharacter[\UTFencname]{x0244}{\textbarcapitalu} %
\DeclareUTFcharacter[\UTFencname]{x0245}{\textturnedcapitalv} %
\DeclareUTFcharacter[\UTFencname]{x0246}{\textstrokecapitale} %
\DeclareUTFcharacter[\UTFencname]{x0247}{\textstrokee} %
\DeclareUTFcharacter[\UTFencname]{x0248}{\textbarcapitalj} %
\DeclareUTFcharacter[\UTFencname]{x0249}{\textbarj} %
\DeclareUTFcharacter[\UTFencname]{x024A}{\texthtcapitalq} %
\DeclareUTFcharacter[\UTFencname]{x024B}{\texthtq} %
\DeclareUTFcharacter[\UTFencname]{x024C}{\textbarcapitalr} %
\DeclareUTFcharacter[\UTFencname]{x024D}{\textbarr} %
\DeclareUTFcharacter[\UTFencname]{x024E}{\textbarcapitaly} %
\DeclareUTFcharacter[\UTFencname]{x024F}{\textbary} %
%
% IPA extensions
%
\DeclareUTFcharacter[\UTFencname]{x0250}{\textturna}        % TIPA-5
\DeclareUTFcharacter[\UTFencname]{x0251}{\textscripta}      % TIPA-A
\DeclareUTFcharacter[\UTFencname]{x0252}{\textturnscripta}  % TIPA-6
\DeclareUTFcomposite[\UTFencname]{x0253}{\m}{b}
\DeclareUTFcomposite[\UTFencname]{x0253}{\m}{b}
\DeclareUTFcharacter[\UTFencname]{x0253}{\texthtb}
\DeclareUTFcharacter[\UTFencname]{x0253}{\textbhook}
\DeclareUTFcomposite[\UTFencname]{x0254}{\m}{o}
\DeclareUTFcharacter[\UTFencname]{x0254}{\textopeno}   %  TIPA-O
\DeclareUTFcharacter[\UTFencname]{x0254}{\textoopen}
\DeclareUTFcharacter[\UTFencname]{x0255}{\textctc}     % TIPA-C
\DeclareUTFcomposite[\UTFencname]{x0256}{\M}{d}
\DeclareUTFcharacter[\UTFencname]{x0256}{\textrtaild}
\DeclareUTFcharacter[\UTFencname]{x0256}{\textdtail}
\DeclareUTFcomposite[\UTFencname]{x0257}{\m}{d}
\DeclareUTFcharacter[\UTFencname]{x0257}{\texthtd}
\DeclareUTFcharacter[\UTFencname]{x0257}{\textdhook}
\DeclareUTFcharacter[\UTFencname]{x0258}{\textreve}    % TIPA-9
\DeclareUTFcharacter[\UTFencname]{x0259}{\schwa}
\DeclareUTFcharacter[\UTFencname]{x0259}{\textschwa}   %  TIPA-@
\DeclareUTFcomposite[\UTFencname]{x025A}{\m}{\schwa} % with hook above
\DeclareUTFcomposite[\UTFencname]{x025A}{\texthookabove}{\schwa} % with hook above
\DeclareUTFcharacter[\UTFencname]{x025A}{\textrhookschwa}
\DeclareUTFcomposite[\UTFencname]{x025B}{\m}{e}
\DeclareUTFcharacter[\UTFencname]{x025B}{\textepsilon} % ??   % TIPA-E
\DeclareUTFcharacter[\UTFencname]{x025B}{\texteopen}
\DeclareUTFcharacter[\UTFencname]{x025C}{\textrevepsilon}     % TIPA-3
\DeclareUTFcomposite[\UTFencname]{x025D}{\texthookabove}{\textrevepsilon} % with hook above
\DeclareUTFcharacter[\UTFencname]{x025D}{\textrhookrevepsilon}
\DeclareUTFcharacter[\UTFencname]{x025E}{\textcloserevepsilon}
\DeclareUTFcomposite[\UTFencname]{x025F}{\B}{j}
\DeclareUTFcharacter[\UTFencname]{x025F}{\textbardotlessj}
\DeclareUTFcharacter[\UTFencname]{x025F}{\textObardotlessj}
\DeclareUTFcomposite[\UTFencname]{x0260}{\texthookabove}{g} % with hook above
\DeclareUTFcharacter[\UTFencname]{x0260}{\texthtg}
\DeclareUTFcharacter[\UTFencname]{x0261}{\textscriptg}
\let\textg\textscriptg  %  added for v0.7  RRM  2006/04/09
\DeclareUTFcharacter[\UTFencname]{x0262}{\textscg}
\DeclareUTFcomposite[\UTFencname]{x0263}{\m}{g}
\DeclareUTFcharacter[\UTFencname]{x0263}{\textbabygamma}
\DeclareUTFcharacter[\UTFencname]{x0263}{\textgammalatinsmall}
\DeclareUTFcharacter[\UTFencname]{x0264}{\textramshorns} % ram's horn %  TIPA-7
\DeclareUTFcharacter[\UTFencname]{x0265}{\textturnh}    % TIPA-4
\DeclareUTFcomposite[\UTFencname]{x0266}{\m}{h} % with hook above
\DeclareUTFcomposite[\UTFencname]{x0266}{\texthookabove}{h} % with hook above
\DeclareUTFcharacter[\UTFencname]{x0266}{\texthth}  % TIPA-H
\DeclareUTFcomposite[\UTFencname]{x0267}{\texthookabove}{\textheng} % with hook above
\DeclareUTFcharacter[\UTFencname]{x0267}{\texththeng}
\DeclareUTFcomposite[\UTFencname]{x0268}{\B}{i}
\DeclareUTFcharacter[\UTFencname]{x0268}{\textbari}    % TIPA-1
\DeclareUTFcomposite[\UTFencname]{x0269}{\m}{i}
\DeclareUTFcharacter[\UTFencname]{x0269}{\textiota} % ?? see x0196
\DeclareUTFcharacter[\UTFencname]{x0269}{\textiotalatin} 
\DeclareUTFcharacter[\UTFencname]{x026A}{\textsci}     % TIPA-I
\DeclareUTFcharacter[\UTFencname]{x026B}{\textltilde}
\DeclareUTFcharacter[\UTFencname]{x026C}{\textbeltl}
\DeclareUTFcomposite[\UTFencname]{x026D}{\textrethookbelow}{l} % with retroflex hook
\DeclareUTFcharacter[\UTFencname]{x026D}{\textrtaill}
\DeclareUTFcharacter[\UTFencname]{x026E}{\textlyoghlig}
\DeclareUTFcharacter[\UTFencname]{x026E}{\textOlyoghlig}
\DeclareUTFcharacter[\UTFencname]{x026F}{\textturnm}   %  TIPA-W
\DeclareUTFcharacter[\UTFencname]{x0270}{\textturnmrleg}
\DeclareUTFcomposite[\UTFencname]{x0271}{\m}{m}
\DeclareUTFcharacter[\UTFencname]{x0271}{\textltailm}  %  TIPA-M
\DeclareUTFcomposite[\UTFencname]{x0272}{\m}{j}
\DeclareUTFcharacter[\UTFencname]{x0272}{\textltailn}
\DeclareUTFcharacter[\UTFencname]{x0272}{\textnhookleft}
\DeclareUTFcomposite[\UTFencname]{x0273}{\m}{n}
\DeclareUTFcharacter[\UTFencname]{x0273}{\textrtailn}
\DeclareUTFcharacter[\UTFencname]{x0274}{\textscn}
\DeclareUTFcharacter[\UTFencname]{x0275}{\textbaro}    % TIPA-8
\DeclareUTFcharacter[\UTFencname]{x0276}{\textscoelig}
\DeclareUTFcharacter[\UTFencname]{x0277}{\textcloseomega}
\DeclareUTFcharacter[\UTFencname]{x0278}{\textphi}   %  TIPA-F
\DeclareUTFcharacter[\UTFencname]{x0279}{\textturnr}
\DeclareUTFcharacter[\UTFencname]{x027A}{\textturnlonglegr}
\DeclareUTFcomposite[\UTFencname]{x027B}{\textrethookbelow}{r} % turned with ret hook
\DeclareUTFcharacter[\UTFencname]{x027B}{\textturnrrtail}
\DeclareUTFcharacter[\UTFencname]{x027C}{\textlonglegr}
\DeclareUTFcomposite[\UTFencname]{x027D}{\textrethookbelow}{r} % with ret hook
\DeclareUTFcharacter[\UTFencname]{x027D}{\textrtailr}
\DeclareUTFcharacter[\UTFencname]{x027E}{\textfishhookr}  %  TIPA-R
\DeclareUTFcharacter[\UTFencname]{x027F}{\textlhti} %
\DeclareUTFcharacter[\UTFencname]{x027F}{\textlhtlongi} %
\DeclareUTFcharacter[\UTFencname]{x0280}{\textscr}
\DeclareUTFcharacter[\UTFencname]{x0281}{\textinvscr}   %  TIPA-K
\DeclareUTFcomposite[\UTFencname]{x0282}{\textrethookbelow}{s} % with hook
\DeclareUTFcharacter[\UTFencname]{x0282}{\textrtails}
\DeclareUTFcomposite[\UTFencname]{x0283}{\m}{s}
\DeclareUTFcharacter[\UTFencname]{x0283}{\esh}
\DeclareUTFcharacter[\UTFencname]{x0283}{\textesh}      %  TIPA-S
\DeclareUTFcharacter[\UTFencname]{x0284}{\texthtbardotlessj}
\DeclareUTFcharacter[\UTFencname]{x0284}{\texthtObardotlessj}
\DeclareUTFcharacter[\UTFencname]{x0284}{\texthtbardotlessjvar}
\DeclareUTFcomposite[\UTFencname]{x0285}{\m}{S} % squat reversed esh
\DeclareUTFcharacter[\UTFencname]{x0285}{\textvibyi}
\DeclareUTFcharacter[\UTFencname]{x0286}{\textctesh}
\DeclareUTFcharacter[\UTFencname]{x0287}{\textturnt}
\DeclareUTFcomposite[\UTFencname]{x0288}{\M}{t}
\DeclareUTFcharacter[\UTFencname]{x0288}{\textrtailt}
\DeclareUTFcharacter[\UTFencname]{x0288}{\texttretroflexhook}
\DeclareUTFcomposite[\UTFencname]{x0289}{\B}{u}
\DeclareUTFcharacter[\UTFencname]{x0289}{\textbaru}        % TIPA-0
\DeclareUTFcharacter[\UTFencname]{x028A}{\textscupsilon}  % TIPA-U
\DeclareUTFcomposite[\UTFencname]{x028B}{\m}{u}
\DeclareUTFcomposite[\UTFencname]{x028B}{\m}{v}
\DeclareUTFcharacter[\UTFencname]{x028B}{\textscriptv}  % TIPA-V
\DeclareUTFcharacter[\UTFencname]{x028B}{\textvhook}
\DeclareUTFcharacter[\UTFencname]{x028C}{\textturnv}    % TIPA-2
\DeclareUTFcharacter[\UTFencname]{x028D}{\textturnw}
\DeclareUTFcharacter[\UTFencname]{x028E}{\textturny}    % TIPA-L
\DeclareUTFcharacter[\UTFencname]{x028F}{\textscy}      % TIPA-Y
\DeclareUTFcomposite[\UTFencname]{x0290}{\textrethookbelow}{z} % with retroflex hook
\DeclareUTFcharacter[\UTFencname]{x0290}{\textrtailz}
\DeclareUTFcharacter[\UTFencname]{x0291}{\textctz}
\DeclareUTFcomposite[\UTFencname]{x0292}{\m}{z}
\DeclareUTFcharacter[\UTFencname]{x0292}{\ezh}
\DeclareUTFcharacter[\UTFencname]{x0292}{\textezh}
\DeclareUTFcharacter[\UTFencname]{x0292}{\textyogh} % TIPA-Z !!!  see Ux021D
\DeclareUTFcharacter[\UTFencname]{x0293}{\textctyogh}
\DeclareUTFcharacter[\UTFencname]{x0294}{\textglotstop} %  TIPA-P
\DeclareUTFcharacter[\UTFencname]{x0295}{\textrevglotstop} %  TIPA-Q
\DeclareUTFcharacter[\UTFencname]{x0296}{\textinvglotstop}
\DeclareUTFcharacter[\UTFencname]{x0297}{\textstretchc}
\DeclareUTFcharacter[\UTFencname]{x0297}{\textstretchcvar}
\DeclareUTFcharacter[\UTFencname]{x0298}{\textbullseye}
\DeclareUTFcharacter[\UTFencname]{x0298}{\textObullseye}
\DeclareUTFcharacter[\UTFencname]{x0299}{\textscb}
\DeclareUTFcharacter[\UTFencname]{x029A}{\textcloseepsilon}
\DeclareUTFcharacter[\UTFencname]{x029B}{\texthtscg}
\DeclareUTFcharacter[\UTFencname]{x029C}{\textsch}
\DeclareUTFcharacter[\UTFencname]{x029D}{\textctj}   %  TIPA-J
\DeclareUTFcharacter[\UTFencname]{x029D}{\textctjvar}
\DeclareUTFcharacter[\UTFencname]{x029E}{\textturnk}
\DeclareUTFcharacter[\UTFencname]{x029F}{\textscl}
\DeclareUTFcomposite[\UTFencname]{x02A0}{\m}{q}
\DeclareUTFcharacter[\UTFencname]{x02A0}{\texthtq}
\DeclareUTFcharacter[\UTFencname]{x02A1}{\textbarglotstop}
\DeclareUTFcharacter[\UTFencname]{x02A2}{\textbarrevglotstop}
\DeclareUTFcharacter[\UTFencname]{x02A3}{\textdzlig}
\DeclareUTFcharacter[\UTFencname]{x02A4}{\textdyoghlig}
\DeclareUTFcharacter[\UTFencname]{x02A5}{\textdctzlig}
\DeclareUTFcharacter[\UTFencname]{x02A6}{\texttslig}
\DeclareUTFcharacter[\UTFencname]{x02A7}{\textteshlig}
\DeclareUTFcharacter[\UTFencname]{x02A7}{\texttesh}
\DeclareUTFcharacter[\UTFencname]{x02A8}{\texttctclig}
\DeclareUTFcharacter[\UTFencname]{x02A9}{\textfenglig}
\DeclareUTFcharacter[\UTFencname]{x02AA}{\textlslig}
\DeclareUTFcharacter[\UTFencname]{x02AB}{\textlzlig}
\DeclareUTFcharacter[\UTFencname]{x02AE}{\textlhtlongy}
\DeclareUTFcharacter[\UTFencname]{x02AF}{\textvibyy}

% raised letters, etc.
\DeclareUTFcharacter[\UTFencname]{x02B0}{\textsuph}
\DeclareUTFcharacter[\UTFencname]{x02B1}{\textsuphth}
\DeclareUTFcharacter[\UTFencname]{x02B2}{\textsupj}
\DeclareUTFcharacter[\UTFencname]{x02B3}{\textsupr}
\DeclareUTFcharacter[\UTFencname]{x02B4}{\textsupturnr}
\DeclareUTFcharacter[\UTFencname]{x02B5}{\textsupturnrrtail}
\DeclareUTFcharacter[\UTFencname]{x02B6}{\textsupinvscr}
\DeclareUTFcharacter[\UTFencname]{x02B7}{\textsupw}
\DeclareUTFcharacter[\UTFencname]{x02B8}{\textsupy}
\DeclareUTFcharacter[\UTFencname]{x02B9}{\cprime}
\DeclareUTFcharacter[\UTFencname]{x02B9}{\textceltpal} % ??
\DeclareUTFcharacter[\UTFencname]{x02BA}{\cdprime}
\DeclareUTFcharacter[\UTFencname]{x02BB}{\textturncomma}
\DeclareUTFcharacter[\UTFencname]{x02BC}{\rasp}
\DeclareUTFcharacter[\UTFencname]{x02BD}{\lasp}
\DeclareUTFcharacter[\UTFencname]{x02BD}{\textrevapostrophe}
\DeclareUTFcharacter[\UTFencname]{x02BE}{\texthamza}
\DeclareUTFcharacter[\UTFencname]{x02BF}{\textain}
\DeclareUTFcharacter[\UTFencname]{x02C0}{\textraiseglotstop}
\DeclareUTFcharacter[\UTFencname]{x02C1}{\textraiserevglotstop}
\DeclareUTFcharacter[\UTFencname]{x02C2}{\textlptr}
\DeclareUTFcharacter[\UTFencname]{x02C3}{\textrptr}
\DeclareUTFcharacter[\UTFencname]{x02C4}{\textuptr}
\DeclareUTFcharacter[\UTFencname]{x02C5}{\textdptr}
\DeclareUTFcharacter[\UTFencname]{x02C6}{\textcircumaccent} % see also  x005E
\DeclareUTFcharacter[\UTFencname]{x02C7}{\textasciicaron}
\DeclareUTFcharacter[\UTFencname]{x02C7}{\textcaronaccent}  % see also   x00 ??
\DeclareUTFcharacter[\UTFencname]{x02C8}{\textprimstress} % ?? see x02B9   %  TIPA-"
\DeclareUTFcharacter[\UTFencname]{x02C9}{\textmacronaccent}  % see also   x00AF
\DeclareUTFcharacter[\UTFencname]{x02C9}{\textmacron}  % see also   x00AF
\DeclareUTFcharacter[\UTFencname]{x02CA}{\textacuteaccent}  % see also   x00B4
\DeclareUTFcharacter[\UTFencname]{x02CA}{\textacute}  % see also   x00B4
\DeclareUTFcharacter[\UTFencname]{x02CB}{\textgraveaccent}  % see also   x0060
\DeclareUTFcharacter[\UTFencname]{x02CB}{\textgrave}  % see also   x0060
\DeclareUTFcharacter[\UTFencname]{x02CC}{\textsecstress} % ??
\DeclareUTFcharacter[\UTFencname]{x02CD}{\textlowmacron}
\DeclareUTFcharacter[\UTFencname]{x02CE}{\textlowgrave}
\DeclareUTFcharacter[\UTFencname]{x02CF}{\textlowacute}

\DeclareUTFcharacter[\UTFencname]{x02D0}{\textlengthmark} % triangular colon  %  TIPA-:
\DeclareUTFcharacter[\UTFencname]{x02D1}{\texthalflength} % half-triang colon %  TIPA-;
\DeclareUTFcharacter[\UTFencname]{x02D2}{\textrhalfring}
\DeclareUTFcharacter[\UTFencname]{x02D3}{\textlhalfring}
\DeclareUTFcharacter[\UTFencname]{x02D4}{\textraised}
\DeclareUTFcharacter[\UTFencname]{x02D5}{\textlowered}
\DeclareUTFcharacter[\UTFencname]{x02D6}{\textadvanced}
\DeclareUTFcharacter[\UTFencname]{x02D7}{\textretracted}
\DeclareUTFcharacter[\UTFencname]{x02D8}{\textbreveaccent}  % see also   x00 ??
\DeclareUTFcharacter[\UTFencname]{x02D8}{\textbreve}  % see also   x00 ??
\DeclareUTFcharacter[\UTFencname]{x02D8}{\textasciibreve}  % see also   x00 ??
\DeclareUTFcharacter[\UTFencname]{x02D9}{\textdotabove}
\DeclareUTFcharacter[\UTFencname]{x02D9}{\textdotaccent}
\DeclareUTFcharacter[\UTFencname]{x02DA}{\textringabove}
\DeclareUTFcharacter[\UTFencname]{x02DA}{\textringaccent}
\DeclareUTFcharacter[\UTFencname]{x02DB}{\textcedillaaccent}  % see also   x00B8
\DeclareUTFcharacter[\UTFencname]{x02DB}{\textogonek}
\DeclareUTFcharacter[\UTFencname]{x02DC}{\texttildeaccent}      % see also   x007E
\DeclareUTFcharacter[\UTFencname]{x02DC}{\textsmalltilde}
\DeclareUTFcharacter[\UTFencname]{x02DD}{\textacutedbl}
\DeclareUTFcharacter[\UTFencname]{x02DD}{\textdoubleacute}
\DeclareUTFcharacter[\UTFencname]{x02DE}{\textrhoticity}

%: separator

\DeclareUTFcharacter[\UTFencname]{x0E3F}{\textbaht}
\DeclareUTFcharacter[\UTFencname]{x1D00}{\textsca}
\DeclareUTFcharacter[\UTFencname]{x1D07}{\textsce}
\DeclareUTFcharacter[\UTFencname]{x1D0A}{\textscj}
\DeclareUTFcharacter[\UTFencname]{x1D0B}{\textsck}
\DeclareUTFcharacter[\UTFencname]{x1D0D}{\textscm}
\DeclareUTFcharacter[\UTFencname]{x1D18}{\textscp}
\DeclareUTFcharacter[\UTFencname]{x1D19}{\textrevscr}
\DeclareUTFcharacter[\UTFencname]{x1D1B}{\textsct}
\DeclareUTFcharacter[\UTFencname]{x1D1C}{\textscu}
\DeclareUTFcharacter[\UTFencname]{x1D20}{\textscv}
\DeclareUTFcharacter[\UTFencname]{x1D21}{\textscw}
\DeclareUTFcharacter[\UTFencname]{x1D22}{\textscz}
\DeclareUTFcharacter[\UTFencname]{x1D23}{\textscezh}
\DeclareUTFcharacter[\UTFencname]{x1D25}{\textain}
\DeclareUTFcharacter[\UTFencname]{x1D26}{\textscgamma}
\DeclareUTFcharacter[\UTFencname]{x1D27}{\textsclambda}
\DeclareUTFcharacter[\UTFencname]{x1D28}{\textscpi}
\DeclareUTFcharacter[\UTFencname]{x1D29}{\textscrho}
\DeclareUTFcharacter[\UTFencname]{x1D2A}{\textscpsi}
\DeclareUTFcharacter[\UTFencname]{x1D2B}{\textscel}
\DeclareUTFcomposite[\UTFencname]{x1D2C}{\textsuperscript}{A}
\DeclareUTFcomposite[\UTFencname]{x1D2D}{\textsuperscript}{\AE}
\DeclareUTFcomposite[\UTFencname]{x1D2E}{\textsuperscript}{B}
\DeclareUTFcomposite[\UTFencname]{x1D30}{\textsuperscript}{D}
\DeclareUTFcomposite[\UTFencname]{x1D31}{\textsuperscript}{E}
\DeclareUTFcomposite[\UTFencname]{x1D32}{\textsuperscript}{\textreve}
\DeclareUTFcomposite[\UTFencname]{x1D33}{\textsuperscript}{G}
\DeclareUTFcomposite[\UTFencname]{x1D34}{\textsuperscript}{H}
\DeclareUTFcomposite[\UTFencname]{x1D35}{\textsuperscript}{I}
\DeclareUTFcomposite[\UTFencname]{x1D36}{\textsuperscript}{J}
\DeclareUTFcomposite[\UTFencname]{x1D37}{\textsuperscript}{K}
\DeclareUTFcomposite[\UTFencname]{x1D38}{\textsuperscript}{L}
\DeclareUTFcomposite[\UTFencname]{x1D39}{\textsuperscript}{M}
\DeclareUTFcomposite[\UTFencname]{x1D3A}{\textsuperscript}{N}
\DeclareUTFcomposite[\UTFencname]{x1D3C}{\textsuperscript}{O}
\DeclareUTFcomposite[\UTFencname]{x1D3D}{\textsuperscript}{\textou}
\DeclareUTFcomposite[\UTFencname]{x1D3E}{\textsuperscript}{P}
\DeclareUTFcomposite[\UTFencname]{x1D3F}{\textsuperscript}{R}
\DeclareUTFcomposite[\UTFencname]{x1D40}{\textsuperscript}{T}
\DeclareUTFcomposite[\UTFencname]{x1D41}{\textsuperscript}{U}
\DeclareUTFcomposite[\UTFencname]{x1D42}{\textsuperscript}{W}
\DeclareUTFcomposite[\UTFencname]{x1D43}{\textsuperscript}{a}
\DeclareUTFcomposite[\UTFencname]{x1D44}{\textsuperscript}{\textturna}
\DeclareUTFcomposite[\UTFencname]{x1D45}{\textsuperscript}{\textalpha}
\DeclareUTFcomposite[\UTFencname]{x1D47}{\textsuperscript}{b}
\DeclareUTFcomposite[\UTFencname]{x1D48}{\textsuperscript}{d}
\DeclareUTFcomposite[\UTFencname]{x1D49}{\textsuperscript}{e}
\DeclareUTFcomposite[\UTFencname]{x1D4A}{\textsuperscript}{\textreve}
\DeclareUTFcomposite[\UTFencname]{x1D4B}{\textsuperscript}{\textepsilon}
\DeclareUTFcomposite[\UTFencname]{x1D4B}{\textsuperscript}{\epsilon}
\DeclareUTFcomposite[\UTFencname]{x1D4D}{\textsuperscript}{g}
\DeclareUTFcomposite[\UTFencname]{x1D4E}{\textsuperscript}{!}
\DeclareUTFcomposite[\UTFencname]{x1D4F}{\textsuperscript}{k}
\DeclareUTFcomposite[\UTFencname]{x1D50}{\textsuperscript}{m}
\DeclareUTFcomposite[\UTFencname]{x1D51}{\textsuperscript}{\ng}
\DeclareUTFcomposite[\UTFencname]{x1D52}{\textsuperscript}{o}
\DeclareUTFcomposite[\UTFencname]{x1D53}{\textsuperscript}{\textopeno}
\DeclareUTFcomposite[\UTFencname]{x1D56}{\textsuperscript}{p}
\DeclareUTFcomposite[\UTFencname]{x1D57}{\textsuperscript}{t}
\DeclareUTFcomposite[\UTFencname]{x1D58}{\textsuperscript}{u}
\DeclareUTFcomposite[\UTFencname]{x1D5A}{\textsuperscript}{\textturnm}
\DeclareUTFcomposite[\UTFencname]{x1D5B}{\textsuperscript}{v}
\DeclareUTFcomposite[\UTFencname]{x1D5C}{\textsuperscript}{\textain}
\DeclareUTFcomposite[\UTFencname]{x1D5D}{\textsuperscript}{\textbeta}
\DeclareUTFcomposite[\UTFencname]{x1D5D}{\textsuperscript}{\beta}
\DeclareUTFcomposite[\UTFencname]{x1D5E}{\textsuperscript}{\textgamma}
\DeclareUTFcomposite[\UTFencname]{x1D5E}{\textsuperscript}{\gamma}
\DeclareUTFcomposite[\UTFencname]{x1D5F}{\textsuperscript}{\delta}
\DeclareUTFcomposite[\UTFencname]{x1D60}{\textsuperscript}{\phi}
\DeclareUTFcomposite[\UTFencname]{x1D61}{\textsuperscript}{\textchi}
\DeclareUTFcomposite[\UTFencname]{x1D61}{\textsuperscript}{\chi}
\DeclareUTFcomposite[\UTFencname]{x1D62}{\textsubscript}{i}
\DeclareUTFcomposite[\UTFencname]{x1D63}{\textsubscript}{r}
\DeclareUTFcomposite[\UTFencname]{x1D64}{\textsubscript}{u}
\DeclareUTFcomposite[\UTFencname]{x1D65}{\textsubscript}{v}
\DeclareUTFcomposite[\UTFencname]{x1D66}{\textsubscript}{\textbeta}
\DeclareUTFcomposite[\UTFencname]{x1D66}{\textsubscript}{\beta}
\DeclareUTFcomposite[\UTFencname]{x1D67}{\textsubscript}{\textgamma}
\DeclareUTFcomposite[\UTFencname]{x1D67}{\textsubscript}{\gamma}
\DeclareUTFcomposite[\UTFencname]{x1D68}{\textsubscript}{\rho}
\DeclareUTFcomposite[\UTFencname]{x1D69}{\textsubscript}{\textvarphi}
\DeclareUTFcomposite[\UTFencname]{x1D69}{\textsubscript}{\phi}
\DeclareUTFcomposite[\UTFencname]{x1D6A}{\textsubscript}{\textchi}
\DeclareUTFcomposite[\UTFencname]{x1D6A}{\textsubscript}{\chi}
\DeclareUTFcomposite[\UTFencname]{x2C7C}{\textsubscript}{j}

% Phonetic Extensions Supplement

\DeclareUTFcharacter[\UTFencname]{x1D80}{\textlhookb}
\DeclareUTFcomposite[\UTFencname]{x1D80}{\textpalhookbelow}{b}
\DeclareUTFcharacter[\UTFencname]{x1D81}{\textlhookd}
\DeclareUTFcomposite[\UTFencname]{x1D81}{\textpalhookbelow}{d}
\DeclareUTFcharacter[\UTFencname]{x1D82}{\textlhookf}
\DeclareUTFcomposite[\UTFencname]{x1D82}{\textpalhookbelow}{f}
\DeclareUTFcharacter[\UTFencname]{x1D83}{\textlhookg}
\DeclareUTFcomposite[\UTFencname]{x1D83}{\textpalhookbelow}{g}
\DeclareUTFcharacter[\UTFencname]{x1D84}{\textlhookk}
\DeclareUTFcomposite[\UTFencname]{x1D84}{\textpalhookbelow}{k}
\DeclareUTFcharacter[\UTFencname]{x1D85}{\textlhookl}
\DeclareUTFcomposite[\UTFencname]{x1D85}{\textpalhookbelow}{l}
\DeclareUTFcharacter[\UTFencname]{x1D86}{\textlhookm}
\DeclareUTFcomposite[\UTFencname]{x1D86}{\textpalhookbelow}{m}
\DeclareUTFcharacter[\UTFencname]{x1D87}{\textlhookn}
\DeclareUTFcomposite[\UTFencname]{x1D87}{\textpalhookbelow}{n}
\DeclareUTFcharacter[\UTFencname]{x1D88}{\textlhookp}
\DeclareUTFcomposite[\UTFencname]{x1D88}{\textpalhookbelow}{p}
\DeclareUTFcharacter[\UTFencname]{x1D89}{\textlhookr}
\DeclareUTFcomposite[\UTFencname]{x1D89}{\textpalhookbelow}{r}
\DeclareUTFcharacter[\UTFencname]{x1D8A}{\textlhooks}
\DeclareUTFcomposite[\UTFencname]{x1D8A}{\textpalhookbelow}{s}
\DeclareUTFcharacter[\UTFencname]{x1D8B}{\textlhookesh}
\DeclareUTFcomposite[\UTFencname]{x1D8B}{\textpalhookbelow}{\textesh}
\DeclareUTFcharacter[\UTFencname]{x1D8C}{\textlhookv}
\DeclareUTFcomposite[\UTFencname]{x1D8C}{\textpalhookbelow}{v}
\DeclareUTFcharacter[\UTFencname]{x1D8D}{\textlhookx}
\DeclareUTFcomposite[\UTFencname]{x1D8D}{\textpalhookbelow}{x}
\DeclareUTFcharacter[\UTFencname]{x1D8E}{\textlhookz}
\DeclareUTFcomposite[\UTFencname]{x1D8E}{\textpalhookbelow}{z}
\DeclareUTFcharacter[\UTFencname]{x1D8F}{\textrhooka}
\DeclareUTFcomposite[\UTFencname]{x1D8F}{\textrethookbelow}{a}
\DeclareUTFcharacter[\UTFencname]{x1D90}{\textrhookalpha}%
\DeclareUTFcomposite[\UTFencname]{x1D90}{\textrethookbelow}{\textalpha}
\DeclareUTFcharacter[\UTFencname]{x1D91}{\texthtrtaild} % Hooktop right-tail D
\DeclareUTFcomposite[\UTFencname]{x1D91}{\textrethookbelow}{\texthtd}
\DeclareUTFcharacter[\UTFencname]{x1D92}{\textrhooke} %
\DeclareUTFcomposite[\UTFencname]{x1D92}{\textrethookbelow}{e}
\DeclareUTFcharacter[\UTFencname]{x1D93}{\textrhookepsilon} %
\DeclareUTFcomposite[\UTFencname]{x1D93}{\textrethookbelow}{\textepsilon}
\DeclareUTFcomposite[\UTFencname]{x1D94}{\textrethookbelow}{\textrevepsilon}
\DeclareUTFcharacter[\UTFencname]{x1D95}{\textrhookturne} %
\DeclareUTFcomposite[\UTFencname]{x1D95}{\textrethookbelow}{\textreve}
\DeclareUTFcharacter[\UTFencname]{x1D96}{\textrhooki} %
\DeclareUTFcomposite[\UTFencname]{x1D96}{\textrethookbelow}{i}
\DeclareUTFcharacter[\UTFencname]{x1D97}{\textrhookopeno} %
\DeclareUTFcomposite[\UTFencname]{x1D97}{\textrethookbelow}{\textopeno}
\DeclareUTFcharacter[\UTFencname]{x1D98}{\textrhookesh} %
\DeclareUTFcomposite[\UTFencname]{x1D98}{\textrethookbelow}{\textesh}
\DeclareUTFcharacter[\UTFencname]{x1D99}{\textrhooku} %
\DeclareUTFcomposite[\UTFencname]{x1D99}{\textrethookbelow}{u}
\DeclareUTFcharacter[\UTFencname]{x1D9A}{\textrhookezh} %
\DeclareUTFcomposite[\UTFencname]{x1D9A}{\textrethookbelow}{\textezh}
\DeclareUTFcomposite[\UTFencname]{x1D9A}{\textrethookbelow}{\textyogh}
\DeclareUTFcomposite[\UTFencname]{x1D9B}{\textsuperscript}{\textturnalpha}% ??
\DeclareUTFcomposite[\UTFencname]{x1D9C}{\textsuperscript}{c}
\DeclareUTFcomposite[\UTFencname]{x1D9D}{\textsuperscript}{\textctc}
\DeclareUTFcomposite[\UTFencname]{x1D9E}{\textsuperscript}{\dh}
\DeclareUTFcomposite[\UTFencname]{x1D9F}{\textsuperscript}{\textrevepsilon}
\DeclareUTFcomposite[\UTFencname]{x1DA0}{\textsuperscript}{f}
\DeclareUTFcomposite[\UTFencname]{x1DA1}{\textsuperscript}{\textbardotlessj}
\DeclareUTFcomposite[\UTFencname]{x1DA2}{\textsuperscript}{g}
\DeclareUTFcomposite[\UTFencname]{x1DA3}{\textsuperscript}{\textturnh}
\DeclareUTFcomposite[\UTFencname]{x1DA4}{\textsuperscript}{\textbari}
\DeclareUTFcomposite[\UTFencname]{x1DA5}{\textsuperscript}{\textiota}%
\DeclareUTFcomposite[\UTFencname]{x1DA6}{\textsuperscript}{\textsci}
\DeclareUTFcomposite[\UTFencname]{x1DA8}{\textsuperscript}{\textctj}
\DeclareUTFcomposite[\UTFencname]{x1DA9}{\textsuperscript}{\textrtaill}
\DeclareUTFcomposite[\UTFencname]{x1DAA}{\textsuperscript}{\textlhookl}
\DeclareUTFcomposite[\UTFencname]{x1DAB}{\textsuperscript}{\textscl}
\DeclareUTFcomposite[\UTFencname]{x1DAC}{\textsuperscript}{\textltailm}
\DeclareUTFcomposite[\UTFencname]{x1DAD}{\textsuperscript}{\textturnmrleg}
\DeclareUTFcomposite[\UTFencname]{x1DAE}{\textsuperscript}{\textltailn}
\DeclareUTFcomposite[\UTFencname]{x1DAF}{\textsuperscript}{\textrtailn}
\DeclareUTFcomposite[\UTFencname]{x1DB0}{\textsuperscript}{\textscn}
\DeclareUTFcomposite[\UTFencname]{x1DB1}{\textsuperscript}{\textbaro}
\DeclareUTFcomposite[\UTFencname]{x1DB2}{\textsuperscript}{\textphi}
\DeclareUTFcomposite[\UTFencname]{x1DB3}{\textsuperscript}{\textscn}
\DeclareUTFcomposite[\UTFencname]{x1DB4}{\textsuperscript}{\textesh}
\DeclareUTFcomposite[\UTFencname]{x1DB5}{\textsuperscript}{\textlhookt}
\DeclareUTFcomposite[\UTFencname]{x1DB6}{\textsuperscript}{\textbaru}
\DeclareUTFcomposite[\UTFencname]{x1DB7}{\textsuperscript}{\textupsilon}
\DeclareUTFcomposite[\UTFencname]{x1DB7}{\textsuperscript}{\textscupsilon}
\DeclareUTFcomposite[\UTFencname]{x1DB8}{\textsuperscript}{\textscu}
\DeclareUTFcomposite[\UTFencname]{x1DB9}{\textsuperscript}{\textscriptv}
\DeclareUTFcomposite[\UTFencname]{x1DBA}{\textsuperscript}{\textturnv}
\DeclareUTFcomposite[\UTFencname]{x1DBB}{\textsuperscript}{z}
\DeclareUTFcomposite[\UTFencname]{x1DBC}{\textsuperscript}{\textrtailz}
\DeclareUTFcomposite[\UTFencname]{x1DBD}{\textsuperscript}{\textctz}
\DeclareUTFcomposite[\UTFencname]{x1DBE}{\textsuperscript}{\textschwa}
\DeclareUTFcomposite[\UTFencname]{x1DBE}{\textsuperscript}{\schwa}
\DeclareUTFcomposite[\UTFencname]{x1DBF}{\textsuperscript}{\texttheta}


\DeclareUTFcomposite[\UTFencname]{x1E00}{\textsubring}{A} % with ring below
\DeclareUTFcomposite[\UTFencname]{x1E01}{\textsubring}{a} % with ring below
\DeclareUTFcomposite[\UTFencname]{x1E02}{\.}{B}
\DeclareUTFcomposite[\UTFencname]{x1E02}{\capitaldotaccent}{B}
\DeclareUTFcomposite[\UTFencname]{x1E03}{\.}{b}
\DeclareUTFcomposite[\UTFencname]{x1E04}{\d}{B}
\DeclareUTFcomposite[\UTFencname]{x1E05}{\d}{b}
\DeclareUTFcomposite[\UTFencname]{x1E06}{\b}{B}
\DeclareUTFcomposite[\UTFencname]{x1E07}{\b}{b}
\DeclareUTFcomposite[\UTFencname]{x1E08}{\'}{\c C}
\DeclareUTFcomposite[\UTFencname]{x1E09}{\'}{\c C}
\DeclareUTFcomposite[\UTFencname]{x1E0A}{\.}{D}
\DeclareUTFcomposite[\UTFencname]{x1E0A}{\capitaldotaccent}{D}
\DeclareUTFcomposite[\UTFencname]{x1E0B}{\.}{d}
\DeclareUTFcomposite[\UTFencname]{x1E0C}{\d}{D}
\DeclareUTFcomposite[\UTFencname]{x1E0D}{\d}{d}
\DeclareUTFcomposite[\UTFencname]{x1E0E}{\b}{D}
\DeclareUTFcomposite[\UTFencname]{x1E0F}{\b}{d}
\DeclareUTFcomposite[\UTFencname]{x1E10}{\c}{D}
\DeclareUTFcomposite[\UTFencname]{x1E10}{\capitalcedilla}{D}
\DeclareUTFcomposite[\UTFencname]{x1E11}{\c}{d}
\DeclareUTFcomposite[\UTFencname]{x1E12}{\textsubcircum}{D} % with circumflex below
\DeclareUTFcomposite[\UTFencname]{x1E13}{\textsubcircum}{d} % with circumflex below
\DeclareUTFcomposite[\UTFencname]{x1E14}{\`}{\=E}
\DeclareUTFcomposite[\UTFencname]{x1E14}{\textgravemacron}{E}
\DeclareUTFcomposite[\UTFencname]{x1E15}{\`}{\=e}
\DeclareUTFcomposite[\UTFencname]{x1E15}{\textgravemacron}{e}
\DeclareUTFcomposite[\UTFencname]{x1E16}{\'}{\=E}
\DeclareUTFcomposite[\UTFencname]{x1E16}{\textacutemacron}{E}
\DeclareUTFcomposite[\UTFencname]{x1E17}{\'}{\=e}
\DeclareUTFcomposite[\UTFencname]{x1E17}{\textacutemacron}{e}
\DeclareUTFcomposite[\UTFencname]{x1E18}{\textsubcircum}{E} % with circumflex below
\DeclareUTFcomposite[\UTFencname]{x1E19}{\textsubcircum}{e} % with circumflex below
\DeclareUTFcomposite[\UTFencname]{x1E1A}{\textsubtilde}{E} % with tilde below
\DeclareUTFcomposite[\UTFencname]{x1E1B}{\textsubtilde}{e} % with tilde below
\DeclareUTFcomposite[\UTFencname]{x1E1C}{\u}{\c E}
\DeclareUTFcomposite[\UTFencname]{x1E1D}{\u}{\c e}
\DeclareUTFcomposite[\UTFencname]{x1E1E}{\.}{F}
\DeclareUTFcomposite[\UTFencname]{x1E1E}{\capitaldotaccent}{F}
\DeclareUTFcomposite[\UTFencname]{x1E1F}{\.}{f}
\DeclareUTFcomposite[\UTFencname]{x1E20}{\=}{G}
\DeclareUTFcomposite[\UTFencname]{x1E20}{\capitalmacron}{G}
\DeclareUTFcomposite[\UTFencname]{x1E21}{\=}{g}
\DeclareUTFcomposite[\UTFencname]{x1E22}{\.}{H}
\DeclareUTFcomposite[\UTFencname]{x1E22}{\capitaldotaccent}{H}
\DeclareUTFcomposite[\UTFencname]{x1E23}{\.}{h}
\DeclareUTFcomposite[\UTFencname]{x1E24}{\d}{H}
\DeclareUTFcomposite[\UTFencname]{x1E25}{\d}{h}
\DeclareUTFcomposite[\UTFencname]{x1E26}{\"}{H}
\DeclareUTFcomposite[\UTFencname]{x1E26}{\capitaldieresis}{H}
\DeclareUTFcomposite[\UTFencname]{x1E27}{\"}{h}
\DeclareUTFcomposite[\UTFencname]{x1E28}{\c}{H}
\DeclareUTFcomposite[\UTFencname]{x1E28}{\capitalcedilla}{H}
\DeclareUTFcomposite[\UTFencname]{x1E29}{\c}{h}
\DeclareUTFcomposite[\UTFencname]{x1E2A}{\textbottomtiebar}{H} % with breve below
\DeclareUTFcomposite[\UTFencname]{x1E2B}{\textbottomtiebar}{h} % with breve below
\DeclareUTFcomposite[\UTFencname]{x1E2C}{\textsubtilde}{I} % with tilde below
\DeclareUTFcomposite[\UTFencname]{x1E2D}{\textsubtilde}{i} % with tilde below
\DeclareUTFcomposite[\UTFencname]{x1E2E}{\'}{\"I}
\DeclareUTFcomposite[\UTFencname]{x1E2F}{\'}{\"i}
\DeclareUTFcomposite[\UTFencname]{x1E30}{\'}{K}
\DeclareUTFcomposite[\UTFencname]{x1E30}{\capitalacute}{K}
\DeclareUTFcomposite[\UTFencname]{x1E31}{\'}{k}
\DeclareUTFcomposite[\UTFencname]{x1E32}{\d}{K}
\DeclareUTFcomposite[\UTFencname]{x1E33}{\d}{k}
\DeclareUTFcomposite[\UTFencname]{x1E34}{\b}{K}
\DeclareUTFcomposite[\UTFencname]{x1E35}{\b}{k}
\DeclareUTFcomposite[\UTFencname]{x1E36}{\d}{L}
\DeclareUTFcomposite[\UTFencname]{x1E37}{\d}{l}
\DeclareUTFcomposite[\UTFencname]{x1E38}{\=}{\.L}
\DeclareUTFcomposite[\UTFencname]{x1E39}{\=}{\.l}
\DeclareUTFcomposite[\UTFencname]{x1E3A}{\b}{L}
\DeclareUTFcomposite[\UTFencname]{x1E3B}{\b}{l}
\DeclareUTFcomposite[\UTFencname]{x1E3C}{\textsubcircum}{L} % with circumflex below
\DeclareUTFcomposite[\UTFencname]{x1E3D}{\textsubcircum}{l} % with circumflex below
\DeclareUTFcomposite[\UTFencname]{x1E3E}{\'}{M}
\DeclareUTFcomposite[\UTFencname]{x1E3E}{\capitalacute}{M}
\DeclareUTFcomposite[\UTFencname]{x1E3F}{\'}{m}
\DeclareUTFcomposite[\UTFencname]{x1E40}{\.}{M}
\DeclareUTFcomposite[\UTFencname]{x1E40}{\capitaldotaccent}{M}
\DeclareUTFcomposite[\UTFencname]{x1E41}{\.}{m}
\DeclareUTFcomposite[\UTFencname]{x1E42}{\d}{M}
\DeclareUTFcomposite[\UTFencname]{x1E43}{\d}{m}
\DeclareUTFcomposite[\UTFencname]{x1E44}{\.}{N}
\DeclareUTFcomposite[\UTFencname]{x1E44}{\capitaldotaccent}{N}
\DeclareUTFcomposite[\UTFencname]{x1E45}{\.}{n}
\DeclareUTFcomposite[\UTFencname]{x1E46}{\d}{N}
\DeclareUTFcomposite[\UTFencname]{x1E47}{\d}{n}
\DeclareUTFcomposite[\UTFencname]{x1E48}{\b}{N}
\DeclareUTFcomposite[\UTFencname]{x1E49}{\b}{n}
\DeclareUTFcomposite[\UTFencname]{x1E4A}{\textsubcircum}{N} % with circumflex below
\DeclareUTFcomposite[\UTFencname]{x1E4B}{\textsubcircum}{n} % with circumflex below
\DeclareUTFcomposite[\UTFencname]{x1E4C}{\'}{\~O}
\DeclareUTFcomposite[\UTFencname]{x1E4D}{\'}{\~o}
\DeclareUTFcomposite[\UTFencname]{x1E4E}{\"}{\~O}
\DeclareUTFcomposite[\UTFencname]{x1E4F}{\"}{\~o}
\DeclareUTFcomposite[\UTFencname]{x1E50}{\`}{\=O}
\DeclareUTFcomposite[\UTFencname]{x1E50}{\textgravemacron}{O}
\DeclareUTFcomposite[\UTFencname]{x1E51}{\`}{\=o}
\DeclareUTFcomposite[\UTFencname]{x1E51}{\textgravemacron}{o}
\DeclareUTFcomposite[\UTFencname]{x1E52}{\'}{\=O}
\DeclareUTFcomposite[\UTFencname]{x1E52}{\textacutemacron}{O}
\DeclareUTFcomposite[\UTFencname]{x1E53}{\'}{\=o}
\DeclareUTFcomposite[\UTFencname]{x1E53}{\textacutemacron}{o}
\DeclareUTFcomposite[\UTFencname]{x1E54}{\'}{P}
\DeclareUTFcomposite[\UTFencname]{x1E54}{\capitalacute}{P}
\DeclareUTFcomposite[\UTFencname]{x1E55}{\'}{p}
\DeclareUTFcomposite[\UTFencname]{x1E56}{\.}{P}
\DeclareUTFcomposite[\UTFencname]{x1E56}{\capitaldotaccent}{P}
\DeclareUTFcomposite[\UTFencname]{x1E57}{\.}{p}
\DeclareUTFcomposite[\UTFencname]{x1E58}{\.}{R}
\DeclareUTFcomposite[\UTFencname]{x1E58}{\capitaldotaccent}{R}
\DeclareUTFcomposite[\UTFencname]{x1E59}{\.}{r}
\DeclareUTFcomposite[\UTFencname]{x1E5A}{\d}{R}
\DeclareUTFcomposite[\UTFencname]{x1E5B}{\d}{r}
\DeclareUTFcomposite[\UTFencname]{x1E5C}{\=}{\d R}
\DeclareUTFcomposite[\UTFencname]{x1E5D}{\=}{\d r}
\DeclareUTFcomposite[\UTFencname]{x1E5E}{\b}{R}
\DeclareUTFcomposite[\UTFencname]{x1E5F}{\b}{r}
\DeclareUTFcomposite[\UTFencname]{x1E60}{\.}{S}
\DeclareUTFcomposite[\UTFencname]{x1E60}{\capitaldotaccent}{S}
\DeclareUTFcomposite[\UTFencname]{x1E61}{\.}{s}
\DeclareUTFcomposite[\UTFencname]{x1E62}{\d}{S}
\DeclareUTFcomposite[\UTFencname]{x1E63}{\d}{s}
\DeclareUTFcomposite[\UTFencname]{x1E64}{\.}{\'S}
\DeclareUTFcomposite[\UTFencname]{x1E64}{\textdotacute}{S}
\DeclareUTFcomposite[\UTFencname]{x1E65}{\.}{\'s}
\DeclareUTFcomposite[\UTFencname]{x1E65}{\textdotacute}{s}
\DeclareUTFcomposite[\UTFencname]{x1E66}{\.}{\v S}
\DeclareUTFcomposite[\UTFencname]{x1E67}{\.}{\v s}
\DeclareUTFcomposite[\UTFencname]{x1E68}{\.}{\d S}
\DeclareUTFcomposite[\UTFencname]{x1E69}{\.}{\d s}
\DeclareUTFcomposite[\UTFencname]{x1E6A}{\.}{T}
\DeclareUTFcomposite[\UTFencname]{x1E6A}{\capitaldotaccent}{T}
\DeclareUTFcomposite[\UTFencname]{x1E6B}{\.}{t}
\DeclareUTFcomposite[\UTFencname]{x1E6C}{\d}{T}
\DeclareUTFcomposite[\UTFencname]{x1E6D}{\d}{t}
\DeclareUTFcomposite[\UTFencname]{x1E6E}{\b}{T}
\DeclareUTFcomposite[\UTFencname]{x1E6F}{\b}{t}
\DeclareUTFcomposite[\UTFencname]{x1E70}{\textsubcircum}{T} % with circumflex below
\DeclareUTFcomposite[\UTFencname]{x1E71}{\textsubcircum}{t} % with circumflex below
\DeclareUTFcomposite[\UTFencname]{x1E72}{\textsubumlaut}{U} % with dieresis below
\DeclareUTFcomposite[\UTFencname]{x1E73}{\textsubumlaut}{u} % with dieresis below
\DeclareUTFcomposite[\UTFencname]{x1E74}{\textsubtilde}{U} % with tilde below
\DeclareUTFcomposite[\UTFencname]{x1E75}{\textsubtilde}{u} % with tilde below
\DeclareUTFcomposite[\UTFencname]{x1E76}{\textsubcircum}{U} % with circumflex below
\DeclareUTFcomposite[\UTFencname]{x1E77}{\textsubcircum}{u} % with circumflex below
\DeclareUTFcomposite[\UTFencname]{x1E78}{\~}{\'U}
\DeclareUTFcomposite[\UTFencname]{x1E79}{\~}{\'u}
\DeclareUTFcomposite[\UTFencname]{x1E7A}{\"}{\=U}
\DeclareUTFcomposite[\UTFencname]{x1E7B}{\"}{\=u}
\DeclareUTFcomposite[\UTFencname]{x1E7C}{\~}{V}
\DeclareUTFcomposite[\UTFencname]{x1E7C}{\capitaltilde}{V}
\DeclareUTFcomposite[\UTFencname]{x1E7D}{\~}{v}
\DeclareUTFcomposite[\UTFencname]{x1E7E}{\d}{V}
\DeclareUTFcomposite[\UTFencname]{x1E7F}{\d}{v}
\DeclareUTFcomposite[\UTFencname]{x1E80}{\`}{W}
\DeclareUTFcomposite[\UTFencname]{x1E80}{\capitalgrave}{W}
\DeclareUTFcomposite[\UTFencname]{x1E81}{\`}{w}
\DeclareUTFcomposite[\UTFencname]{x1E82}{\'}{W}
\DeclareUTFcomposite[\UTFencname]{x1E82}{\capitalacute}{W}
\DeclareUTFcomposite[\UTFencname]{x1E83}{\'}{w}
\DeclareUTFcomposite[\UTFencname]{x1E84}{\"}{W}
\DeclareUTFcomposite[\UTFencname]{x1E84}{\capitaldieresis}{W}
\DeclareUTFcomposite[\UTFencname]{x1E85}{\"}{w}
\DeclareUTFcomposite[\UTFencname]{x1E86}{\.}{W}
\DeclareUTFcomposite[\UTFencname]{x1E86}{\capitaldotaccent}{W}
\DeclareUTFcomposite[\UTFencname]{x1E87}{\.}{w}
\DeclareUTFcomposite[\UTFencname]{x1E88}{\d}{W}
\DeclareUTFcomposite[\UTFencname]{x1E89}{\d}{w}
\DeclareUTFcomposite[\UTFencname]{x1E8A}{\.}{X}
\DeclareUTFcomposite[\UTFencname]{x1E8A}{\capitaldotaccent}{X}
\DeclareUTFcomposite[\UTFencname]{x1E8B}{\.}{x}
\DeclareUTFcomposite[\UTFencname]{x1E8C}{\"}{X}
\DeclareUTFcomposite[\UTFencname]{x1E8C}{\capitaldieresis}{X}
\DeclareUTFcomposite[\UTFencname]{x1E8D}{\"}{x}
\DeclareUTFcomposite[\UTFencname]{x1E8E}{\.}{Y}
\DeclareUTFcomposite[\UTFencname]{x1E8E}{\capitaldotaccent}{Y}
\DeclareUTFcomposite[\UTFencname]{x1E8F}{\.}{y}
\DeclareUTFcomposite[\UTFencname]{x1E90}{\^}{Z}
\DeclareUTFcomposite[\UTFencname]{x1E90}{\capitalcircumflex}{Z}
\DeclareUTFcomposite[\UTFencname]{x1E91}{\^}{z}
\DeclareUTFcomposite[\UTFencname]{x1E92}{\d}{Z}
\DeclareUTFcomposite[\UTFencname]{x1E93}{\d}{z}
\DeclareUTFcomposite[\UTFencname]{x1E94}{\b}{Z}
\DeclareUTFcomposite[\UTFencname]{x1E95}{\b}{z}
\DeclareUTFcomposite[\UTFencname]{x1E96}{\b}{h}
\DeclareUTFcomposite[\UTFencname]{x1E97}{\"}{t}
\DeclareUTFcomposite[\UTFencname]{x1E98}{\r}{w}
\DeclareUTFcomposite[\UTFencname]{x1E99}{\r}{y}

\DeclareUTFcomposite[\UTFencname]{x1EA0}{\d}{A}
\DeclareUTFcomposite[\UTFencname]{x1EA1}{\d}{a}
\DeclareUTFcomposite[\UTFencname]{x1EA2}{\texthookabove}{A} % with hook above
\DeclareUTFcomposite[\UTFencname]{x1EA3}{\texthookabove}{a} % with hook above
\DeclareUTFcomposite[\UTFencname]{x1EA4}{\'}{\^A}
\DeclareUTFcomposite[\UTFencname]{x1EA4}{\textcircumacute}{A}
\DeclareUTFcomposite[\UTFencname]{x1EA5}{\'}{\^a}
\DeclareUTFcomposite[\UTFencname]{x1EA5}{\textcircumacute}{a}
\DeclareUTFcomposite[\UTFencname]{x1EA6}{\`}{\^A}
\DeclareUTFcomposite[\UTFencname]{x1EA6}{\textcircumgrave}{A}
\DeclareUTFcomposite[\UTFencname]{x1EA7}{\`}{\^a}
\DeclareUTFcomposite[\UTFencname]{x1EA7}{\textcircumgrave}{a}
\DeclareUTFcomposite[\UTFencname]{x1EA8}{\texthookabove}{\^A} % with hook above
\DeclareUTFcomposite[\UTFencname]{x1EA8}{\texthookcircum}{A}
\DeclareUTFcomposite[\UTFencname]{x1EA9}{\texthookabove}{\^a} % with hook above
\DeclareUTFcomposite[\UTFencname]{x1EA9}{\texthookcircum}{a}
\DeclareUTFcomposite[\UTFencname]{x1EAA}{\~}{\^A}
\DeclareUTFcomposite[\UTFencname]{x1EAA}{\texttildecircum}{A}
\DeclareUTFcomposite[\UTFencname]{x1EAB}{\~}{\^a}
\DeclareUTFcomposite[\UTFencname]{x1EAB}{\texttildecircum}{a}
\DeclareUTFcomposite[\UTFencname]{x1EAC}{\d}{\^A}
\DeclareUTFcomposite[\UTFencname]{x1EAC}{\textcircumdotbelow}{A}
\DeclareUTFcomposite[\UTFencname]{x1EAD}{\d}{\^a}
\DeclareUTFcomposite[\UTFencname]{x1EAD}{\textcircumdotbelow}{a}
\DeclareUTFcomposite[\UTFencname]{x1EAE}{\'}{\u A}
\DeclareUTFcomposite[\UTFencname]{x1EAE}{\textbreveacute}{A}
\DeclareUTFcomposite[\UTFencname]{x1EAF}{\'}{\u a}
\DeclareUTFcomposite[\UTFencname]{x1EAF}{\textbreveacute}{a}
\DeclareUTFcomposite[\UTFencname]{x1EB0}{\`}{\u A}
\DeclareUTFcomposite[\UTFencname]{x1EB0}{\textbrevegrave}{A}
\DeclareUTFcomposite[\UTFencname]{x1EB1}{\`}{\u a}
\DeclareUTFcomposite[\UTFencname]{x1EB1}{\textbrevegrave}{a}
\DeclareUTFcomposite[\UTFencname]{x1EB2}{\texthookabove}{\u A} % with hook above
\DeclareUTFcomposite[\UTFencname]{x1EB2}{\textbrevehook}{A}
\DeclareUTFcomposite[\UTFencname]{x1EB3}{\texthookabove}{\u a} % with hook above
\DeclareUTFcomposite[\UTFencname]{x1EB3}{\textbrevehook}{a}
\DeclareUTFcomposite[\UTFencname]{x1EB4}{\~}{\u A}
\DeclareUTFcomposite[\UTFencname]{x1EB4}{\textbrevetilde}{A}
\DeclareUTFcomposite[\UTFencname]{x1EB5}{\~}{\u a}
\DeclareUTFcomposite[\UTFencname]{x1EB5}{\textbrevetilde}{a}
\DeclareUTFcomposite[\UTFencname]{x1EB6}{\d}{\u A}
\DeclareUTFcomposite[\UTFencname]{x1EB6}{\textbrevedotbelow}{A}
\DeclareUTFcomposite[\UTFencname]{x1EB7}{\d}{\u a}
\DeclareUTFcomposite[\UTFencname]{x1EB7}{\textbrevedotbelow}{a}
\DeclareUTFcomposite[\UTFencname]{x1EB8}{\d}{E}
\DeclareUTFcomposite[\UTFencname]{x1EB9}{\d}{e}
\DeclareUTFcomposite[\UTFencname]{x1EBA}{\texthookabove}{E} % with hook above
\DeclareUTFcomposite[\UTFencname]{x1EBB}{\texthookabove}{e} % with hook above
\DeclareUTFcomposite[\UTFencname]{x1EBC}{\~}{E}
\DeclareUTFcomposite[\UTFencname]{x1EBC}{\capitaltilde}{E}
\DeclareUTFcomposite[\UTFencname]{x1EBD}{\~}{e}
\DeclareUTFcomposite[\UTFencname]{x1EBE}{\'}{\^E}
\DeclareUTFcomposite[\UTFencname]{x1EBE}{\textcircumacute}{E}
\DeclareUTFcomposite[\UTFencname]{x1EBF}{\'}{\^e}
\DeclareUTFcomposite[\UTFencname]{x1EBF}{\textcircumacute}{e}
\DeclareUTFcomposite[\UTFencname]{x1EC0}{\`}{\^E}
\DeclareUTFcomposite[\UTFencname]{x1EC0}{\textcircumgrave}{E}
\DeclareUTFcomposite[\UTFencname]{x1EC1}{\`}{\^e}
\DeclareUTFcomposite[\UTFencname]{x1EC1}{\textcircumgrave}{e}
\DeclareUTFcomposite[\UTFencname]{x1EC2}{\texthookabove}{\^E} % with hook above
\DeclareUTFcomposite[\UTFencname]{x1EC2}{\texthookcircum}{E}
\DeclareUTFcomposite[\UTFencname]{x1EC3}{\texthookabove}{\^e} % with hook above
\DeclareUTFcomposite[\UTFencname]{x1EC3}{\texthookcircum}{e}
\DeclareUTFcomposite[\UTFencname]{x1EC4}{\~}{\^E}
\DeclareUTFcomposite[\UTFencname]{x1EC4}{\texttildecircum}{E}
\DeclareUTFcomposite[\UTFencname]{x1EC5}{\~}{\^e}
\DeclareUTFcomposite[\UTFencname]{x1EC5}{\texttildecircum}{e}
\DeclareUTFcomposite[\UTFencname]{x1EC6}{\d}{\^E}
\DeclareUTFcomposite[\UTFencname]{x1EC6}{\textcircumdotbelow}{E}
\DeclareUTFcomposite[\UTFencname]{x1EC7}{\d}{\^e}
\DeclareUTFcomposite[\UTFencname]{x1EC7}{\textcircumdotbelow}{e}
\DeclareUTFcomposite[\UTFencname]{x1EC8}{\texthookabove}{I} % with hook above
\DeclareUTFcomposite[\UTFencname]{x1EC9}{\texthookabove}{i} % with hook above
\DeclareUTFcomposite[\UTFencname]{x1EC9}{\texthookabove}{\i} % with hook above
\DeclareUTFcomposite[\UTFencname]{x1ECA}{\d}{I}
\DeclareUTFcomposite[\UTFencname]{x1ECB}{\d}{i}
\DeclareUTFcomposite[\UTFencname]{x1ECC}{\d}{O}
\DeclareUTFcomposite[\UTFencname]{x1ECD}{\d}{o}
\DeclareUTFcomposite[\UTFencname]{x1ECE}{\texthookabove}{O} % with hook above
\DeclareUTFcomposite[\UTFencname]{x1ECF}{\texthookabove}{o} % with hook above
\DeclareUTFcomposite[\UTFencname]{x1ED0}{\'}{\^O}
\DeclareUTFcomposite[\UTFencname]{x1ED0}{\textcircumacute}{O}
\DeclareUTFcomposite[\UTFencname]{x1ED1}{\'}{\^o}
\DeclareUTFcomposite[\UTFencname]{x1ED1}{\textcircumacute}{o}
\DeclareUTFcomposite[\UTFencname]{x1ED2}{\`}{\^O}
\DeclareUTFcomposite[\UTFencname]{x1ED2}{\textcircumgrave}{O}
\DeclareUTFcomposite[\UTFencname]{x1ED3}{\`}{\^o}
\DeclareUTFcomposite[\UTFencname]{x1ED3}{\textcircumgrave}{o}
\DeclareUTFcomposite[\UTFencname]{x1ED4}{\texthookabove}{\^O} % with hook above
\DeclareUTFcomposite[\UTFencname]{x1ED4}{\texthookcircum}{O}
\DeclareUTFcomposite[\UTFencname]{x1ED5}{\texthookabove}{\^o} % with hook above
\DeclareUTFcomposite[\UTFencname]{x1ED5}{\texthookcircum}{o}
\DeclareUTFcomposite[\UTFencname]{x1ED6}{\~}{\^O}
\DeclareUTFcomposite[\UTFencname]{x1ED6}{\texttildecircum}{O}
\DeclareUTFcomposite[\UTFencname]{x1ED7}{\~}{\^o}
\DeclareUTFcomposite[\UTFencname]{x1ED7}{\texttildecircum}{o}
\DeclareUTFcomposite[\UTFencname]{x1ED8}{\d}{\^O}
\DeclareUTFcomposite[\UTFencname]{x1ED8}{\textcircumdotbelow}{O}
\DeclareUTFcomposite[\UTFencname]{x1ED9}{\d}{\^o}
\DeclareUTFcomposite[\UTFencname]{x1ED9}{\textcircumdotbelow}{o}
\DeclareUTFcomposite[\UTFencname]{x1EDA}{\'}{\textrighthorn O} % O with horn
\DeclareUTFcomposite[\UTFencname]{x1EDA}{\textacutehorn}{O}
\DeclareUTFcomposite[\UTFencname]{x1EDB}{\'}{\textrighthorn o} % o with horn
\DeclareUTFcomposite[\UTFencname]{x1EDB}{\textacutehorn}{o}
\DeclareUTFcomposite[\UTFencname]{x1EDC}{\`}{\textrighthorn O} % O with horn
\DeclareUTFcomposite[\UTFencname]{x1EDC}{\textgravehorn}{O}
\DeclareUTFcomposite[\UTFencname]{x1EDD}{\`}{\textrighthorn o} % o with horn
\DeclareUTFcomposite[\UTFencname]{x1EDD}{\textgravehorn}{o}
\DeclareUTFcomposite[\UTFencname]{x1EDE}{\texthookabove}{\textrighthorn O} % with hook above and horn
\DeclareUTFcomposite[\UTFencname]{x1EDE}{\texthookhorn}{O}
\DeclareUTFcomposite[\UTFencname]{x1EDF}{\texthookabove}{\textrighthorn o} % with hook above and horn
\DeclareUTFcomposite[\UTFencname]{x1EDF}{\texthookhorn}{o}
\DeclareUTFcomposite[\UTFencname]{x1EE0}{\~}{\textrighthorn O} % O with horn
\DeclareUTFcomposite[\UTFencname]{x1EE0}{\texttildehorn}{O}
\DeclareUTFcomposite[\UTFencname]{x1EE1}{\~}{\textrighthorn o} % o with horn
\DeclareUTFcomposite[\UTFencname]{x1EE1}{\texttildehorn}{o}
\DeclareUTFcomposite[\UTFencname]{x1EE2}{\d}{\textrighthorn O} % O with horn
\DeclareUTFcomposite[\UTFencname]{x1EE2}{\textdotbelowhorn}{O}
\DeclareUTFcomposite[\UTFencname]{x1EE3}{\d}{\textrighthorn o} % o with horn
\DeclareUTFcomposite[\UTFencname]{x1EE3}{\textdotbelowhorn}{o}
\DeclareUTFcomposite[\UTFencname]{x1EE4}{\d}{U}
\DeclareUTFcomposite[\UTFencname]{x1EE5}{\d}{u}
\DeclareUTFcomposite[\UTFencname]{x1EE6}{\texthookabove}{U} % with hook above
\DeclareUTFcomposite[\UTFencname]{x1EE7}{\texthookabove}{u} % with hook above
\DeclareUTFcomposite[\UTFencname]{x1EE8}{\'}{\textrighthorn U} % U with horn
\DeclareUTFcomposite[\UTFencname]{x1EE8}{\textacutehorn}{U}
\DeclareUTFcomposite[\UTFencname]{x1EE9}{\'}{\textrighthorn u} % u with horn
\DeclareUTFcomposite[\UTFencname]{x1EE9}{\textacutehorn}{u}
\DeclareUTFcomposite[\UTFencname]{x1EEA}{\`}{\textrighthorn U} % U with horn
\DeclareUTFcomposite[\UTFencname]{x1EEA}{\textgravehorn}{U}
\DeclareUTFcomposite[\UTFencname]{x1EEB}{\`}{\textrighthorn u} % u with horn
\DeclareUTFcomposite[\UTFencname]{x1EEB}{\textgravehorn}{u}
\DeclareUTFcomposite[\UTFencname]{x1EEC}{\texthookabove}{\textrighthorn U} % with hook above and horn
\DeclareUTFcomposite[\UTFencname]{x1EEC}{\texthookhorn}{U}
\DeclareUTFcomposite[\UTFencname]{x1EED}{\texthookabove}{\textrighthorn u} % with hook above and horn
\DeclareUTFcomposite[\UTFencname]{x1EED}{\texthookhorn}{u}
\DeclareUTFcomposite[\UTFencname]{x1EEE}{\~}{\textrighthorn U} % U with horn
\DeclareUTFcomposite[\UTFencname]{x1EEE}{\texttildehorn}{U}
\DeclareUTFcomposite[\UTFencname]{x1EEF}{\~}{\textrighthorn u} % u with horn
\DeclareUTFcomposite[\UTFencname]{x1EEF}{\texttildehorn}{u}
\DeclareUTFcomposite[\UTFencname]{x1EF0}{\d}{\textrighthorn U} % U with horn
\DeclareUTFcomposite[\UTFencname]{x1EF0}{\textdotbelowhorn}{U}
\DeclareUTFcomposite[\UTFencname]{x1EF1}{\d}{\textrighthorn u} % u with horn
\DeclareUTFcomposite[\UTFencname]{x1EF1}{\textdotbelowhorn}{u}
\DeclareUTFcomposite[\UTFencname]{x1EF2}{\`}{Y}
\DeclareUTFcomposite[\UTFencname]{x1EF2}{\capitalgrave}{Y}
\DeclareUTFcomposite[\UTFencname]{x1EF3}{\`}{y}
\DeclareUTFcomposite[\UTFencname]{x1EF4}{\d}{Y}
\DeclareUTFcomposite[\UTFencname]{x1EF5}{\d}{y}
\DeclareUTFcomposite[\UTFencname]{x1EF6}{\texthookabove}{Y} % with hook above
\DeclareUTFcomposite[\UTFencname]{x1EF7}{\texthookabove}{y} % with hook above
\DeclareUTFcomposite[\UTFencname]{x1EF8}{\~}{Y}
\DeclareUTFcomposite[\UTFencname]{x1EF8}{\capitaltilde}{Y}
\DeclareUTFcomposite[\UTFencname]{x1EF9}{\~}{y}


\DeclareUTFcharacter[\UTFencname]{x2012}{\textthreequartersemdash}
\DeclareUTFcharacter[\UTFencname]{x2013}{\textendash}
\DeclareUTFcharacter[\UTFencname]{x2014}{\textemdash}
\DeclareUTFcharacter[\UTFencname]{x2015}{\texttwelveudash}
\DeclareUTFcharacter[\UTFencname]{x2016}{\textbardbl}
\DeclareUTFcharacter[\UTFencname]{x2016}{\textdoublevertline}
\DeclareUTFcharacter[\UTFencname]{x2018}{\textquoteleft}
\DeclareUTFcharacter[\UTFencname]{x2019}{\textquoteright}
\DeclareUTFcharacter[\UTFencname]{x201A}{\quotesinglbase}
\DeclareUTFcharacter[\UTFencname]{x201A}{\textquotestraightbase}
\DeclareUTFcharacter[\UTFencname]{x201C}{\textquotedblleft}
\DeclareUTFcharacter[\UTFencname]{x201D}{\textquotedblright}
\DeclareUTFcharacter[\UTFencname]{x201E}{\quotedblbase}
\DeclareUTFcharacter[\UTFencname]{x201E}{\textquotestraightdblbase}
\DeclareUTFcharacter[\UTFencname]{x2020}{\textdagger}
\DeclareUTFcharacter[\UTFencname]{x2021}{\textdaggerdbl}
\DeclareUTFcharacter[\UTFencname]{x2022}{\textbullet}
\DeclareUTFcharacter[\UTFencname]{x2026}{\textellipsis}
\DeclareUTFcharacter[\UTFencname]{x2030}{\textperthousand}
\DeclareUTFcharacter[\UTFencname]{x2031}{\textpertenthousand}
\DeclareUTFcharacter[\UTFencname]{x2031}{\textpermille}
\DeclareUTFcharacter[\UTFencname]{x2038}{\textcaret}
\DeclareUTFcharacter[\UTFencname]{x2039}{\guilsinglleft}
\DeclareUTFcharacter[\UTFencname]{x203A}{\guilsinglright}
\DeclareUTFcharacter[\UTFencname]{x203B}{\textreferencemark}
\DeclareUTFcharacter[\UTFencname]{x203D}{\textinterrobang}
\DeclareUTFcharacter[\UTFencname]{x203F}{\textbottomtiebar}
\DeclareUTFcharacter[\UTFencname]{x2042}{\textasterism}
\DeclareUTFcharacter[\UTFencname]{x2044}{\textfractionsolidus}
\DeclareUTFcharacter[\UTFencname]{x2045}{\textlquill}
\DeclareUTFcharacter[\UTFencname]{x2046}{\textrquill}
\DeclareUTFcharacter[\UTFencname]{x2052}{\textdiscount}

% superscripts and subscripts
\DeclareUTFcomposite[\UTFencname]{x2070}{\textsuperscript}{0}
\DeclareUTFcomposite[\UTFencname]{x2071}{\textsuperscript}{i}
\DeclareUTFcomposite[\UTFencname]{x2074}{\textsuperscript}{4}
\DeclareUTFcomposite[\UTFencname]{x2075}{\textsuperscript}{5}
\DeclareUTFcomposite[\UTFencname]{x2076}{\textsuperscript}{6}
\DeclareUTFcomposite[\UTFencname]{x2077}{\textsuperscript}{7}
\DeclareUTFcomposite[\UTFencname]{x2078}{\textsuperscript}{8}
\DeclareUTFcomposite[\UTFencname]{x2079}{\textsuperscript}{9}
\DeclareUTFcomposite[\UTFencname]{x207A}{\textsuperscript}{+}
\DeclareUTFcomposite[\UTFencname]{x207B}{\textsuperscript}{-}
\DeclareUTFcomposite[\UTFencname]{x207C}{\textsuperscript}{=}
\DeclareUTFcomposite[\UTFencname]{x207D}{\textsuperscript}{(}
\DeclareUTFcomposite[\UTFencname]{x207E}{\textsuperscript}{)}
\DeclareUTFcomposite[\UTFencname]{x207F}{\textsuperscript}{n}
\DeclareUTFcomposite[\UTFencname]{x2080}{\textsubscript}{0}
\DeclareUTFcomposite[\UTFencname]{x2081}{\textsubscript}{1}
\DeclareUTFcomposite[\UTFencname]{x2082}{\textsubscript}{2}
\DeclareUTFcomposite[\UTFencname]{x2083}{\textsubscript}{3}
\DeclareUTFcomposite[\UTFencname]{x2084}{\textsubscript}{4}
\DeclareUTFcomposite[\UTFencname]{x2085}{\textsubscript}{5}
\DeclareUTFcomposite[\UTFencname]{x2086}{\textsubscript}{6}
\DeclareUTFcomposite[\UTFencname]{x2087}{\textsubscript}{7}
\DeclareUTFcomposite[\UTFencname]{x2088}{\textsubscript}{8}
\DeclareUTFcomposite[\UTFencname]{x2089}{\textsubscript}{9}
\DeclareUTFcomposite[\UTFencname]{x208A}{\textsubscript}{+}
\DeclareUTFcomposite[\UTFencname]{x208B}{\textsubscript}{-}
\DeclareUTFcomposite[\UTFencname]{x208C}{\textsubscript}{=}
\DeclareUTFcomposite[\UTFencname]{x208D}{\textsubscript}{(}
\DeclareUTFcomposite[\UTFencname]{x208E}{\textsubscript}{)}
\DeclareUTFcomposite[\UTFencname]{x2090}{\textsubscript}{a}
\DeclareUTFcomposite[\UTFencname]{x2091}{\textsubscript}{e}
\DeclareUTFcomposite[\UTFencname]{x2092}{\textsubscript}{o}
\DeclareUTFcomposite[\UTFencname]{x2093}{\textsubscript}{x}
\DeclareUTFcomposite[\UTFencname]{x2094}{\textsubscript}{\schwa}
\DeclareUTFcomposite[\UTFencname]{x2094}{\textsubscript}{\textschwa}

\DeclareUTFcharacter[\UTFencname]{x20A1}{\textcolonmonetary}
\DeclareUTFcharacter[\UTFencname]{x20A4}{\textlira}
\DeclareUTFcharacter[\UTFencname]{x20A6}{\textnaira}
\DeclareUTFcharacter[\UTFencname]{x20A9}{\textwon}
\DeclareUTFcharacter[\UTFencname]{x20AB}{\textdong}
\DeclareUTFcharacter[\UTFencname]{x20AC}{\texteuro}
\DeclareUTFcharacter[\UTFencname]{x20B1}{\textpeso}

\DeclareUTFcharacter[\UTFencname]{x2103}{\textcelsius}
\DeclareUTFcharacter[\UTFencname]{x210F}{\hbar}
\DeclareUTFcharacter[\UTFencname]{x2116}{\textnumero}
\DeclareUTFcharacter[\UTFencname]{x2117}{\textcircledP}
\DeclareUTFcharacter[\UTFencname]{x211E}{\textrecipe}
\DeclareUTFcharacter[\UTFencname]{x2120}{\textservicemark}
\DeclareUTFcharacter[\UTFencname]{x2122}{\texttrademark}
\DeclareUTFcharacter[\UTFencname]{x2126}{\textohm}
\DeclareUTFcharacter[\UTFencname]{x2127}{\textmho}
\DeclareUTFcharacter[\UTFencname]{x212E}{\textestimated}

%: separator

\DeclareUTFcharacter[\UTFencname]{x2190}{\textleftarrow}
\DeclareUTFcharacter[\UTFencname]{x2191}{\textuparrow}
\DeclareUTFcharacter[\UTFencname]{x2191}{\textupfullarrow} % ?? old IPA
\DeclareUTFcharacter[\UTFencname]{x2192}{\textrightarrow}
\DeclareUTFcharacter[\UTFencname]{x2193}{\textdownarrow}
\DeclareUTFcharacter[\UTFencname]{x2193}{\textdownfullarrow} % ?? old IPA
\DeclareUTFcharacter[\UTFencname]{x2194}{\textleftrightarrow}
\DeclareUTFcharacter[\UTFencname]{x2195}{\textupdownarrow}
\DeclareUTFcharacter[\UTFencname]{x2196}{\textnwarrow}
\DeclareUTFcharacter[\UTFencname]{x2196}{\textglobrise}
\DeclareUTFcharacter[\UTFencname]{x2197}{\textnearrow}
\DeclareUTFcharacter[\UTFencname]{x2198}{\textsearrow}
\DeclareUTFcharacter[\UTFencname]{x2198}{\textglobfall}
\DeclareUTFcharacter[\UTFencname]{x2199}{\textswarrow}

\DeclareUTFcharacter[\UTFencname]{x2212}{\textminus}
\DeclareUTFcharacter[\UTFencname]{x221A}{\textsurd}
\DeclareUTFcharacter[\UTFencname]{x2422}{\textblank}
\DeclareUTFcharacter[\UTFencname]{x2423}{\textvisiblespace}
\DeclareUTFcharacter[\UTFencname]{x25E6}{\textopenbullet}
\DeclareUTFcharacter[\UTFencname]{x2640}{\textfemale} % ???
\DeclareUTFcharacter[\UTFencname]{x266A}{\textmusicalnote}
\DeclareUTFcharacter[\UTFencname]{x26AD}{\textmarried}
\DeclareUTFcharacter[\UTFencname]{x26AE}{\textdivorced}
\DeclareUTFcharacter[\UTFencname]{x26B2}{\textuncrfemale} % neuter

% Latin Extended-C
\DeclareUTFcharacter[\UTFencname]{xF246}{\textlhookfour}% in Charis SIL, also  Ux2C70

% Latin Extended-D
\DeclareUTFcharacter[\UTFencname]{xA726}{\textcapitalheng}
\DeclareUTFcharacter[\UTFencname]{xA727}{\textheng}
\DeclareUTFcharacter[\UTFencname]{xA728}{\textcapitaltzlig}
\DeclareUTFcharacter[\UTFencname]{xA729}{\texttzlig}
\DeclareUTFcharacter[\UTFencname]{xA72A}{\textcapitaltresillo}
\DeclareUTFcharacter[\UTFencname]{xA72B}{\texttresillo}
\DeclareUTFcharacter[\UTFencname]{xA730}{\textscf}
\DeclareUTFcharacter[\UTFencname]{xA731}{\textscs}
\DeclareUTFcharacter[\UTFencname]{xA732}{\textcapitalaalig}
\DeclareUTFcharacter[\UTFencname]{xA733}{\textaalig}
\DeclareUTFcharacter[\UTFencname]{xA734}{\textcapitalaolig}
\DeclareUTFcharacter[\UTFencname]{xA735}{\textaolig}
\DeclareUTFcharacter[\UTFencname]{xA736}{\textcapitalaulig}
\DeclareUTFcharacter[\UTFencname]{xA737}{\textaulig}
\DeclareUTFcharacter[\UTFencname]{xA738}{\textcapitalavlig}
\DeclareUTFcharacter[\UTFencname]{xA739}{\textavlig}
\DeclareUTFcharacter[\UTFencname]{xA73A}{\textbarcapitalavlig}
\DeclareUTFcharacter[\UTFencname]{xA73B}{\textbaravlig}
\DeclareUTFcharacter[\UTFencname]{xA73C}{\textcapitalaylig}
\DeclareUTFcharacter[\UTFencname]{xA73D}{\textaylig}
\DeclareUTFcharacter[\UTFencname]{xA73E}{\textcapitalrevcdot}
\DeclareUTFcharacter[\UTFencname]{xA73F}{\textrevcdot}

\DeclareUTFcharacter[\UTFencname]{xA779}{\textcapitalinsd}
\DeclareUTFcharacter[\UTFencname]{xA77A}{\textinsd}
\DeclareUTFcharacter[\UTFencname]{xA77B}{\textcapitalinsf}
\DeclareUTFcharacter[\UTFencname]{xA77C}{\textinsf}
\DeclareUTFcharacter[\UTFencname]{xA77D}{\textcapitalinsg}
\DeclareUTFcharacter[\UTFencname]{xA77E}{\textcapitalturninsg}
\DeclareUTFcharacter[\UTFencname]{xA77F}{\textturninsg}
\DeclareUTFcharacter[\UTFencname]{xA780}{\textcapitalturnl}
\DeclareUTFcharacter[\UTFencname]{xA781}{\textturnl}
\DeclareUTFcharacter[\UTFencname]{xA782}{\textcapitalinsr}
\DeclareUTFcharacter[\UTFencname]{xA783}{\textinsr}
\DeclareUTFcharacter[\UTFencname]{xA784}{\textcapitalinss}
\DeclareUTFcharacter[\UTFencname]{xA785}{\textinss}
\DeclareUTFcharacter[\UTFencname]{xA786}{\textcapitalinst}
\DeclareUTFcharacter[\UTFencname]{xA787}{\textinst}
\DeclareUTFcharacter[\UTFencname]{xA78B}{\textcapitalsaltillo}
\DeclareUTFcharacter[\UTFencname]{xA78C}{\textsaltillo}


\endinput
%
  \makeatother}%