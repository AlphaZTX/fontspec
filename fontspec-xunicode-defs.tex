\ProvidesExplFile{fontspec-xunicode-defs.tex}{2015/10/01}{v0.1}{fontspec-xunicode definitions}

% Use \DeclareUTFcharacter to assign a cs-name to
% access a Unicode code-point...
\newcommand{\DeclareUTFcharacter}[3][\UTFencname]
  {
    \DeclareUTFcharacter:nnn {#1} {#2} {#3}
  }

% Use \DeclareUTFcomposite to assign a cs-name to access
% accents or composite characters via Unicode code-points,
% or the Unicode "Composing Character" mechanism ...
\newcommand{\DeclareUTFcomposite}[4][\UTFencname]
  {
    \cs_set:cpn
      { \cs_to_str:N #1 \string #3 - \string #4 }
      { \char #2 \relax }
  }
  
\newcommand{\DeclareEncodedCompositeCharacter}[4]
  %  #1 = encoding
  %  #2 = accent-macro in TeX
  %  #3 = position of combining glyph in Unicode
  %  #4 = bare accent position, in Unicode
  %  ##1 = slot for the accented letter
  {
    \expandafter\def\expandafter\next@ii\expandafter{
     \expandafter\expandafter\expandafter\@text@composite\expandafter
      \csname #1\string#2\endcsname####1\@empty
       \@text@composite{\add@UTF@accent{#3}{####1}{#4}}}
    \expandafter\def\expandafter\next@i\expandafter{\expandafter\expandafter
     \expandafter\def\expandafter\csname #1\string#2\endcsname####1}
    \expandafter\next@i\expandafter{\next@ii}
    % i tried simplifying the above (as below) and it didn't work :(
  }

\newcommand{\DeclareEncodedCompositeAccents}[4]
  {
    \cs_set_nopar:cpx { #1 \string #2 } ##1
      {
        \exp_not:n { \@text@composite }
        \exp_not:c { #1 \string #2 } ##1
        \exp_not:n { \@empty \@text@composite }
          { \add@UTF@accents {#4} {##1} {#3} } 
      }
  }

\cs_set:Npn \DeclareUTFcharacter:nnn #1#2#3
  {
     \ifcat \active\noexpand#3

       \cs_set:cpn { \UTFencname \string#3 } { \char#2\relax }
       
       \cs_if_free:NTF #3
        {
          \cs_if_eq:NNTF \cf@encoding \UTFencname
           { \DeclareTextCommand{#3}{OT1}{\undefined} }
           { \DeclareTextCommand{#3}{\cf@encoding}{\undefined} }
        }
        {
          \group_begin: % macro #3 exists already ...
            \let\protect\noexpand
            \edef\UTF@testi{#3}\def\UTF@testii{#3}%
            \cs_if_eq:NNTF \UTF@testi \UTF@testii
              { \group_insert_after:N \use_none:n }
              { \group_insert_after:N \use:n }
          \group_end:
          {
           % ... but when it isn't robust, make it so
           \cs_set_eq:cN {?-\string#3} #3
           \use:x
             {
               \exp_not:n
                 { \DeclareTextCommand {#3} }
               {\cf@encoding}
               {\expandafter\noexpand\csname?-\string#3\endcsname}
             }
          }
        }
      
     \else % not active catcode --- shouldn't happen
       % WSPR: actually happens once!!

       \cs_set:cpn
         {
           \expandafter\string\csname\UTFencname\endcsname
           \expandafter\string\csname#3\endcsname
         }
         {\char#2\relax}

       \cs_if_eq:NNTF \cf@encoding \UTFencname
         {
           \exp_args:Nc \DeclareTextCommand {#3}{OT1}{\undefined}
         }
         {
           \exp_args:Nc \DeclareTextCommand {#3}{\cf@encoding}{\undefined}
         }

     \fi % end of \ifcat     
  }

\cs_set_protected:Npn \add@UTF@accent  #1#2#3 { #2 \char "#1 \relax }
\cs_set_protected:Npn \add@UTF@accents #1#2#3 { #2 \char "#1 \char "#3 \relax}

\endinput
