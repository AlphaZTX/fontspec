
% \section{Extended font encodings}
%
% \iffalse
%    \begin{macrocode}
%<*fontspec&(xetexx|luatex)>
%    \end{macrocode}
% \fi

%    \begin{macrocode}

\providecommand\UnicodeFontFile[2]{"[#1]:#2"}
\providecommand\UnicodeFontName[2]{"#1:#2"}

\providecommand\add@unicode@accent[2]{#2\char#1\relax}
\providecommand\DeclareUnicodeAccent[3]{%
  \DeclareTextCommand{#1}{#2}{\add@unicode@accent{#3}}%
}

%    \end{macrocode}
%
%    \begin{macrocode}

\DeclareDocumentCommand\EncodingCommand{mO{}m}{%
  \bool_if:NF \l_@@_defining_encoding_bool { \ERROR }
  \DeclareTextCommand{#1}{\LastDeclaredEncoding}[#2]{#3}%
}

\def\EncodingAccent#1#2{%
  \bool_if:NF \l_@@_defining_encoding_bool { \ERROR }
  \DeclareTextCommand{#1}{\LastDeclaredEncoding}{\add@unicode@accent{#2}}%
}

\def\EncodingSymbol#1#2{%
  \bool_if:NF \l_@@_defining_encoding_bool { \ERROR }
  \DeclareTextSymbol{#1}{\LastDeclaredEncoding}{#2}%
}

\def\EncodingComposite#1#2#3{%
  \bool_if:NF \l_@@_defining_encoding_bool { \ERROR }
  \DeclareTextComposite{#1}{\LastDeclaredEncoding}{#2}{#3}%
}

\def\EncodingCompositeCommand#1#2#3{%
  \bool_if:NF \l_@@_defining_encoding_bool { \ERROR }
  \DeclareTextCompositeCommand{#1}{\LastDeclaredEncoding}{#2}{#3}%
}


%% COMMANDS FOR DEFINING NEW ENCODINGS FROM FONT RANGES

\cs_new:Nn \@@_new_unicode_encoding:n {%
  \DeclareFontEncoding{#1}{}{}
  \DeclareErrorFont{\LastDeclaredEncoding}{lmr}{m}{n}{10}
  \DeclareFontSubstitution{\LastDeclaredEncoding}{lmr}{m}{n}
  \DeclareFontFamily{\LastDeclaredEncoding}{lmr}{}
  \DeclareFontShape{\LastDeclaredEncoding}{lmr}{m}{n}
       {<->\UnicodeFontFile{lmroman10-regular}{\UnicodeFontTeXLigatures}}{}
  \DeclareFontShape{\LastDeclaredEncoding}{lmr}{m}{it}
       {<->\UnicodeFontFile{lmroman10-italic}{\UnicodeFontTeXLigatures}}{}
  \DeclareFontShape{\LastDeclaredEncoding}{lmr}{m}{sc}
       {<->\UnicodeFontFile{lmromancaps10-regular}{\UnicodeFontTeXLigatures}}{}
  \DeclareFontShape{\LastDeclaredEncoding}{lmr}{bx}{n}
       {<->\UnicodeFontFile{lmroman10-bold}{\UnicodeFontTeXLigatures}}{}
  \DeclareFontShape{\LastDeclaredEncoding}{lmr}{bx}{it}
       {<->\UnicodeFontFile{lmroman10-bolditalic}{\UnicodeFontTeXLigatures}}{}
}

\DeclareDocumentCommand \DeclareUnicodeEncoding {mm} {
  \@@_new_unicode_encoding:n {#1}
  \bool_set_true:N \l_@@_defining_encoding_bool
  #2
  \bool_set_false:N \l_@@_defining_encoding_bool
}

\DeclareDocumentCommand \ImportTU {} {
  \bool_if:NF \l_@@_defining_encoding_bool { \ERROR }
  \tl_set_eq:NN \l_@@_unicode_name_tl \UnicodeEncodingName
  \tl_set_eq:NN \UnicodeEncodingName \LastDeclaredEncoding
  \ProvidesFile{tuenc.def}
    [2015/12/31 v0.1 Unicode font encoding for LaTeX2e]


\providecommand\UnicodeEncodingName{TU}


%% WRAPPERS NEEDED FOR FONT LOADING (.fd FILE COMMANDS)

\begingroup\expandafter\expandafter\expandafter\endgroup
\expandafter\ifx\csname XeTeXrevision\endcsname\relax\else
  \def\UnicodeFontTeXLigatures{mapping=tex-text;}
\fi

\begingroup\expandafter\expandafter\expandafter\endgroup
\expandafter\ifx\csname directlua\endcsname\relax\else
  \def\UnicodeFontTeXLigatures{+tlig;} % "+trep;" no longer needed
\fi

\def\UnicodeFontFile#1#2{"[#1]:#2"}
\def\UnicodeFontName#1#2{"#1:#2"}

% Alternative luaotfload definitions for LuaLaTeX:
%    \def\UnicodeFontFile#1#2{"file:#1:#2"}
%    \def\UnicodeFontName#1#2{"name:#1:#2"}


%% SETTING UP ENCODING AND ERROR FONTS

\DeclareFontEncoding{\UnicodeEncodingName}{}{}
\DeclareErrorFont{\UnicodeEncodingName}{lmr}{m}{n}{10}
\DeclareFontSubstitution{\UnicodeEncodingName}{lmr}{m}{n}


%% COMMANDS FOR SYMBOL DEFINITIONS

\def\unicode@activate@licr#1#2{%
  \ifnum#1>127\relax
    \catcode#1=13\relax
    \begingroup
      \lccode`\~=#1\relax
      \lowercase{\endgroup\protected\def~}{#2}%
  \fi
}

% Accents in Unicode are postpended: 
\def\add@unicode@accent#1#2{#2\char#1\relax}

\def\DeclareUnicodeCommand#1#2{%
  \DeclareTextCommand{#1}{\UnicodeEncodingName}{#2}%
}

\def\DeclareUnicodeAccent#1#2{%
  \DeclareTextCommand{#1}{\UnicodeEncodingName}{\add@unicode@accent{#2}}%
}

\def\DeclareUnicodeSymbol#1#2{%
  \DeclareTextSymbol{#1}{\UnicodeEncodingName}{#2}%
%  \unicode@activate@licr{#2}{#1}%
}

\def\DeclareUnicodeComposite#1#2#3{%
  \DeclareTextComposite{#1}{\UnicodeEncodingName}{#2}{#3}%
%  \unicode@activate@licr{#3}{#1{#2}}%
}

\def\DeclareUnicodeCompositeCommand#1#2#3{%
  \DeclareTextCompositeCommand{#1}{\UnicodeEncodingName}{#2}{#3}%
}


%% COMMANDS FOR DEFINING NEW ENCODINGS FROM SUBSETS

\def\NewUnicodeEncoding#1{%
  \def\UnicodeEncodingName{#1}
  \DeclareFontEncoding{\UnicodeEncodingName}{}{}
  \DeclareErrorFont{\UnicodeEncodingName}{lmr}{m}{n}{10}
  \DeclareFontSubstitution{\UnicodeEncodingName}{lmr}{m}{n}
  \DeclareFontFamily{\UnicodeEncodingName}{lmr}{}
  \DeclareFontShape{\UnicodeEncodingName}{lmr}{m}{n}
       {<->\UnicodeFontFile{lmroman10-regular}{\UnicodeFontTeXLigatures}}{}
  \DeclareFontShape{\UnicodeEncodingName}{lmr}{m}{it}
       {<->\UnicodeFontFile{lmroman10-italic}{\UnicodeFontTeXLigatures}}{}
  \DeclareFontShape{\UnicodeEncodingName}{lmr}{m}{sc}
       {<->\UnicodeFontFile{lmromancaps10-regular}{\UnicodeFontTeXLigatures}}{}
  \DeclareFontShape{\UnicodeEncodingName}{lmr}{bx}{n}
       {<->\UnicodeFontFile{lmroman10-bold}{\UnicodeFontTeXLigatures}}{}
  \DeclareFontShape{\UnicodeEncodingName}{lmr}{bx}{it}
       {<->\UnicodeFontFile{lmroman10-bolditalic}{\UnicodeFontTeXLigatures}}{}
}

\def\DeclareEncodingFromSubsets#1#2#3{%
  \let\@@UnicodeEncodingName\UnicodeEncodingName
  \NewUnicodeEncoding{#1}%
  \@for\@ii:=#2\do{%
    \InputIfFileExists{ufontsub-\@ii.def}{}{%
      \@latex@error{Encoding subset file `ufontsub-\@ii.def' not found}{\@ehd}%
    }%
  }%
  #3%
  \let\UnicodeEncodingName\@@UnicodeEncodingName
}


%% CHECKING FONTS SUPPORT AN ENCODING

\def\@unicode@fontchar@test#1#2{%
  \iffontchar#1#2\relax\else
    \g@addto@macro\@unicode@font@encoding@check{#2}%
    \def\@unicode@fontchar@test##1##2{%
      \iffontchar##1##2\relax\else\g@addto@macro\@unicode@font@encoding@check{, ##2}\fi
    }
  \fi
}

\def\CheckFontSupportsSubsets#1#2{%
  \begingroup
    \gdef\@unicode@font@encoding@check{}
    \def\DeclareUnicodeCommand##1##2{}
    \def\DeclareUnicodeAccent##1##2{\@unicode@fontchar@test{#1}{##2}}
    \def\DeclareUnicodeSymbol##1##2{\@unicode@fontchar@test{#1}{##2}}
    \def\DeclareUnicodeComposite##1##2##3{\@unicode@fontchar@test{#1}{##3}}
    \def\DeclareUnicodeCompositeCommand##1##2##3{}
    \@for\@ii:=#2\do{%
      \InputIfFileExists{ufontsub-\@ii.def}{}{%
        \@latex@error{Encoding subset file `ufontsub-\@ii.def' not found}{\@ehd}%
      }%
    }%
    \ifx\@unicode@font@encoding@check\@empty\else
      \@latex@warning{^^JFont "\string#1" is missing the following chars from encoding subsets "#2":^^J\@unicode@font@encoding@check^^J}%
    \fi
  \endgroup
}

%% THE ENCODING ITSELF

\DeclareEncodingFromSubsets{TU}{T1,TS1}{}

\endinput

  \tl_set_eq:NN \UnicodeEncodingName \l_@@_unicode_name_tl
}

\DeclareDocumentCommand \ImportEncodingFile {m} {
  \bool_if:NF \l_@@_defining_encoding_bool { \ERROR }
  \clist_map_inline:nn {#1}
    {
      \InputIfFileExists{##1}{}{
        \@latex@error{Unicode~ encoding~ file~ `##1'~ not~ found}{\@ehd}%
      }
    }
}

% \DeclareUnicodeEncoding{user-legacy}{T1,TS1}{}

%    \end{macrocode}


% From 2e documenation:
% \begin{quote}
%    If you say:
%\begin{verbatim}
%    \DeclareTextCommand{\foo}{T1}...
%\end{verbatim}
%    then |\foo| is defined to be |\T1-cmd \foo \T1\foo|,
%    where |\T1\foo| is \emph{one} control sequence, not two!
% \end{quote}
%    \begin{macrocode}
\DeclareDocumentCommand \UndeclareSymbol {m}
  {
    \UndeclareTextCommand {#1} {\LastDeclaredEncoding}
  }


\DeclareDocumentCommand \UndeclareComposite {mm}
  {
    \cs_undefine:c
      { \c_backslash_str \LastDeclaredEncoding \token_to_str:N #1 - \tl_to_str:n {#2} }
  }
%    \end{macrocode}
%

% \iffalse
%    \begin{macrocode}
%</fontspec&(xetexx|luatex)>
%    \end{macrocode}
% \fi
