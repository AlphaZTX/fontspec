
% \section{Extended font encodings}
%
% \iffalse
%    \begin{macrocode}
%<*fontspec&(xetexx|luatex)>
%    \end{macrocode}
% \fi

%    \begin{macrocode}

\providecommand\UnicodeFontFile[2]{"[#1]:#2"}
\providecommand\UnicodeFontName[2]{"#1:#2"}

\providecommand\add@unicode@accent[2]{#2\char#1\relax}
\providecommand\DeclareUnicodeAccent[3]{%
  \DeclareTextCommand{#1}{#2}{\add@unicode@accent{#3}}%
}

%    \end{macrocode}
%
%    \begin{macrocode}

\def\EncodingCommand#1#2{%
  \bool_if:NF \l_@@_defining_encoding_bool { \ERROR }
  \DeclareTextCommand{#1}{\LastDeclaredEncoding}{#2}%
}

\def\EncodingAccent#1#2{%
  \bool_if:NF \l_@@_defining_encoding_bool { \ERROR }
  \DeclareTextCommand{#1}{\LastDeclaredEncoding}{\add@unicode@accent{#2}}%
}

\def\EncodingSymbol#1#2{%
  \bool_if:NF \l_@@_defining_encoding_bool { \ERROR }
  \DeclareTextSymbol{#1}{\LastDeclaredEncoding}{#2}%
}

\def\EncodingComposite#1#2#3{%
  \bool_if:NF \l_@@_defining_encoding_bool { \ERROR }
  \DeclareTextComposite{#1}{\LastDeclaredEncoding}{#2}{#3}%
}

\def\EncodingCompositeCommand#1#2#3{%
  \bool_if:NF \l_@@_defining_encoding_bool { \ERROR }
  \DeclareTextCompositeCommand{#1}{\LastDeclaredEncoding}{#2}{#3}%
}


%% COMMANDS FOR DEFINING NEW ENCODINGS FROM FONT RANGES

\cs_new:Nn \@@_new_unicode_encoding:n {%
  \DeclareFontEncoding{#1}{}{}
  \DeclareErrorFont{\LastDeclaredEncoding}{lmr}{m}{n}{10}
  \DeclareFontSubstitution{\LastDeclaredEncoding}{lmr}{m}{n}
  \DeclareFontFamily{\LastDeclaredEncoding}{lmr}{}
  \DeclareFontShape{\LastDeclaredEncoding}{lmr}{m}{n}
       {<->\UnicodeFontFile{lmroman10-regular}{\UnicodeFontTeXLigatures}}{}
  \DeclareFontShape{\LastDeclaredEncoding}{lmr}{m}{it}
       {<->\UnicodeFontFile{lmroman10-italic}{\UnicodeFontTeXLigatures}}{}
  \DeclareFontShape{\LastDeclaredEncoding}{lmr}{m}{sc}
       {<->\UnicodeFontFile{lmromancaps10-regular}{\UnicodeFontTeXLigatures}}{}
  \DeclareFontShape{\LastDeclaredEncoding}{lmr}{bx}{n}
       {<->\UnicodeFontFile{lmroman10-bold}{\UnicodeFontTeXLigatures}}{}
  \DeclareFontShape{\LastDeclaredEncoding}{lmr}{bx}{it}
       {<->\UnicodeFontFile{lmroman10-bolditalic}{\UnicodeFontTeXLigatures}}{}
}

\DeclareDocumentCommand \DeclareUnicodeEncoding {mmm} {
  \@@_new_unicode_encoding:n {#1}
  \bool_set_true:N \l_@@_defining_encoding_bool
  \clist_map_inline:nn {#2}
    {
      \InputIfFileExists{fontrange-##1.def}{}{
        \@latex@error{Unicode font range file `fontrange-\@ii.def' not found}{\@ehd}%
      }
    }
  #3
  \bool_set_false:N \l_@@_defining_encoding_bool
}


% \DeclareUnicodeEncoding{user-legacy}{T1,TS1}{}

%    \end{macrocode}


% From 2e documenation:
% \begin{quote}
%    If you say:
%\begin{verbatim}
%    \DeclareTextCommand{\foo}{T1}...
%\end{verbatim}
%    then |\foo| is defined to be |\T1-cmd \foo \T1\foo|,
%    where |\T1\foo| is \emph{one} control sequence, not two!
% \end{quote}
%    \begin{macrocode}
\providecommand\UndeclareSymbol[1]
  {
    \cs_undefine:c { \LastDeclaredEncoding \token_to_str:N #1 }
  }
%    \end{macrocode}
%

% \iffalse
%    \begin{macrocode}
%</fontspec&(xetexx|luatex)>
%    \end{macrocode}
% \fi
