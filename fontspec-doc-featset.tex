

\part{Selecting font features}
\label{sec:selectingfeature}

The commands discussed so far such as \cs{fontspec} each take an optional argument for
accessing the font features of the requested font.
Commands are provided to set default features to be applied for all fonts, and even to change the features that a font is presently loaded with.
Different font shapes can be loaded with separate features, and different features can even be selected for different sizes that the font appears in.
This part discusses these options.

\section{Default settings}
\label{sec:defaults}

\cmdbox{\cmd\defaultfontfeatures\marg{font features}}

It is sometimes useful to define
font features that are applied to every subsequent font selection command.
This may be defined with the
\cmd{\defaultfontfeatures} command, shown in \exref{dff}.
New calls of \cs{defaultfontfeatures} overwrite previous ones, and defaults can be reset by calling the command with an empty argument.

\begin{Xexample}{dff}{A demonstration of the \cs{defaultfontfeatures} command.}
  \fontspec{texgyreadventor-regular.otf}
  Some default text 0123456789 \\
  \defaultfontfeatures{
     Numbers=OldStyle, Color=888888
  }
  \fontspec{texgyreadventor-regular.otf}
  Now grey, with old-style figures:
  0123456789
\end{Xexample}

\cmdbox{\cmd\defaultfontfeatures\oarg{font name}\marg{font features}}

Default font features can be specified on a per-font and per-face basis
by using the optional argument to \cs{defaultfontfeatures} as shown.
\begin{Verbatim}
  \defaultfontfeatures[texgyreadventor-regular.otf]{Color=blue}
  \setmainfont{texgyreadventor-regular.otf}% will be blue
\end{Verbatim}
Multiple fonts may be affected by using a comma separated list of font names.

\cmdbox{\cmd\defaultfontfeatures\oarg{\cs{font-switch}}\marg{font features}}

\textbf{New in v2.4}.
Defaults can also be applied to symbolic families such as those created with the |\newfontfamily| command and for |\rmfamily|, |\sffamily|, and |\ttfamily|:
\begin{Verbatim}
  \defaultfontfeatures[\rmfamily,\sffamily]{Ligatures=TeX}
  \setmainfont{texgyreadventor-regular.otf}% will use standard TeX ligatures
\end{Verbatim}
The line above to set \TeX-like ligatures is now activated by \emph{default} in \texttt{fontspec.cfg}.
To reset default font features, simply call the command with an empty argument:
\begin{Verbatim}
  \defaultfontfeatures[\rmfamily,\sffamily]{}
  \setmainfont{texgyreadventor-regular.otf}% will no longer use standard TeX ligatures
\end{Verbatim}

\cmdbox{\cmd\defaultfontfeatures\texttt{+}\marg{font features}\\
        \cmd\defaultfontfeatures\texttt{+}\oarg{font name}\marg{font features}}

\textbf{New in v2.4}.
Using the |+| form of the command appends the \meta{font features} to any already-selected defaults.


\section{Working with the currently selected features}
\label{sec:addfontfeatures}


\cmdbox{\cmd\IfFontFeatureActiveTF\marg{font feature}\marg{true code}\marg{false code}}

This command queries the currently selected font face and executes the appropriate branch based on whether the \meta{font feature} as specified by \pkg{fontspec} is currently active.

For example, the following will print `True':
\begin{Verbatim}
\setmainfont{texgyrepagella-regular.otf}[Numbers=OldStyle]
\IfFontFeatureActiveTF{Numbers=OldStyle}{True}{False}
\end{Verbatim}

Note that there is no way for \pkg{fontspec} to know what the default features of a font will be. For example, by default the |texgyrepagella| fonts use lining numbers. But in the following example, querying for lining numbers returns false since they have not been explicitly requested:
\begin{Verbatim}
\setmainfont{texgyrepagella-regular.otf}
\IfFontFeatureActiveTF{Numbers=Lining}{True}{False}
\end{Verbatim}

Please note: At time of writing this function only supports OpenType fonts; AAT/Graphite fonts under the \XeTeX\ engine are not supported.


\cmdbox{\cmd\addfontfeatures\marg{font features}}

This command allows font features to
be changed without knowing what features are currently selected or even what
font is being used. A good example of this could be to add a hook to all
tabular material to use monospaced numbers, as shown in \exref{aff}.
If you attempt to \emph{change} an already-selected feature, \pkg{fontspec} will try to de-activate any features that clash with the new ones.
\Eg, the following two invocations are mutually exclusive:
\begin{Verbatim}
\addfontfeature{Numbers=OldStyle}...
\addfontfeature{Numbers=Lining}...
123
\end{Verbatim}
Since |Numbers=Lining| comes last, it takes precedence and deactivates the call |Numbers=OldStyle|.

\begin{Lexample}{aff}{A demonstration of the \cs{addfontfeatures} command.}
  \fontspec{texgyreadventor-regular.otf}%
           [Numbers={Proportional,OldStyle}]
  `In 1842, 999 people sailed 97 miles in
   13 boats. In 1923, 111 people sailed 54
   miles in 56 boats.'            \bigskip

  {\addfontfeatures{Numbers={Monospaced,Lining}}
  \begin{tabular}{@{} cccc @{}}
            Year & People & Miles & Boats \\
    \hline  1842 &  999   &  75   &  13   \\
            1923 &  111   &  54   &  56
  \end{tabular}}
\end{Lexample}

\DescribeMacro{\addfontfeature}
This command may also be executed under the alias \cmd{\addfontfeature}.


\subsection{Priority of feature selection}
Features defined with \cs{addfontfeatures} override features
specified by \cs{fontspec}, which in turn override features
specified by \cs{defaultfontfeatures}.  If in doubt, whenever a
new font is chosen for the first time, an entry is made in the
transcript (\texttt{.log}) file displaying the font name and the
features requested.


\section{Different features for different font shapes}
\label{sec:bfit-feat}

\cmdbox{
 \feat{BoldFeatures}\texttt=\marg{features} \\
 \feat{ItalicFeatures}\texttt=\marg{features} \\
 \feat{BoldItalicFeatures}\texttt=\marg{features} \\
 \feat{SlantedFeatures}\texttt=\marg{features} \\
 \feat{BoldSlantedFeatures}\texttt=\marg{features} \\
 \feat{SmallCapsFeatures}\texttt=\marg{features}
}

It is entirely possible that separate fonts in a family will require
separate options; \eg, Hoefler Text Italic contains various swash
feature options that are completely unavailable in the upright shapes.

The font features defined at the top level of the optional \cmd\fontspec\
argument are applied to \emph{all} shapes of the family.
Using \feat{Upright-}, \feat{SmallCaps-}, \feat{Bold-},
\feat{Italic-}, and \feat{BoldItalicFeatures},
separate font features may be defined to their respective shapes
\emph{in addition} to, and with precedence over, the `global' font features.
See \exref{itfeat}.

\begin{Xexample}{itfeat}{Features for, say, just italics.}
\fontspec{EBGaramond12-Regular.otf}%
  [ItalicFont=EBGaramond12-Italic.otf]
\itshape Don’t Ask Victoria! \\
\addfontfeature{ItalicFeatures={Style=Swash}}
Don’t Ask Victoria! \\
\end{Xexample}

Note that because most fonts include their small caps glyphs
within the main font, features specified with \feat{SmallCapsFeatures} are applied \emph{in addition} to
any other shape-specific features as defined above, and hence \feat{SmallCapsFeatures}
can be nested within \feat{ItalicFeatures} and friends. Every combination
of upright, italic, bold and small caps can thus be assigned individual
features, as shown in the somewhat ludicrous \exref{scfeat}.

\begin{Xexample}{scfeat}{An example of setting the \feat{SmallCapsFeatures}
separately for each font shape.}
  \fontspec{texgyretermes}[
      Extension = {.otf},
      UprightFont = {*-regular}, ItalicFont = {*-italic},
      BoldFont = {*-bold}, BoldItalicFont = {*-bolditalic},
      UprightFeatures={Color = 220022,
           SmallCapsFeatures = {Color=115511}},
       ItalicFeatures={Color = 2244FF,
           SmallCapsFeatures = {Color=112299}},
         BoldFeatures={Color = FF4422,
           SmallCapsFeatures = {Color=992211}},
   BoldItalicFeatures={Color = 888844,
           SmallCapsFeatures = {Color=444422}},
           ]
  Upright {\scshape Small Caps}\\
  \itshape Italic {\scshape Italic Small Caps}\\
  \upshape\bfseries Bold {\scshape Bold Small Caps}\\
  \itshape Bold Italic {\scshape Bold Italic Small Caps}
\end{Xexample}


\section{Selecting fonts from TrueType Collections (TTC files)}
TrueType Collections are multiple fonts contained within a single file.
Each font within a collection must be explicitly chosen using the \feat{FontIndex} command.
Since TrueType Collections are often used to contain the italic/bold shapes in a family, \pkg{fontspec} automatically selects the italic, bold, and bold italic fontfaces from the same file.
For example, to load the macOS system font Optima:
\begin{Verbatim}
\setmainfont{Optima.ttc}[
  Path = /System/Library/Fonts/ ,
  UprightFeatures    = {FontIndex=0} ,
  BoldFeatures       = {FontIndex=1} ,
  ItalicFeatures     = {FontIndex=2} ,
  BoldItalicFeatures = {FontIndex=3} ,
]
\end{Verbatim}
Support for TrueType Collections has only been tested in \XeTeX, but should also work with an up-to-date version of \LuaTeX\ and the \pkg{luaotfload} package.


\section{Different features for different font sizes}
\label{sec:sizefeature}

\cmdbox{
\ttfamily SizeFeatures = \char`\{\\
\null\quad...\\
\null\quad\char`\{~Size =
\rmfamily\meta{size range}\ttfamily
,
\rmfamily \meta{font features}\ttfamily
~\char`\} , \\
\null\quad\char`\{~Size =
\rmfamily\meta{size range}\ttfamily
, Font =
\rmfamily\meta{font name}\texttt, \meta{font features}\ttfamily
~\char`\} , \\
\null\quad... \\
\char`\}}

The \feat{SizeFeature} feature is a little more complicated
than the previous features discussed. It allows different fonts
and different font features to be selected for a given font
family as the point size varies.

It takes a comma separated list of braced, comma separated lists of features for each size range.
Each sub-list must contain the \opt{Size} option
to declare the size range, and optionally \opt{Font} to change the
font based on size. Other (regular) fontspec features that are added
are used on top of the font features that would be used anyway.
A demonstration to clarify these details is shown in \exref{sizefeat}.
A less trivial example is shown in the context of optical font sizes
in \vref{sec:opticalsize}.

\begin{Xexample}{sizefeat}{An example of specifying different font features for different sizes of font with \feat{SizeFeatures}.}
  \fontspec{texgyrechorus-mediumitalic.otf}[
    SizeFeatures={
      {Size={-8}, Font=texgyrebonum-italic.otf, Color=AA0000},
      {Size={8-14}, Color=00AA00},
      {Size={14-}, Color=0000AA}} ]

  {\scriptsize Small\par} Normal size\par {\Large Large\par}
\end{Xexample}

To be precise, the \opt{Size} sub-feature accepts arguments in the form shown in \vref{tab:sizing}.
Braces around the size range are optional. For an exact font size (|Size=X|)
font sizes chosen near that size will `snap'. For example, for size definitions
at exactly 11pt and 14pt, if a 12pt font is requested \emph{actually} the
11pt font will be selected. This is a remnant of the past when fonts were designed
in metal (at obviously rigid sizes) and later when bitmap fonts were similarly
designed for fixed sizes.

If additional features are only required for a single size, the other sizes
must still be specified.  As in:
\begin{Verbatim}
  SizeFeatures={
     {Size=-10,Numbers=Uppercase},
     {Size=10-}}
\end{Verbatim}
Otherwise, the font sizes greater than 10 won't be defined at all!

\begin{table}
\caption{Syntax for specifying the size to apply custom font features.}\label{tab:sizing}
\centering
\begin{tabular}{@{}ll@{}}
\toprule
Input & Font size, $s$ \\
\midrule
 |Size = X-| & $s \geq \texttt{X}$ \\
 |Size = -Y| & $s < \texttt{Y}$ \\
 |Size = X-Y| & $\texttt{X} \leq s < \texttt{Y}$ \\
 |Size = X| & $s = \texttt{X}$ \\
\bottomrule
\end{tabular}
\end{table}

\paragraph{Interaction with other features}
For \feat{SizeFeatures} to work with \feat{ItalicFeatures}, \feat{BoldFeatures}, etc., and \feat{SmallCapsFeatures}, a strict heirarchy is required:
\begin{Verbatim}
 UprightFeatures =
  {
   SizeFeatures =
    {
     {
      Size = -10,
      Font = ..., % if necessary
      SmallCapsFeatures = {...},
      ... % other features for this size range
     },
     ... % other size ranges
    }
  }
\end{Verbatim}
Suggestions on simplifying this interface welcome.


\section{Font independent options}
\label{sec:font-ind-features}

Features introduced in this section may be used with any font.

\subsection{Colour}

\feat{Color} (or \feat{Colour}) uses font specifications to set the colour of
the text.
You should think of this as the literal glyphs of the font being coloured in a certain way.
Notably, this mechanism is different to that of the \pkg{color}/\pkg{xcolor}/\pkg{hyperref}/etc.\ packages, and in fact using \pkg{fontspec} commands to set colour will prevent your text from changing colour using those packages at all!
For example, if you set the colour in a \verb|\setmainfont| command, \verb|\color{...}| and related commands, including hyperlink colouring, will no longer have any effect on text in this font.)
Therefore, \pkg{fontspec}'s colour commands are best used to set explicit colours in specific situations, and the \pkg{xcolor} package is recommended for more general colour functionality.

The colour is defined as a triplet of two-digit Hex RGB
values, with optionally another value for the transparency (where
|00| is completely transparent and |FF| is opaque.)
\begin{Lexample}{color}{Selecting colour with transparency.}
  \fontsize{48}{48}
  \fontspec{texgyrebonum-bold.otf}
  {\addfontfeature{Color=FF000099}W}\kern-0.4ex
  {\addfontfeature{Color=0000FF99}S}\kern-0.4ex
  {\addfontfeature{Color=DDBB2299}P}\kern-0.5ex
  {\addfontfeature{Color=00BB3399}R}
\end{Lexample}
Transparency is supported by \LuaLaTeX; \XeLaTeX\ with the \texttt{xdvipdfmx} driver
does not support this feature.

If you load the \pkg{xcolor} package, you may use any named colour instead
of writing the colours in hexadecimal.
\begin{Verbatim}
 \usepackage{xcolor}
 ...
 \fontspec[Color=red]{Verdana} ...
 \definecolor{Foo}{rgb}{0.3,0.4,0.5}
 \fontspec[Color=Foo]{Verdana} ...
\end{Verbatim}
The \pkg{color} package is \emph{not} supported; use \pkg{xcolor} instead.

You may specify the transparency with a named colour using the \feat{Opacity}
feature which takes an decimal from zero to one corresponding to
transparent to opaque respectively:
\begin{Verbatim}
 \fontspec[Color=red,Opacity=0.7]{Verdana} ...
\end{Verbatim}
It is still possible to specify a colour in six-char hexadecimal form
while defining opacity in this way, if you like.

\subsection{Scale}

\cmdbox{
 \feat{Scale} = \meta{number} \\
 \feat{Scale} = \opt{MatchLowercase} \\
 \feat{Scale} = \opt{MatchUppercase}
}

In its explicit form, \feat{Scale} takes a single
numeric argument for linearly scaling the font, as demonstrated
in \exref{fontload}.
It is now possible to
measure the correct dimensions of the fonts loaded and
calculate values to scale them automatically.

As well as a numerical argument, the \feat{Scale} feature
also accepts options \opt{MatchLowercase}
and \opt{MatchUppercase}, which will scale the font being selected to match
the current default roman font to either the height of the lowercase or
uppercase letters, respectively; these features are shown in \exref{scale}.

\begin{Xexample}{scale}{Automatically calculated scale values.}
  \setmainfont{Georgia}
  \newfontfamily\lc[Scale=MatchLowercase]{Verdana}
   The perfect match {\lc is hard to find.}\\
  \newfontfamily\uc[Scale=MatchUppercase]{Arial}
   L O G O \uc F O N T
\end{Xexample}

The amount of scaling used in each instance is reported in the \texttt{.log} file.
Since there is some subjectivity about the exact scaling to be used, these values
should be used to fine-tune the results.

Note that when |Scale=MatchLowercase| is used with |\setmainfont|, the new `main' font of the document will be scaled to match the old default.
This may be undesirable in some cases, so to achieve `natural' scaling for the main font but automatically scale all other fonts selected, you may write
\begin{Verbatim}
  \defaultfontfeatures{ Scale = MatchLowercase }
  \defaultfontfeatures[\rmfamily]{ Scale = 1}
\end{Verbatim}
One or both of these lines may be placed into a local |fontspec.cfg| file (see \vref{sec:config}) for this behaviour to be effected in your own documents automatically.
(Also see \vref{sec:defaults} for more information on setting font defaults.)



\subsection{Interword space}

While the space between words can be varied on an individual
basis with the \TeX\ primitive \cmd\spaceskip\ command, it is
more convenient to specify this information when the font is
first defined.

The space in between words in a paragraph will be chosen automatically,
and generally will not need to be adjusted. For those
times when the precise details are important, the \feat{WordSpace}
feature is
provided, which takes either a single scaling factor to scale the
default value, or a triplet of comma-separated
values to scale the nominal value, the stretch, and the shrink of the
interword space by, respectively. (|WordSpace={|$x$|}| is the same as
|WordSpace={|$x$|,|$x$|,|$x$|}|.)

\begingroup
\let\centering\relax
\begin{Xexample}{wordspace}{Scaling the default interword space. An exaggerated value has been chosen to emphasise the effects here.}
  \fontspec{texgyretermes-regular.otf}
  Some text for our example to take
  up some space, and to demonstrate
  the default interword space.
  \bigskip

  \fontspec{texgyretermes-regular.otf}%
    [WordSpace = 0.3]
  Some text for our example to take
  up some space, and to demonstrate
  the default interword space.
\end{Xexample}
\endgroup

Note that \TeX's optimisations in how it loads fonts means that you cannot
use this feature in \cs{addfontfeatures}.

\subsection{Post-punctuation space}

If \cmd\frenchspacing\ is \emph{not} in effect, \TeX\ will allow extra
space after some punctuation in its goal of justifying the lines of text.
Generally, this is considered old-fashioned, but occasionally in small amounts the
effect can be justified, pardon the pun.

The \feat{PunctuationSpace} feature takes a scaling factor by which to
adjust the nominal value chosen for the font; this is demonstrated in
\exref{punctspace}.
Note that |PunctuationSpace=0|
is \emph{not} equivalent to \cmd\frenchspacing, although the difference
will only be apparent when a line of text is under-full.

\begin{Lexample}{punctspace}{Scaling the default post-punctuation space.}
  \nonfrenchspacing
  \fontspec{texgyreschola-regular.otf}
   Letters, Words. Sentences.          \par
  \fontspec{texgyreschola-regular.otf}[PunctuationSpace=2]
   Letters, Words. Sentences.          \par
  \fontspec{texgyreschola-regular.otf}[PunctuationSpace=0]
   Letters, Words. Sentences.
\end{Lexample}

Note that \TeX's optimisations in how it loads fonts means that you cannot
use this feature in \cs{addfontfeatures}.



\subsection{The hyphenation character}

The letter used for hyphenation may be chosen with the \feat{HyphenChar}
feature.
With one exception (\feat{HyphenChar} \texttt{=} \opt{None}),
this is a \XeTeX-only feature since \LuaTeX\ cannot set the hyphenation character on a per-font basis;
see its \cs{prehyphenchar} primitive for further details.

\feat{HyphenChar} takes three types of input, which are chosen according to some
simple rules. If the input is the string \opt{None}, then hyphenation is
suppressed for this font.
If the input is a single character, then this character is used.
Finally, if the input is longer than a single character
it must be the UTF-8 slot number of the hyphen character you desire.

This package redefines \LaTeX's \cmd\-\ macro such that it adjusts along with the above changes.

\begin{Xexample}{hyphchar}{Explicitly choosing the hyphenation character.}
 \def\text{\fbox{\parbox{1.55cm}{%
   EXAMPLE HYPHENATION%
 }}\qquad\qquad\null\par\bigskip}

 \fontspec{LinLibertine_R.otf}[HyphenChar=None]
 \text
 \fontspec{LinLibertine_R.otf}[HyphenChar={+}]
 \text
\end{Xexample}

Note that \TeX's optimisations in how it loads fonts means that you cannot
use this feature in \cs{addfontfeatures}.

\subsection{Optical font sizes} \label{sec:opticalsize}

Optically scaled fonts thicken out as the font size decreases
in order to make the glyph shapes more robust (less prone to losing
detail), which improves legibility. Conversely, at large optical
sizes the serifs and other small details may be more delicately
rendered.

OpenType fonts with optical scaling will exist in
several discrete sizes, and these will be selected by \XeTeX\
and Lua\TeX\
\emph{automatically} determined by the current font size as in
\exref{optsize}, in which we've scaled down some large text in order to be
able to compare the difference for equivalent font sizes.

The
\feat{OpticalSize} feature may be used to specify a different optical
size.
With \feat{OpticalSize} set
to zero, no optical size font substitution is performed, as shown in
\exref{optsize0}.

\begin{Xexample}{optsize}{A demonstration of automatic optical size selection.}
  \fontspec{Latin Modern Roman}
   Automatic optical size                  \\
  \scalebox{0.4}{\Huge
   Automatic optical size}
\end{Xexample}

\begin{Xexample}{optsize0}{Optical size substitution is suppressed when set to zero.}
  \fontspec{Latin Modern Roman 5 Regular}[OpticalSize=0]
   Latin Modern optical sizes                \\
  \fontspec{Latin Modern Roman 8 Regular}[OpticalSize=0]
   Latin Modern optical sizes                \\
  \fontspec{Latin Modern Roman 12 Regular}[OpticalSize=0]
   Latin Modern optical sizes                \\
  \fontspec{Latin Modern Roman 17 Regular}[OpticalSize=0]
   Latin Modern optical sizes
\end{Xexample}

The \feat{SizeFeatures} feature (\vref*{sec:sizefeature}) can be
used to specify exactly which optical sizes will be used for ranges
of font size. For example, something like:
\begin{Verbatim}
  \fontspec{Latin Modern Roman}[
    UprightFeatures = { SizeFeatures = {
      {Size=-10,     OpticalSize=8 },
      {Size= 10-14,  OpticalSize=10},
      {Size= 14-18,  OpticalSize=14},
      {Size=    18-, OpticalSize=18}}}
           ]
\end{Verbatim}

\subsection{Font transformations}

In rare situations users may want to mechanically distort the shapes of the glyphs in the current font such as shown in \exref{fake}. Please don't overuse these features; they are \emph{not} a good alternative to having the real shapes.

\begin{Xexample}{fake}{Articifial font transformations.}
  \fontspec{Quattrocento.otf} \emph{ABCxyz} \quad
  \fontspec{Quattrocento.otf}[FakeSlant=0.2] ABCxyz

  \fontspec{Quattrocento.otf}  ABCxyz \quad
  \fontspec{Quattrocento.otf}[FakeStretch=1.2] ABCxyz

  \fontspec{Quattrocento.otf} \textbf{ABCxyz} \quad
  \fontspec{Quattrocento.otf}[FakeBold=1.5] ABCxyz
\end{Xexample}

If values are omitted, their defaults are as shown above.

If you want the bold shape to be faked automatically, or the italic shape
to be slanted automatically, use the \feat{AutoFakeBold} and
\feat{AutoFakeSlant} features. For example, the following two invocations
are equivalent:
\begin{Verbatim}
  \fontspec[AutoFakeBold=1.5]{Charis SIL}
  \fontspec[BoldFeatures={FakeBold=1.5}]{Charis SIL}
\end{Verbatim}
If both of the \feat{AutoFake...} features are used, then the bold italic
font will also be faked.

The \feat{FakeBold} and \feat{AutoFakeBold} features are only available with the \XeTeX\ engine and will be ignored in \LuaTeX.


\subsection{Letter spacing}
Letter spacing, or tracking, is the term given to adding (or subtracting) a small amount of horizontal space in between adjacent characters. It is specified with the \feat{LetterSpace}, which takes a numeric argument,
shown in \exref{tracking}.

The letter spacing parameter is a normalised additive factor (not a scaling factor); it is defined as a percentage of the font size. That is, for a 10\,pt font, a letter spacing parameter of `|1.0|' will add 0.1\,pt between each letter.

\begin{Xexample}{tracking}{The \feat{LetterSpace} feature.}
  \fontspec{Didot}
  \addfontfeature{LetterSpace=0.0}
  USE TRACKING FOR DISPLAY CAPS TEXT \\
  \addfontfeature{LetterSpace=2.0}
  USE TRACKING FOR DISPLAY CAPS TEXT
\end{Xexample}

This functionality is not generally used for lowercase text in modern typesetting but does have historic precedent in a variety of situations.
In particular, small amounts of letter spacing can be very useful, when setting small caps or all caps titles.
Also see the OpenType \opt{Uppercase} option of the \feat{Letters} feature (\vref*{sec:letters}).


% /©
%
% (COPYRIGHT GOES HERE)
%
% ©/
