
\part{Getting started}

\section{History}

This package began life as a \LaTeX\ interface to select system-installed
\MacOSX\ fonts in \name{Jonathan Kew}'s \XeTeX, the first widely-used
Unicode extension to \TeX. Over time, \XeTeX\ was extended to support OpenType
fonts and then was ported into a cross-platform program to run also on Windows
and Linux.

More recently, \LuaTeX\ is fast becoming the \TeX\ engine of the day; it
supports Unicode encodings and OpenType fonts and opens up the internals of
\TeX\ via the Lua programming language. Hans Hagen's Con\TeX t Mk.\,IV is a
re-write of his powerful typesetting system, taking full advantage of
\LuaTeX's features including font support; a kernel of his work in this area
has been extracted to be useful for other \TeX\ macro systems as well, and
this has enabled \pkg{fontspec} to be adapted for \LaTeX\ when run with the
\LuaTeX\ engine.

\section{Introduction}

The \pkg{fontspec} package allows users of either \XeTeX\ or \LuaTeX\ to
load OpenType fonts in a \LaTeX\ document. No font installation is necessary,
and font features can be selected and used as desired throughout the document.

Without \pkg{fontspec}, it is necessary to write cumbersome font definition
files for \LaTeX, since \LaTeX's font selection scheme (known as the
`\textsc{nfss}') has a lot going on behind the scenes to allow easy
commands like \cmd\emph\ or \cmd\bfseries. With an uncountable number of
fonts now available for use, however, it becomes less desirable to have to
write these font definition (|.fd|) files for every font one wishes to use.

Because \pkg{fontspec} is designed to work in a variety of modes, this
user documentation is split into separate sections that are designed to be
relatively independent. Nonetheless, the basic functionality all behaves in
the same way, so previous users of \pkg{fontspec} under \XeTeX\ should have
little or no difficulty switching over to \LuaTeX.

This manual can get rather in-depth, as there are a lot of details
to cover. See the documents \path{fontspec-example.tex} for a complete minimal example
to get started quickly.


\subsection{Acknowledgements}

This package could not have been possible without the early and continued support
the author of \XeTeX, Jonathan Kew. When I started this package, he steered
me many times in the right direction.

I've had great
feedback over the years on feature requests, documentation queries, bug reports, font suggestions, and so on from lots of people all around the world.
Many thanks to you all.

Thanks to David Perry and Markus B\"ohning for numerous documentation
improvements and David Perry again for contributing the text for one of the
sections of this manual.

Special thanks to Khaled Hosny, who was the driving force behind the support for \LuaLaTeX, ultimately leading to version 2.0 of the package.

\section{Package loading and options}

For basic use, no package options are required:
\begin{Verbatim}
  \usepackage{fontspec}
\end{Verbatim}
Package options will be introduced below; some preliminary details are discussed first.

\paragraph{Font encodings}
\null\marginpar{\null\hfill\fbox{\color{red}\textsc{update}!}}%
The 2016 release of \pkg{fontspec} initiated some changes for font encodings and the loading of \pkg{xunicode}.

A new package option, \texttt{tuenc}, which is selected by default, switches the \textsc{nfss} font encoding to \texttt{TU}.
\texttt{TU} is a new Unicode font encoding, intended for both \XeTeX\ and \LuaTeX\ engines, and automatically contains support for symbols covered by \LaTeX's traditional \texttt{T1} and \texttt{TS1} font encodings (for example, |\%|, |\textbullet|, |\"u|, and so on).
As a result, with this package option, Ross Moore's \pkg{xunicode} package is \textbf{not} loaded.

The old behaviour can be achieved by loading the \texttt{euenc} package option. This selects the \texttt{EU1} or \texttt{EU2} encoding (\XeTeX/\LuaTeX, resp.) and loads the \pkg{xunicode} package.
Package authors and users who have referred explicitly to the encoding names \texttt{EU1} or \texttt{EU2} should update their code or documents.
(See internal variable names described in \vref{sec:api} for how to do this properly.)


\paragraph{\LuaTeX\ users only}
In order to load fonts by their name rather than by their filename (\eg,
`Latin Modern Roman' instead of `ec-lmr10'), you may need to run the script
\texttt{luaotfload-tool}, which is distributed with the \pkg{luaotfload}
package. Note that if you do not execute this script beforehand, the first
time you attempt to typeset the process will pause for (up to) several
minutes. (But only the first time.)
Please see the \pkg{luaotfload} documentation for more information.


\paragraph{\pkg{babel}}
\emph{The \pkg{babel} package is only supported for certain languages.}
Especially Vietnamese, Greek, and Hebrew at least might not work correctly, as far as I can tell.
There's a better chance with Cyrillic and Latin-based languages, however---\pkg{fontspec} ensures at least that fonts should load correctly.
The \pkg{polyglossia} package is recommended instead as a modern replacement for \pkg{babel}.



\subsection{Maths fonts adjustments}
By default, \pkg{fontspec} adjusts \LaTeX's default maths setup in order to maintain the correct Computer Modern symbols when the roman font changes.
However, it will attempt to avoid doing this if another maths font package is loaded (such as \pkg{mathpazo} or the \pkg{unicode-math} package).

If you find that \pkg{fontspec} is incorrectly changing the maths font when it shouldn't be, apply the |no-math| package option to manually suppress its selection of the maths fonts.

\subsection{Configuration}
\label{sec:config}

If you wish to customise any part of the
\pkg{fontspec} interface, this should be done by creating your own
\texttt{fontspec.cfg} file,
which will be automatically loaded if it is found by \XeTeX\ or \LuaTeX.
A |fontspec.cfg| file is distributed with \pkg{fontspec} with a small number of defaults set up within it.

To customise \pkg{fontspec} to your liking, use the standard |.cfg| file as a starting point or write your own from scratch, then either place it in the same folder as the main document for isolated cases, or in a location
that \XeTeX\ or \LuaTeX\ searches by default; \eg\ in Mac\TeX: \path{~/Library/texmf/tex/latex/}.

The package option |no-config| will suppress the loading of the |fontspec.cfg| file under all circumstances.

\subsection{Warnings}
\label{sec:quiet-warnings}

This package can give some warnings that can be harmless if you know what
you're doing. Use the |quiet| package option to write these warnings to the
transcript (\texttt{.log}) file instead.

Use the |silent| package option to completely suppress these warnings if you
don't even want the |.log| file cluttered up.


