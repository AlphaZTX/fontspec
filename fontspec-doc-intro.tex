%%^^A%%  fontspec-doc-intro.tex -- part of FONTSPEC <wspr.io/fontspec>

\part{Getting started}

\section{History}

This package began life as a \LaTeX\ interface to select system-installed
\MacOSX\ fonts in \name{Jonathan Kew}'s \XeTeX, the first widely-used
Unicode extension to \TeX. Over time, \XeTeX\ was extended to support OpenType
fonts and then was ported into a cross-platform program to run also on Windows
and Linux.

More recently, \LuaTeX\ is fast becoming the \TeX\ engine of the day; it
supports Unicode encodings and OpenType fonts and opens up the internals of
\TeX\ via the Lua programming language. Hans Hagen's Con\TeX t Mk.\,IV is a
re-write of his powerful typesetting system, taking full advantage of
\LuaTeX's features including font support; a kernel of his work in this area
has been extracted to be useful for other \TeX\ macro systems as well, and
this has enabled \pkg{fontspec} to be adapted for \LaTeX\ when run with the
\LuaTeX\ engine.



\section{Introduction}

The \pkg{fontspec} package allows users of either \XeTeX\ or \LuaTeX\ to
load OpenType fonts in a \LaTeX\ document. No font installation is necessary,
and font features can be selected and used as desired throughout the document.

Without \pkg{fontspec}, it is necessary to write cumbersome font definition
files for \LaTeX, since \LaTeX's font selection scheme (known as the
`\textsc{nfss}') has a lot going on behind the scenes to allow easy
commands like \cmd\emph\ or \cmd\bfseries. With an uncountable number of
fonts now available for use, however, it becomes less desirable to have to
write these font definition (|.fd|) files for every font one wishes to use.

Because \pkg{fontspec} is designed to work in a variety of modes, this
user documentation is split into separate sections that are designed to be
relatively independent. Nonetheless, the basic functionality all behaves in
the same way, so previous users of \pkg{fontspec} under \XeTeX\ should have
little or no difficulty switching over to \LuaTeX.

This manual can get rather in-depth, as there are a lot of details
to cover. See the documents \path{fontspec-example.tex} for a complete minimal example
to get started quickly.


\subsection{Acknowledgements}

This package could not have been possible without the early and continued support
the author of \XeTeX, Jonathan Kew. When I started this package, he steered
me many times in the right direction.

I've had great
feedback over the years on feature requests, documentation queries, bug reports, font suggestions, and so on from lots of people all around the world.
Many thanks to you all.

Thanks to David Perry and Markus B\"ohning for numerous documentation
improvements and David Perry again for contributing the text for one of the
sections of this manual.

Special thanks to Khaled Hosny, who was the driving force behind the support for \LuaLaTeX, ultimately leading to version 2.0 of the package.



\section{Package loading and options}

For basic use, no package options are required:
\begin{Verbatim}
  \usepackage{fontspec}
\end{Verbatim}
Package options will be introduced below; some preliminary details are discussed first.


\subsection{Font encodings}
The 2016 release of \pkg{fontspec} initiated some changes for font encodings and the loading of \pkg{xunicode}.
The 2017 release rolls out those changes as default.

The now-default \texttt{tuenc} package option switches the \textsc{nfss} font encoding to \texttt{TU}.
\texttt{TU} is a new Unicode font encoding, intended for both \XeTeX\ and \LuaTeX\ engines, and automatically contains support for symbols covered by \LaTeX's traditional \texttt{T1} and \texttt{TS1} font encodings (for example, |\%|, |\textbullet|, |\"u|, and so on).
As a result, with this package option, Ross Moore's \pkg{xunicode} package is \textbf{not} loaded.
Some new, experimental, features are now provided to customise some encoding details; see Part~\vref{part:enc} for further details.

Pre-2017 behaviour can be achieved with the \texttt{euenc} package option.
This selects the \texttt{EU1} or \texttt{EU2} encoding (\XeTeX/\LuaTeX, resp.) and loads the \pkg{xunicode} package.
Package authors and users who have referred explicitly to the encoding names \texttt{EU1} or \texttt{EU2} should update their code or documents.
(See internal variable names described in \vref{sec:api} for how to do this properly.)


\subsection{Maths fonts adjustments}
By default, \pkg{fontspec} adjusts \LaTeX's default maths setup in order to maintain the correct Computer Modern symbols when the roman font changes.
However, it will attempt to avoid doing this if another maths font package is loaded (such as \pkg{mathpazo} or the \pkg{unicode-math} package).

If you find that \pkg{fontspec} is incorrectly changing the maths font when it shouldn't be, apply the |no-math| package option to manually suppress its behaviour here.


\subsection{Configuration}
\label{sec:config}

If you wish to customise any part of the
\pkg{fontspec} interface, this should be done by creating your own
\texttt{fontspec.cfg} file,
which will be automatically loaded if it is found by \XeTeX\ or \LuaTeX.
A |fontspec.cfg| file is distributed with \pkg{fontspec} with a small number of defaults set up within it.

To customise \pkg{fontspec} to your liking, use the standard |.cfg| file as a starting point or write your own from scratch, then either place it in the same folder as the main document for isolated cases, or in a location
that \XeTeX\ or \LuaTeX\ searches by default; \eg\ in Mac\TeX: \path{~/Library/texmf/tex/latex/}.

The package option |no-config| will suppress the loading of the |fontspec.cfg| file under all circumstances.


\subsection{Warnings}
\label{sec:quiet-warnings}

This package can give some warnings that can be harmless if you know what
you're doing. Use the |quiet| package option to write these warnings to the
transcript (\texttt{.log}) file instead.

Use the |silent| package option to completely suppress these warnings if you
don't even want the |.log| file cluttered up.



\section{Interaction with \LaTeXe\ and other packages}

This section documents some areas of adjustment that \pkg{fontspec} makes
to improve default behaviour with \LaTeXe\ and third-party packages.


\subsection{Verbatim}
\label{sec:verb}

Many verbatim mechanisms assume the existence of a `visible space' character that exists in the \textsc{ascii} space slot of the typewriter font. This character is known in Unicode as \unichar{2423}{box open}, which looks like this: `\verb*| |'.

When a Unicode typewriter font is used, \LaTeX\ no longer prints visible spaces for the |verbatim*| environment and |\verb*| command.
This problem is fixed by using the correct Unicode glyph, and the following packages are patched to do the same:
\pkg{listings}, \pkg{fancyvrb}, \pkg{moreverb}, and \pkg{verbatim}.

In the case that the typewriter font does not contain `\verb*| |', the Latin Modern Mono font is used as a fallback.


\subsection{Discretionary hyphenation: \cmd\-}
\label{sec:hyphen}

\DescribeMacro{\-}
\LaTeX\ defines the macro \cmd\-\ to insert discretionary hyphenation points.
However, it is hard-coded in \LaTeX\ to use the hyphen |-| character.
Since \pkg{fontspec} provides features to change the hyphenation character on
a per font basis, the definition of \cmd\-\ is changed to adapt accordingly.


\subsection{Commands for old-style and lining numbers}

\DescribeMacro{\oldstylenums}
\DescribeMacro{\liningnums}
\LaTeX's definition of \cs{oldstylenums} relies on strange font encodings.
We provide a \pkg{fontspec}-compatible alternative and while we're at it
also throw in the reverse option as well. Use \cs{oldstylenums}\marg{text}
to explicitly use old-style (or lowercase) numbers in \meta{text}, and
the reverse for \cs{liningnums}\marg{text}.


\subsection{Italic small caps}

\DescribeMacro{\itshape}
\DescribeMacro{\slshape}
\DescribeMacro{\scshape}
Note that this package redefines the \cs{itshape}, \cs{slshape}, and \cs{scshape} commands in order to allow them to select italic small caps in conjunction.
With these changes, writing |\itshape\scshape| will lead to italic small caps, and |\upshape| subsequently then moves back to small caps only. |\upshape| again returns from small caps to upright regular.
(And similarly for for |\slshape|. In addition, once italic small caps are selected then |\slshape| will switch to slanted small caps, and vice versa.)


\subsection{Emphasis and nested emphasis}

\DescribeMacro{\eminnershape}
\LaTeXe\ allows you to specify the behaviour of \cs{emph} nested within \cs{emph} by setting the \cs{eminnershape} command.
For example,
\begin{Verbatim}
  \renewcommand\eminnershape{\upshape\scshape}
\end{Verbatim}
will produce small caps within |\emph{\emph{...}}|.

\DescribeMacro{\emfontdeclare}
The \pkg{fontspec} package takes this idea one step further to allow arbitrary font shape changes and arbitrary levels of nesting within emphasis.
This is performed using the |\emfontdeclare| command, which takes a comma-separated list of font switches corresponding to increasing levels of emphasis.
An example:
\begin{enumerate}
\item |\emfontdeclare{\itshape,\upshape\scshape,\itshape}| will lead to `italics', `small caps', then `italic small caps' as the level of emphasis increases, as long as italic small caps are defined for the font.
  Note that |\upshape| is required because the font changes are cascading.
\end{enumerate}
The implementation of this feature tries to be `smart' and guess what level of emphasis to use in the case of manual font changing.
This is reliable only if you use shape-changing commands in \cs{emfontdeclare}.
For example:
\begin{Verbatim}
    \emfontdeclare{\itshape,\upshape\scshape,\itshape}
    ...
    \scshape small caps \emph{hello}
\end{Verbatim}
Here, the emphasised text `hello' will be printed in italic small caps since |\emph| can detect that the current font shape is already in the second `mode' of emphasis.

\DescribeMacro{\emreset}
Finally, if you have so much nested emphasis that |\emfontdeclare| runs out of options, it will insert |\emreset| (by default just |\upshape|) and start again from the beginning.


\subsection{Strong emphasis}

\DescribeMacro{\strong}
\DescribeMacro{\strongenv}
The \cs{strong} macro is used analogously to \cs{emph} but produces variations in weight.
If you need it in environment form, use |\begin{strongenv}...\end{strongenv}|.

As with emphasis, this font-switching command is intended to move through a range
of font weights. For example, if the fonts are set up correctly it allows usage such as
|\strong{...\strong{...}}| in which each nested \cs{strong} macro increases the
weight of the font.

\DescribeMacro{\strongfontdeclare}
Currently this feature set is somewhat experimental and there is no syntactic sugar
to easily define a range of font weights using \pkg{fontspec} commands.
Use, say, the following to define first bold and then black (|k|) font faces for \cs{strong}:
\begin{Verbatim}
  \strongfontdeclare{\bfseries,\fontseries{k}\selectfont}
\end{Verbatim}

\DescribeMacro{\strongreset}
If too many levels of \cs{strong} are reached, \cs{strongreset} is inserted.
By default this is a no-op and the font will simply remain the same.
Use \cs{renewcommand}\cs{strongreset}|{\mdseries}| to start again from the beginning if desired.

An example for setting up a font family for use with \cs{strong} is discussed in \vref{sec:strong-example}.


% /©
% ------------------------------------------------
% The FONTSPEC package  <wspr.io/fontspec>
% ------------------------------------------------
% Copyright  2004-2017  Will Robertson, LPPL "maintainer"
% Copyright  2009-2013  Khaled Hosny
% ------------------------------------------------
% This package is free software and may be redistributed and/or modified under
% the conditions of the LaTeX Project Public License, version 1.3c or higher
% (your choice): <http://www.latex-project.org/lppl/>.
% ------------------------------------------------
% ©/
